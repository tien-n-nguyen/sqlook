\subsection{Row-based Slicing across PHP and SQL}

\begin{figure}[tbp]
  \centering
  \includegraphics[width=0.95\columnwidth]{images/Execution-Trace-1.pdf}\\
  \caption{Execution trace of row-based test cases}\label{fig:Execution-Trace}
\end{figure}



During the above monitoring process, {\tool} also records the PHP
statements and the parts (\code{True} and \code{False}) of
\code{WHERE} clause(s) that are exercised in the execution of a
row-based test case, thus, involved in the output of a data row. It
computes program slices corresponding to each data row.
%To gather more information about the fault, \tool{} considers each
%table row as a test case and computes the program slice corresponding
%to each test case. 
Figures~\ref{fig:Execution-Trace} and ~\ref{fig:Rowbased-Slicing}
illustrate the computation of the slices for the example in
Fig.~\ref{fig:example-phpcode} (with lines 5' and 7 added for
illustrative purposes). Fig.~\ref{fig:Execution-Trace} shows the
trace in the execution of our instrumented interpreter. The
nodes denote the PHP and SQL entities, and the arrows show the order
in the execution trace. Besides PHP statements, {\tool} needs to
consider two parts of a \code{WHERE} clause (see lines 4a and 4b).
%\tool{} introduces a new type of program
%entity, namely the state of the \query{WHERE} condition in an SQL
%query (lines 4a and 4b).
We use \code{1-Alice}, \code{2-Bob}, \code{3-Carol}, and
\code{4-Daniel} to denote the data records in the result set \code{T}
after step 2 in Fig.~\ref{fig:Instrumented-Code}, i.e., after
executing the modified SQL query.

As a PHP entity or SQL part is executed, \tool{} recognizes the
corresponding processed data row and includes that entity/part into
the slice of that row-based test case with its order in the execution
trace. For example, since the \query{WHERE} expression evaluates to
\code{True} for \code{Bob} and \code{Carol}, line 4a is included in
the slices of the test cases corresponding to those data rows
(see Fig.~\ref{fig:Rowbased-Slicing}). Similarly, line 4b is included in
the slices for \code{1-Alice} and \code{4-Daniel}, corresponding to
\code{WhereExp = False}. Since only \code{2-Bob} and \code{3-Carol}
are returned in the query's result, the PHP code on lines 5-6 performs
two iterations to print out each name. Whenever a PHP row-retrieving
function such as \code{mysql\_fetch\_array} is executed and returns a
\emph{non-}\code{null} table row (line 5), \tool{} includes 
in the slice of the corresponding row-based test case the PHP
statements having data dependencies with that row. Thus, lines 5, 5',
and 6 are included in \code{2-Bob}'s and \code{3-Carol}'s slices.
%the slices for \code{2-Bob} and \code{3-Carol}.
%executed twice (for \code{2-Bob} and \code{3-Carol})
%the statements that follow in the slice of the corresponding row-based
%test case. Therefore, statements 5-6 are included in the slices of
%\code{Bob} and \code{Carol}.
As line~5~is executed for \code{1-Alice} and
\code{4-Daniel}, the variable \code{\$row} is \code{null} and the
execution exits the \code{while} loop; i.e., no table row is accessed.
Thus, lines 5' and 6 are not included for \code{1-Alice}'s and
\code{4-Daniel}'s slices, while line 5 is included for all slices.~Since 
line 7 (or any next line) is exercised regardless of
data rows, it will be included in all slices of all test cases.
Similarly, the PHP statements before line 4 are exercised and included
in all slices.

%When a statement is not associated with any particular test case
%(e.g., statements 4, 7, and statement 5 in its last execution),
%\tool{} includes them in the slices of all test cases. The slices for
%these row-based test cases are shown in
%Figure~\ref{fig:Rowbased-Slicing}.

%tbp
\begin{figure}[t]
  \centering
  \includegraphics[width=0.81\columnwidth]{images/Row-based-Slicing-1.pdf}\\ %0.84
  \caption{Row-based slicing across PHP and SQL}\label{fig:Rowbased-Slicing}
\end{figure}
