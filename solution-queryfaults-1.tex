\subsection{Monitoring the Execution of SQL Queries}



Figure~\ref{fig:SQL-Transformed} illustrates the execution of a PHP
program with SQL queries by a regular PHP interpreter. It contains the
source code to evaluate various expressions in a PHP program, e.g.
assignments, variables, function calls, etc. Among them, we focus on
the evaluation of the PHP
\code{mysql\_query} statement where database interactions take place
(the shaded part in the PHP interpreter's source code). The
interpreter evaluates the \code{mysql\_query} statement by sending the
SQL query to the database management system (DBMS) and retrieving its
returned result. Since the actual query execution is performed by the
DBMS, the interpreter does not have access to the internal operations
that evaluate the query's \query{WHERE} condition and extract data
records from database. Therefore, to monitor the evaluation of the
expressions in the
\query{WHERE} clause at run time, we instrument the original PHP
interpreter and replace the source code handling database queries with
our instrumented code that performs the query's operations
(Figure~\ref{fig:SQL-Transformed}).



%By evaluating the \query{WHERE} condition instead of delegating the evaluation to the DBMS, the instrumented interpreter now has the evaluation results of the \query{WHERE} expression for individual data records.

The operations that fulfill an SQL \query{SELECT} query consist of the
following: (a) retrieving data from one or more database tables
specified by the \query{FROM} part of the SQL query, (b) extracting
the data records that satisfy the criteria specified by the
\query{WHERE} condition of the SQL query, and (c) projecting the set
of columns (attributes) given in the \query{SELECT} part of query into
the final result set. As an example, the SQL query \code{SELECT}
\code{Name FROM Users WHERE Age $>$= 25 AND Gender = `Female' OR
Country $<>$ `USA'} retrieves the names of all the users from the
\code{`Users'} database table that meet the condition \code{Age $>$=
  25 AND Gender = `Female' OR Country $<>$ `USA'}.


Let us explain the instrumentation for our monitoring
process. Our instrumented interpreter re-implements these three
operations in four steps (shown in the instrumented code in
Figure~\ref{fig:SQL-Transformed}). The detailed instrumentation
execution is illustrated in Figure~\ref{fig:Instrumented-Code}. (In
our implementation, we instrument Quercus
(http://quercus.caucho.\-com/), a PHP interpreter, and use JSqlParser
(http://jsqlparser.source\-forge.\-net/) to parse SQL code. Specifically,
we have the following steps:

%; both are written in Java.)

%~\footnote{In our implementation, \tool{} instruments
%Quercus (http://quercus.caucho.com/), a PHP interpreter, and uses
%JSqlParser (http://jsqlparser.sourceforge.net/) to parse SQL queries;
%both are written in Java.}

\begin{figure*}[tbp]
  \centering
  \includegraphics[width=4.55in]{images/Instrumented-Code-2.eps}\\
  \caption{Instrumented code to monitor the execution of SQL queries~\cite{icsm13}}\label{fig:Instrumented-Code}
\end{figure*}

%\begin{figure*}[tbp]
%  \centering
%  \includegraphics[width=0.8\textwidth]{images/Instrumented-Code-1.eps}\\
%  \caption{Instrumented code to monitor the execution of SQL queries}\label{fig:Instrumented-Code}
%\end{figure*}

\vspace{0.04in}
\textbf{Step 1. Modifying an SQL query:}
The original SQL query is parsed and modified. Its
\query{WHERE} part is removed, and~the column set is changed into \code{`*'}
(i.e., all columns will be retrieved). The \code{ModifiedSql} variable
now contains the modified SQL query, which retrieves all data from one
or more database tables (the SQL query may specify a \query{JOIN}
operation on multiple tables). The \query{WHERE} clause and the column
set in the original SQL query are also extracted out (to the variables
\code{WhereExp} and \code{SelectedCols}, respectively) so that the
rows and columns can be filtered from the modified query's result in
the next steps. In Figure~\ref{fig:Instrumented-Code}, the dashed box
at the top shows an example of an SQL query that is the input of step
1. The values of the three variables \code{ModifiedSql},
\code{WhereExp}, and \code{SelectedCols} are shown in the dashed boxes
coming out of step 1, respectively.

% and will be used to illustrate the following steps.



\vspace{0.04in}
\textbf{Step 2. Executing the modified SQL query:}
The modified SQL query obtained from step 1 will be sent to the DBMS
to be executed there. Compared to the result of the original query,
the result of the modified query contains all data from the database
table(s) in which the rows and columns have not been filtered out
according to the specifications in the original query. In
Figure~\ref{fig:Instrumented-Code}, the dashed box coming out of step
2 (table \code{T}) shows the result after executing the
\code{ModifiedSql} query.

\vspace{0.04in}
\textbf{Step 3. Filtering rows:} In this step, \tool{} loops through
each row \code{R} in table \code{T} and evaluates the expression
\code{WhereExp} in the \query{WHERE} clause of the original SQL query~for row \code{R}.
If a row \code{R} satisfies the condition specified
by \code{WhereExp}, it will be extracted out, in the same way that a
DBMS processes the original SQL query. In
Figure~\ref{fig:Instrumented-Code}, the extracted rows
(\code{FilteredRows}) are the rows 2 and 3 of the \code{Users}
table. Importantly, by evaluating the individual predicates in the
\query{WHERE} expression, \tool{} is able to determine whether the
\query{WHERE} condition evaluates to \code{True} or \code{False} for a
given row.

\vspace{0.04in}
\textbf{Step 4. Filtering columns:} In the final step, \tool{}
extracts the column set specified by the original SQL query from the
data obtained from step 3. In Figure~\ref{fig:Instrumented-Code}, the
final result consists of the names \code{Bob} and \code{Carol},
which preserves the original query's result as if it was executed by
the DBMS.

%\vspace{0.06in}
%\noindent {\bf Evaluation Rules.}

\subsection{Evaluation Rules for Step 3}

Since SQL is a declarative language, when evaluating the SQL
expressions in step 3, \tool{} needs~to understand and implement their
semantics in the interpreter. Table~\ref{tab:transformation-rules}
shows the rules to evaluate an SQL expression $E$ with the data from a
given table row~$R$.

\begin{table}[t]
    \centering 
    %\scriptsize %\setlength{\tabcolsep}{3pt}
    \small
    \caption{Evaluation Rules for SQL expressions}\label{tab:transformation-rules}
\begin{tabular}{@{}cll@{}}
    \toprule
    \textbf{No.} & \textbf{SQL Expression E} & \textbf{Eval. Result on E for Row R} \\
    \midrule
    1.  & E ::= E1 AND E2     &   E(R) $\leftarrow$ E1(R) \&\& E2(R) \\
    \midrule
    2.  & E ::= E1 OR E2      &   E(R) $\leftarrow$ E1(R) $||$ E2(R) \\
    \midrule
    3.  & E ::= E1 BETWEEN    &   E(R) $\leftarrow$ E2(R) $<=$ E1(R)  \\
        & \hspace{30pt} E2 AND E3       & \hspace{30pt} \&\& E1(R) $<=$ E3(R) \\
    \midrule
    4.  & E ::= E' LIKE               &   $Pattern$.replace(\{`\%', `.*'\}, \{`\_', `.'\}) \\
        & \hspace{30pt} $Pattern$     &   E(R) $\leftarrow$ match($Pattern$, E'(R)) \\
    \midrule
    5.  & E ::= $Table$ AS $Alias$   & $Alias$.RealName $\leftarrow$ $Table$ \\
    \midrule
    6.  & E ::= $Table$.$Column$      &   if $Table$ is alias then \\
        &                               &   \hspace{15pt}$Table$ $\leftarrow$ $Table$.RealName \\
        &                              & E(R) $\leftarrow$ R.getCol($Table$.$Column$) \\
    \midrule
    7.  & E ::= E' IN (\{$E_i$\}),    &   E(R) $\leftarrow$ \{$E_i(R)$\}.contains(E'(R)) \\
        & \hspace{30pt} $i = 1 \ldots n$ \\
    \midrule
    8.  & E ::= EXISTS E'             &   E(R) $\leftarrow$ E'(R) != null \\
    \midrule
    9. & E ::= $SubSelect$          &   Send $SubSelect$ to DBMS \\
       &                            &   Store returned result in E(R) \\
    \bottomrule
\end{tabular}
\end{table}


\textbf{Rules 1 and 2:}
\code{E ::= E1 AND/OR E2}. For an SQL logic expression (e.g. \code{AND}
and \code{OR}), \tool{} evaluates the two sub-expressions
\code{E1} and \code{E2}, and returns the result based on the semantics
of the operator in the expression \code{E}. Other arithmetic
expressions such as addition, subtraction, and comparison expressions
(e.g., \code{`='} and \code{`$>$='}) are realized in a similar
manner.

\textbf{Rule 3:} \code{E ::= E1 BETWEEN E2 AND E3}. \tool{}
evaluates three expressions \code{E1}, \code{E2}, \code{E3}, and
checks the condition that the value of \code{E1(R)} is in the
range of \code{E2(R)} and \code{E3(R)}.

\textbf{Rule 4:} \code{E ::= E' LIKE $Pattern$}. \tool{}
implements the SQL regular expression $Pattern$ by determining if the
evaluation result from \code{E'(R)} matches that pattern.

\textbf{Rule 5:} \code{E ::= $Table$ AS $Alias$}. In an SQL query,
database tables can be referred to via aliases (e.g., the SQL query
\code{SELECT * FROM Users AS u WHERE u.Age $>=$ 25} assigns the alias
name \code{`u'} for the \code{Users} table). Thus, \tool{}
records the real table name of the alias for later reference.

\textbf{Rule 6:} \code{E ::= $Table$.$Column$}. For an SQL expression
to access a table column (e.g., \code{`u.Age'}), \tool{} first checks
if the table name is an alias name, in which case the real table name
is used. Then, \tool{} retrieves the corresponding table column from
the current row \code{R}. If the expression does not specify a table
name, the only table name in the query will be used.

%For example, given SQL query in the motivating example and the row
%\code{(1, Alice, 20, Female, USA)}, the evaluation result for column
%\code{Name} is \code{Alice}.

\textbf{Rule 7:} \code{E ::= E' IN (\{$E_i$\}), $i = 1 \ldots n$}.
\tool{} evaluates \code{E'} and the set of expressions \{$E_i$\}
and determines if the evaluation result of \code{E'} is contained
in the evaluation results of \{$E_i$\}. Note that \{$E_i$\} can be
another SQL \query{SELECT} query.

\textbf{Rule 8:} \code{E ::= EXISTS E'}. \tool{} evaluates \code{E'}
and checks if the result is non-empty (not null). Similar to
rule~7, \code{E'} is often an SQL sub-query.

% which is handled by the next evaluation rule.

\textbf{Rule 9:} \code{E ::= $SubSelect$}. For an expression specifying
a sub-query, \tool{} sends the sub-query to the DBMS to be
executed and retrieves its returned result. It monitors only the
extraction of table rows from the top-level query and transfers the
executions of all sub-queries (if any) to the DBMS.

%Although the SQL query in the motivating example is simple for
%illustrative purposes,

\tool{} is also able to handle more complex SQL
constructs such as \query{JOIN}, \query{ORDER BY}, \query{TOP},
%\query{ALL}, \query{ANY} and SQL functions such as \query{MAX},
\query{LEN}, and \query{UCASE}. It performs these operations by
either re-implementing them or delegating them to the DBMS (step 2 of
Figure~\ref{fig:Instrumented-Code} or rule 9 in Table
\ref{tab:transformation-rules}).

% Currently, \tool{} does not handle \query{UNION}. 
