\subsection{Localizing a Faulty Predicate}

%s with a Single Fault}

\begin{table}
    \footnotesize
    \setlength{\tabcolsep}{1.5pt}
    \centering
    \caption{\tool{}'s fault localization using predicate switching}\label{tab:predicate-table}
\begin{tabular}{clllcccc}
  \addlinespace
  \toprule
  \textbf{Row} & \textbf{P1} & \textbf{P2} & \textbf{P3} & \textbf{P1 AND } & \textbf{Act.} & \textbf{Exp.} & \textbf{Passed/} \\
  &    &   &   & {\bf P2 OR P3}  &  {\bf Output} & {\bf Output} & {\bf Failed}\\ 
  \midrule
  1 & False ($\times$) & True & False ($\times$) & False & (none) & Alice & Failed \\
  2 & False & False & True ($\times$) & True & Bob & (none) & Failed \\
  3 & True & True & True & True & Carol & Carol & Passed \\
  4 & True & False ($\times$) & False ($\times$) & False & (none) & Daniel & Failed \\
  \midrule
  $R$ score & 1 & 1 & 3 \\
  \bottomrule
  \addlinespace
  \multicolumn{8}{l}{P1: Age $>$= 25 \hspace{1em} P2: Gender = `Female' \hspace{1em} P3: Country $<>$ `USA'}
\end{tabular}
\end{table}

%\begin{table*}
%    \footnotesize
%    \centering
%    \caption{Illustration of \tool{}'s fault localization using predicate switching}\label{tab:predicate-table}
%\begin{tabular}{clllcccc}
%  \addlinespace
%  \toprule
%  \textbf{Row} & \textbf{P1} & \textbf{P2} & \textbf{P3} & \textbf{P1 AND P2 OR P3} & \textbf{Act. Output} & \textbf{Exp. Output} & \textbf{Passed/Failed} \\
%  \midrule
%  1 & False ($\times$) & True & False ($\times$) & False & (none) & Alice & Failed \\
%  2 & False & False & True ($\times$) & True & Bob & (none) & Failed \\
%  3 & True & True & True & True & Carol & Carol & Passed \\
%  4 & True & False ($\times$) & False ($\times$) & False & (none) & Daniel & Failed \\
%  \midrule
%  $R$ score & 1 & 1 & 3 \\
%  \bottomrule
%  \addlinespace
%  \multicolumn{8}{l}{P1: Age $>$= 25 \hspace{1em} P2: Gender = `Female' \hspace{1em} P3: Country $<>$ `USA'}
%\end{tabular}
%\end{table*}

Let us explain our technique to localize a faulty predicate and
then our generalization to support the localizing of multiple faulty
ones.  The technique for localizing a single faulty predicate is used
only internally by {\tool}, while the general algorithm is for localizing
any numbers of faulty predicates. Table~\ref{tab:predicate-table}
(called a \emph{predicate table}) illustrates our predicate switching
technique to localize the faulty predicate for the running example.
% in Section 2.
%, assuming that there is \emph{a single fault} in the SQL query. 
The predicate table contains the evaluation results of individual
(atomic) predicates in the \query{WHERE} clause. This information is
recorded at run time as the instrumented PHP interpreter evaluates the
predicates in the
\query{WHERE} condition (see Section 3.1). In
Table~\ref{tab:predicate-table}, the three predicates \code{Age $>$=
25}, \code{Gender = `Female'}, and \code{Country $<>$ `USA'} are denoted
by \code{P1}, \code{P2}, and {P3}, respectively. As seen, the
\query{WHERE} expression (\code{P1 AND P2 OR P3}) for rows 2 and 3
evaluates to \code{True}, thus the execution returns
\code{Bob} and \code{Carol} (column \code{Act. Output}). In contrast,
the \query{WHERE} expression for rows 1 and 4 evaluates to
\code{False}, thus \code{Alice} and \code{Daniel} are not included in
the returned result. The columns \code{Exp. Output} and
\code{Passed/Failed} show the expected output and the passed/failed
status of the corresponding row-based test case.

%If the actual output (column \code{Act. Output}) is the same as the expected output (column \code{Exp. Output}), the corresponding row (test case) will be marked as \code{Passed}; otherwise, it will be marked as \code{Failed} (column \code{Passed/Failed}).

Given the table row of a \emph{failed} test case, \tool{} attempts to
switch the current boolean value for one predicate at a time and
re-evaluates the \query{WHERE} clause with the predicate's new
value. If the value of the \query{WHERE} clause changes (from
\code{True} to \code{False} or vice versa), meaning that the given row
is now a \emph{passed} test case, \tool{} records this event to
compute the likelihood that the predicate contains a fault. In
Table~\ref{tab:predicate-table}, those predicates and their
corresponding rows are marked with the notation $\times$. In the first
row, if either \code{P1} or \code{P3} is switched from \code{False} to
\code{True}, the value of the
\query{WHERE} condition \code{P1 AND P2 OR P3} will change to
\code{True}, and therefore \code{Alice} will be output as expected. 
In contrast, changing the value of \code{P2} from \code{True} to
\code{False} does not affect the original output result. After
applying predicate switching to all predicates and table rows,
\tool{} computes the suspiciousness score $R$ for a predicate by
summing up the total of times its switched value changes a failed test
case~to a passed one. In Table~\ref{tab:predicate-table}, \code{P3}'s
$R$ score is 3 whereas \code{P1} and \code{P2}'s scores are 1.
Thus, the predicate having the highest suspiciousness score
(\code{P3}) is the one likely containing the fault.

%Therefore, the real fault at the predicate \code{P3} can be localized using the \tool{}'s predicate switching technique.

%\begin{figure}[t]
%    \centering
%\begin{lstlisting} [
%    emph={foreach, in, if, then},
%	mathescape=true,
%    xleftmargin=11pt
%]
%$R(P_i) \leftarrow 0, \forall $ Predicate $P_i$ in $WhereExp$
%foreach Failed Test Case $R$
%    foreach Predicate $P_i$ in $WhereExp$
%        $P_i(R)$.switchBooleanValue() // $P_i(R)$ = NOT $P_i(R)$
%        result $\leftarrow$ evaluateExp($WhereExp$, $R$)
%        if result = ExpectedResult($WhereExp$, $R$) then
%            $R(P_i)$ = $R(P_i)$ + 1
%        $P_i(R)$.restoreBooleanValue()
%\end{lstlisting}
%    \caption{Predicate switching algorithm to localize single predicate faults}\label{fig:algorithm-singfaults}
%\end{figure}
%
%Figure~\ref{fig:algorithm-singfaults} summarizes \tool{}'s predicate switching technique to localize the fault in a predicate of an SQL query. For each failed test case $R$ and a predicate $P_i$, \tool{} first switches the current value of $P_i$ at row $R$ (line 4) and re-evaluates the \query{WHERE} expression (line 5). If the value of the \query{WHERE} expression after switching is the same as the expected result for row $R$, \tool{} increases the $R$ score of $P_i$ by one (lines 6-7). The value of $P_i(R)$ will then be restored to its original value before \tool{} repeats the same process on other predicates and rows (line 8).


%%To show the usefulness of \tool{}'s predicate switching technique, 
Let us provide the following theorems to give the theoretical evidence
that for an SQL query with a single predicate fault, the faulty
predicate always has the highest suspiciousness score; moreover, given
sufficient test cases, its score is at least double the scores of
other predicates.

\begin{theorem}
\label{thm1}
{\em Let $P_1$, $P_2$, \ldots, $P_n$ be the predicates in the \query{WHERE}
expression of an SQL query. Suppose that there is a single fault in
one of the predicates, namely $P_*$, then $R(P_*) \ge R(P_i), \forall
i = 1 \ldots n$ and $P_i \neq P_*$.}
\end{theorem}

\begin{proof}
Let $m$ be the number of failed test cases in the predicate table. For
a failed test case, switching the value of the faulty predicate $P_*$
makes the test case passed, since the failed test case is caused by
the fault in $P_*$. Thus, the $R$ score for $P_*$ is increased by
one for every failed test case. Overall, the score of
$P_*$ is $R(P_*) = m$ (1). In contrast, changing the values of other
predicates may or may not affect the original outcome since they are
not the real cause of the error. Thus, the $R$ score of a non-faulty
predicate is either not increased or increased by one for every failed
test case. It follows that $R(P_i) \le m, \forall i = 1 \ldots n$ and
$P_i \neq P_*$ (2). From (1) and (2), we have $R(P_*) \ge
R(P_i), \forall i = 1 \ldots n$ and $P_i \neq P_*$.
\end{proof}

%\newtheorem{lemma}{Lemma}
%\newtheorem{corollary}{Corollary}

%\begin{theorem}
%\label{thm1}
%Let $P_1$, $P_2$, \ldots, $P_n$ be the predicates in the \query{WHERE} expression of an SQL query. If there is a single fault in only one of the predicates, namely $P_*$, the suspiciousness score of $P_*$ is never less than that of any other predicate:\\
%$Sus_p(P_*) \ge Sus_p(P_i), \forall i = 1 \ldots n$ and $P_i \neq P_*$.
%\end{theorem}

%\begin{proof}
%This theorem can be easily verified due to the assumption of single fault. For any failed test case, switching the value of the faulty predicate $P_*$ will make it and then the whole \query{WHERE} expression have the correct value; thus, produce the expected output. Therefore, the suspiciousness score of $P_*$ is the same as the number of the failed tests. Since the maximum score of any predicate is the number of the failed tests, the suspiciousness score of the faulty predicate is never less than that of any other predicate.
%\end{proof}

Although Theorem~\ref{thm1} demonstrates that \tool{} never ranks the
faulty predicate lower than non-faulty ones, it does not help much in
distinguishing it with the others. The next theorem shows that when
provided with a good test suite, our technique will be able to locate
the faulty predicate.

\begin{theorem}
\label{thm2}
{\em Let $P_1$, $P_2$, \ldots, $P_n$ be the predicates in the \query{WHERE}
expression of an SQL query. If there is a single fault in only one of
the predicates, namely $P_*$, and the test cases cover all possible
combinations of all predicates' values, the suspiciousness score of
$P_*$ is greater than that of any other predicate: $Sus_p(P_*) >
Sus_p(P_i), \forall i = 1 \ldots n$ and $P_i \neq P_*$.}
\end{theorem}

\begin{proof}
Let $R_p*$ be a row corresponding to a failed test case where
$P_*=p*$.  Switching the value of $P_*$ from $p*$ to $\overline{p*}$
changes the value of the \query{WHERE} clause from the actual one
$A_p*$ to the expected one $E_p*$ where $A_p* \neq E_p*$. If the test
cases cover all possible combinations of all predicates' values, there
exists a row, $R_{\overline{p*}}$, having the same values on all
non-faulty predicates as in $R_p*$ and $P_*=\overline{p*}$, which also
corresponds to a failed test case. Since the values of the predicates
on $R_{\overline{p*}}$ is the same as those on $R_p*$ after switching
$P_*$, $R_{\overline{p*}}$ will produce $E_p*$ as the actual value of
\query{WHERE} and $A_p*$ as the expected one. 

Now let us prove that switching any non-faulty predicate $P_i$ cannot
change the output on either $R_p*$ or $R_{\overline{p*}}$. Because the
\query{WHERE} expression contains only binary operators ($\vee$ and
$\wedge$) and contains each predicate $P_*$ or $P_i$ exactly once,
there always exists an operator $\odot$ having two operands $W_i$ and
$W_*$ such that they are two boolean expressions, and $W_i$ contains
$P_i$ and $W_*$ contains $P_*$.  Thus, the
\query{WHERE} condition can always be expressed in the following form
$W = W_1 \odot (W_2 \odot \dots (W_i \odot W_*)\dots)$.  Since the two
rows produce two different values for the \query{WHERE} expression
($A_p*$ on $R_p*$ and $E_p*$ on $R_{\overline{p*}}$), and $P_*$ is the
only predicate that has different values among all predicates on the
two rows, $W_*$ also has different values on them: \code{True} on one
row and \code{False} on the other. Let us prove this by
contradiction. Assume that the value of $W_*$ is the same on two rows,
because all other predicates have the same values on two rows, the
parts other than $W_*$ also have same values on two rows. Thus, the
values of the \query{WHERE} clause on two rows are the same, which is
a contradiction.

If operator $\odot$ in $W_i \odot W_*$ is a logical \code{OR}, since the
values of $W_*$ on $R_p*$ and $R_{\overline{p*}}$ are different, one
of them must be \code{True} making $W_i \odot W_*$ always be
\code{True}. If $\odot$ is a logical \code{AND}, with similar argument, either
row $R_p*$ or $R_{\overline{p*}}$, which has $W_*=$\code{False}, will
make $W_i \odot W_*$ always be \code{False}. Thus, on~at least
one row, the value of $W_i \odot W_*$ is always the same as that of
$W_*$ regardless of the value of $P_i$. Thus, switching~the value of
$P_i$ cannot change the value of $W$ on at least one~row.

In brief, there exists certain failed test row(s) where switching
the single faulty predicate can change the outputs while switching any
non-faulty one cannot. Thus, the  score of the
faulty one is greater than those of other predicates.
\end{proof}

From the proof of Theorem~\ref{thm2}, it can be seen that when the
test cases cover each possible combination of all predicates' values
exactly once, given a set of values for non-faulty predicates, there
are exactly two rows with two different values (\code{True} and
\code{False}) for the faulty predicate. For any of such pairs, if they
correspond to failed test cases, switching the faulty predicate will
change the values of both, while switching any non-faulty predicate
will change the value of at most one of them. Thus, the
suspiciousness score of the faulty one is at least twice as many as
that of any other predicate:


\begin{corollary}
\label{cor}
{\em Let $P_1$, $P_2$, \ldots, $P_n$ be the predicates in the \query{WHERE}
expression of an SQL query. If there is a single fault in one of the
predicates, namely $P_*$, and the test cases cover each possible
combination of all predicates' values exactly once, the suspiciousness
score of $P_*$ is at least twice as many as that of any other
predicate:\\ $Sus_p(P_*) \geq 2 \times Sus_p(P_i), \forall i = 1
\ldots n$ and $P_i \neq P_*$.}
\end{corollary}
 