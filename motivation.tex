\section{Motivation}

\subsection{Motivating Example}

\begin{table}
    \centering
    %    \footnotesize
    \small
    \caption{The \textsf{Users} database table}\label{tab:users-table}
\begin{tabular}{lllll}
  \addlinespace
  \toprule
  % after \\: \hline or \cline{col1-col2} \cline{col3-col4} ...
  \textbf{ID} & \textbf{Name} & \textbf{Age} & \textbf{Gender} & \textbf{Country} \\
  \midrule
  1 & Alice & 20 & Female & USA \\
  2 & Bob & 20 & Male & Canada \\
  3 & Carol & 25 & Female & Canada \\
  4 & Daniel & 30 & Male & USA \\
  \bottomrule
\end{tabular}
\end{table}

Let us start with an example to motivate our
approach. Table~\ref{tab:users-table} shows an example of a database
table with four records on four persons. The table \code{Users} has
four rows corresponding to four data records. The columns of the table
represent the {\em attributes} of each record. In this example, the
data record shows the \code{ID}, \code{Name}, \code{Age},
\code{Gender}, and \code{Country} for each person. For example, the
first row shows the information for a person with the name of
\code{Alice}. In a dynamic Web application, the data can be retrieved
and updated from PHP code with the embedded SQL queries as we will
explain next. For example, one can construct the following SQL query
in a PHP program as in \code{``SELECT Name FROM Users WHERE Age =
  20''}. The result is a {\em result set} containing the names of all
users whose age is 20.

%Web applications are often developed with dynamic data. The data is
%usually stored in databases, such as in the table shown in
%Table~\ref{tab:users-table}. The \code{Users} database table
%contains four rows (\emph{data records}). The five columns of the
%table indicate five \emph{attributes} associated with each record. For
%example, the first record has the information on a user with
%the attributes \code{ID=1, Name=Alice, Age=20, Gender=Female}, and
%\code{Country=USA}. This data can then be accessed or updated from the
%Web application through SQL queries. For instance, the SQL query
%\code{``SELECT Name FROM Users WHERE Age = 20''} returns a
%\emph{result set} containing the names of all users whose age is
%20.

As a PHP-based Web application communicates with a database, a fault
might occur at the code responsible for such
interaction. Figure~\ref{fig:example-phpcode} shows the PHP code that
contains a fault in its SQL query at line 3, resulting the incorrect
output. The goal of the function in Figure~\ref{fig:example-phpcode}
is to display the records of the persons whose ages are greater than
the specified age, and the genders and countries match with the
specified ones. However, the developer made a mistake at the PHP code
to construct the SQL query: the operator in the last predicate
\code{Country $<>$ `\$country'} should be \code{`='}, instead of
\code{`$<>$'}. The string representing this SQL query is constructed
from the string literals at line 3 and the arguments \code{age},
\code{gender}, and \code{country}, after the code responsible for the
preparation of the connection to the database at lines 1--2. The query
is sent to the database server to be executed at line 4 via the PHP
function \code{mysql\_query}. The result set is stored in the variable
\code{result}. Because the query is incorrect, the output result is
not correct as a consequence (lines 5--7 are used to loop through the
result set and display the users' names matched from the search).

%As a Web application interacts with a database through SQL queries,
%database-specific failures can occur. Fig.~\ref{fig:example-phpcode}
%shows an example of a PHP function that produces incorrect output
%values due to an error in its SQL query. The purpose of the function
%is to display the names of the users from the \code{Users} table
%(Table~\ref{tab:users-table}) that satisfy a searching criteria (by
%\code{age, gender}, or \code{country}). First, the connection to the
%database is established (lines 1-2,
%Fig.~\ref{fig:example-phpcode}). The \code{\$sql} variable (line 3)
%contains the SQL query that retrieves the users' names for a given
%search input. This query is then sent to the database server to be
%executed via the PHP function \code{mysql\_query} (line 4), and the
%returned result set is stored in the variable
%%\code{\$result}. Finally, the code on lines 5-7 is used to loop
%through the records in the result set and display the corresponding
%names of the users found via the search. Note that this function
%contains an SQL fault: on line 3, the operator in the last predicate
%\code{Country $<>$ `\$country'} of the SQL query should be \code{`='}
%instead of \code{`$<>$'}.

\begin{figure}[t]
    {\footnotesize\sffamily
    %\setlength{\tabcolsep}{6pt}
    \renewcommand{\arraystretch}{1.3}
    {\normalfont\normalsize A PHP function with an SQL query error on line 3:}\\
\begin{tabular}{@{}p{\columnwidth}@{}}
    \toprule
    function displaySearchResults(\$age, \$gender, \$country) \{ \\
    1\hspace{5pt}\$con = mysql\_connect(`localhost', `admin', `password'); \\
    2\hspace{5pt}mysql\_select\_db(`my\_database', \$con); \\
    3\hspace{5pt}\$sql = ``SELECT Name FROM Users WHERE Age $>=$ \$age \\
    \hspace{20pt}AND Gender = `\$gender' OR \textbf{Country $<>$ `\$country'} ''; \\
    4\hspace{5pt}\$result = mysql\_query(\$sql); \\
    5\hspace{5pt}while(\$row = mysql\_fetch\_array(\$result)) \{ \\
    6\hspace{15pt}echo \$row[`Name'] . `$<$br $/>$'; \\
    7\hspace{5pt}\} \} \\
    \bottomrule
\end{tabular}

\vspace{5pt}
{\normalfont\normalsize Expected SQL query on line 3:}\\
\begin{tabular}{@{}p{\columnwidth}@{}}
    \toprule
    3\hspace{5pt}\$sql = ``SELECT Name FROM Users WHERE Age $>=$ \$age \\
    \hspace{20pt}AND Gender = `\$gender' OR \textbf{Country = `\$country'} ''; \\
    \bottomrule
\end{tabular}}
    \caption{An PHP function with an SQL query error}\label{fig:example-phpcode}
\end{figure}

\begin{figure}[t]
    \centering
    \footnotesize
\begin{minipage}[t]{0.985\columnwidth}
Search Input: \textsf{\$age=25, \$gender=`Female', \$country=`USA'}
\end{minipage}

\vspace{4pt}
\begin{minipage}[t]{0.63\columnwidth}
Actual SQL query:
\begin{lstlisting}[numbers=none,xleftmargin=4pt]
SELECT Name FROM Users WHERE
    Age >= 25 AND Gender = `Female'
        OR @\textbf{Country$<>$`USA'}@
\end{lstlisting}
\end{minipage}
\hspace{0.03\columnwidth}
\begin{minipage}[t]{0.30\columnwidth}
Actual output:
\begin{lstlisting}[numbers=none,xleftmargin=4pt]
Bob
Carol
\end{lstlisting}
\end{minipage}

\begin{minipage}[t]{0.63\columnwidth}
Expected SQL query:
\begin{lstlisting}[numbers=none,xleftmargin=4pt]
SELECT Name FROM Users WHERE
    Age >= 25 AND Gender = `Female'
        OR @\textbf{Country=`USA'}@
\end{lstlisting}
\end{minipage}
\hspace{0.03\columnwidth}
\begin{minipage}[t]{0.30\columnwidth}
Expected output:
\begin{lstlisting}[numbers=none,xleftmargin=4pt]
Alice
Carol
Daniel
\end{lstlisting}
\end{minipage}
%
%\vspace{-7pt}
    \caption{Output of the PHP function in Fig.~\ref{fig:example-phpcode}}\label{fig:example-output}
\end{figure}

%for a given test case



Due to that error in the SQL query, the function does~not display the
expected results. Fig.~\ref{fig:example-output} shows the actual and
expected output values given the search input \code{\$age=25,
\$gender=`Female', \$country=`USA'}. In the correct SQL query, the
three-predicate condition determining the values in the returned
result set is: \code{Age $>=$ 25} \code{AND} \code{Gender = `Female'}
\code{OR} \code{Country = `USA'}. Since the actual query has a fault
in the last predicate, there is a mismatch between its actual and
expected outputs. As seen, Alice's and Daniel's names are expected to
be found in the result, but are not displayed. Meanwhile, Bob's name
is included in the actual output although it must not.

%Alice and Daniel should be included in the returned result whereas Bob should not.

\begin{figure}[t]
    \centering
    \scriptsize
    \setlength{\tabcolsep}{2pt}
    \renewcommand{\arraystretch}{1.1}
{\sffamily
\begin{tabular}{@{}llllll@{}}
    \toprule
                                                                                    & \multicolumn{4}{c}{Test Cases} & Sus.\\
                                                                                    \cmidrule{2-5}
    function displaySearchResults(\$age, \$gender, \$country) \{                    & 1         & 2         & \ldots         & $n$ \\
    \ldots                                                                          & $\bullet$ & $\bullet$ & \ldots    & $\bullet$ & 0.5\\
    4\hspace{5pt}\$result = mysql\_query(\$sql);                                   & $\bullet$ & $\bullet$ & \ldots    & $\bullet$ & 0.5\\
    4$\ast$\hspace{10pt}SELECT...Age$>$=? AND Gender=? OR Country$<>$?                  & $\bullet$ & $\bullet$ & \ldots    & $\bullet$ & \textbf{0.5}\\
    5\hspace{5pt}while(\$row = mysql\_fetch\_array(\$result)) \{                    & $\bullet$ & $\bullet$ & \ldots    & $\bullet$ & 0.5\\
    6\hspace{20pt}echo \$row[`Name'] . `$<$br /$>$';                                    & $\bullet$ & $\bullet$ & \ldots    & $\bullet$ & 0.5\\
    \midrule
    Pass/Fail Status                                                                & F       & P     & \ldots    & F \\
    \bottomrule
\end{tabular}
}
    \caption{Suspiciousness scores of the PHP statements and SQL queries in Fig.~\ref{fig:example-phpcode} computed using the technique by Clark et al. and Tarantula metric}\label{fig:suspiciousness}
\end{figure}

%Given the mismatch between the actual output and the expected one, 
Given such mismatch, it is not obvious which part of the program
accounts for the error. In this example, the fault lies at a predicate
of an SQL query, whereas the Web application is written in
PHP. Moreover, the actual execution of the SQL query occurs at the
database server, which is separate from the main PHP program where the
SQL query is created. Thus, the process of localizing this type of
fault needs to be \emph{database-aware}, namely taking into account
the interaction between the Web application and the database via
queries.

Although there exist many fault localization techniques for
single-language programs (e.g. \cite{abreu-ochiai-07},
\cite{tarantula05}),
%\cite{liblit-pldi05}) 
and data-centric programs (e.g. \cite{dor-issta08}, \cite{litvak10},
\cite{saha11}), little attention has been given to database-specific
faults in a multilingual Web application.~A state-of-the-art approach
in database-aware fault localization for Web applications is
Clark {\em et al.}~\cite{ga-ase11}. Their tool monitors SQL queries
generated at runtime and compute the \emph{suspiciousness} scores for
those SQL queries and their attributes, as well as the
statements in the main program. Similar to other statistical fault
localization methods (e.g. Tarantula~\cite{tarantula05} and
Ochiai~\cite{abreu-ochiai-07}), the idea in computing
suspiciousness scores is to contrast the runtime behaviors of correct
and incorrect executions of the program. Specifically, if a program
entity (program statement or predicate) is exercised by more failing
test cases than passing ones, it is more likely to be responsible for
the failure, and thus assigned with a higher suspiciousness~score.

Compared with the previous fault localization approaches, Clark {\em
et al.}'s method~\cite{ga-ase11} is database-aware in that it
considers SQL queries or SQL attributes as program entities and also
computes their 
%corresponding 
suspiciousness scores. For example, Fig.~\ref{fig:suspiciousness}
illustrates the computation of suspiciousness scores for the program
entities in Fig.~\ref{fig:example-phpcode} including SQL
queries. Line 4$\ast$ shows the SQL query executed by the PHP
statement \code{mysql\_query} on line 4 at runtime, with the question
marks indicating literal values (numbers/strings). If the PHP
\code{mysql\_query} statement executes multiple \emph{unique} SQL
queries (each with a different set of attributes) in different
executions, the suspiciousness scores of the individual queries will
be computed.

That computation is based on the idea that if some of unique
SQL queries are executed by more failing test cases than passing ones,
they will have higher scores. This strategy is useful when there are
multiple unique SQL queries that expose different behaviors in passing
and failing test cases. However, in practice, a PHP
\code{mysql\_query} statement often executes only one SQL
query, in which the set of attributes is fixed, and the concrete SQL
queries in different executions have the same structure and vary only
at the literal values. This phenomenon is also reported by the
authors~\cite{ga-ase11}. As an illustration, the unique query
in this example (line 4$\ast$ of Fig.~\ref{fig:suspiciousness}) is
\code{``SELECT Name FROM Users WHERE Age $>$= ? AND Gender = ? OR
Country $<>$ ?''}, with the set of attributes \code{\{Name, Age,
Gender, Country\}}.

Since there is one unique SQL query executed by the PHP
\code{mysql\_query} statement, the coverage of the SQL query in the
passing and failing test cases is the same as the \code{mysql\_query}
statement, and therefore its suspiciousness score does not provide
further information about the location of the fault. 
%Let us describe the score computation to illustrate this limitation. 
In Fig.~\ref{fig:suspiciousness}, the bullets indicate the program
entities that are exercised by a given test case. At the bottom row,
the letters \code{P} and \code{F} specify a passing and failing test
case, respectively. Column \code{Sus.} shows the suspiciousness score
$S(e)$ for a program entity $e$ using the Tarantula
(\cite{tarantula05}) metric:
\begin{equation}
\small
\label{eq:tarantula}
    S(e) = \dfrac{\tfrac{Failed(e)}{TotalFailed}}{\tfrac{Passed(e)}{TotalPassed} + \tfrac{Failed(e)}{TotalFailed}}
\end{equation}
where $Passed(e)$ is the number of passing test cases that execute
$e$, $Failed(e)$ is the number of failing test cases that execute $e$,
and $TotalPassed$ and $TotalFailed$ are the respective total numbers
of passing and failing test cases.

Although there is a fault in the SQL query, its suspiciousness score
(0.5) is the same as the PHP \code{mysql\_query} statement that
executes it. In fact, all the program entities always have the same
score even when a different suspiciousness metric is used, since the
same set of program entities is executed in every passing or failing
test case (regardless of the test suite). From those suspiciousness
scores,
% computation results, 
no further information about the fault is gained. This limitation motivated us
to develop a new
%database-aware fault localization technique 
%method that provides better diagnosis for database-specific failures.
method to better localize database-specific faults.

\subsection{Approach Overview}

The example illustrates an \emph{incorrect output failure} that may
occur as a Web application interacts with a database. The incorrect
%values in the
returned results are often caused by the way the data records are
selected from a database table, which is specified by the
\query{WHERE} clause of an SQL \query{SELECT} query. In this paper, we
target incorrect output failures due to errors in the \query{WHERE}
part of SQL queries. Another type of database-specific failure is
\emph{execution failure}, which occurs if the SQL query has incorrect
syntax or specifies an invalid operation with the database (e.g. the
query refers to a non-existent database table or attribute). While
execution failures can be fixed by checking the syntax of queries and
ensuring their conformance to the database schemas, incorrect output
failures are more difficult to resolve since they are caused by
semantic errors in the queries.

To learn more on SQL queries, we conducted an exploratory study. We
collected three open-source dynamic Web applications from SourceForge
(\code{AddressBook}, \code{SchoolMate}, and \code{ZenCart}) with a
total of 1,284 PHP files and 225 KLOCs. We wrote a tool to analyze
those files and found 2,518 SQL queries to the databases with 2,672
predicates in \code{WHERE} clauses. There are almost 2 SQL queries in
a PHP file. A query could have up to 10 predicates. There are 304
queries with 3-10 predicates. There are 5-25 lines of PHP code for
each SQL query. Thus, if a tool such as in Clark {\em et
al.}~\cite{ga-ase11} reports to developers that a query is faulty
without details on specific predicates, it would be not efficient for
them to locate the fault. Those numbers motivated us to develop a tool
to help localize the defects due to errors in the predicates of
\code{WHERE} clauses.

In designing our solution, we leverage the fact that the output of a
query reveals information about the correctness of individual records
in the output. If the actual output of the program does not match with
the expected one, we can further analyze individual records in the
actual output to determine which records result in the mismatch. For
example, the actual and expected outputs for the given test case in
Fig.~\ref{fig:example-output} reveal that \code{Carol} is a correct
record whereas \code{Bob} is an incorrect one. To locate the fault,
a developer would then examine the three predicates in the SQL query
that returns \code{Bob}. (S)he would recognize that two
predicates \code{Age $>$= 25} and \code{Gender = `Female'} evaluate to
\code{false} since Bob's age is \code{20} and his gender is
\code{`Male'}. Thus, the first two predicates do not contribute to the
output of \code{Bob}. Only the last predicate whose value is true
determines its presence. Thus, the last predicate is most suspicious.
%The developer could conclude that it most likely causes the fault.

% incorrect output.
%In fact, it is the source of the fault in this example.

%Note that we assume the table \code{Users} and the table column \code{Name} are correct; otherwise, it would have led to an execution failure, which is not the focus of this paper.

Based on the above ideas, we develop {\tool}, a database-aware fault
localization method that works at two levels:

(1) \tool{} localizes the {\em faulty SQL query} by monitoring the
output of individual records and computing the query's suspiciousness
score based on the correctness of these records;

(2) Given an SQL query that is likely to be faulty, \tool{} examines
the predicates in the {\em \query{WHERE} part} of the SQL query to
identify which predicate is responsible for the incorrect output.
%Next section will explain the step (1).

%, we present these two
% methods.
