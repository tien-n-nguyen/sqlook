\subsection{Localizing Predicates with Multiple Faults}

\begin{figure}[t]
    \centering
\begin{lstlisting} [
    emph={function, foreach, do, if, else, end},
	mathescape=true
]
function LocalizeFaults(LogicExpression $E$)
    Rewrite $E$ as $E_1 \oplus E_2 \ldots E_n$ // $\oplus$ is either $\wedge$ or $\vee$.
    foreach $E_i$ do
        Rewrite $E$ as $E_i \oplus \overline{E_i}$, with $\overline{E_i} = E \setminus E_i$
        Compute $R(E_i)$ and $R(\overline{E_i})$ by predicate switching
    end
    $E_m = \{E_i: R(E_i) / (R(\overline{E_i}) + 1) \ge \sigma\}$
    if $|E_m|$ > 1
        Report multiple faults $E_m$, then exit
    if $E_m$ contains only one expression $E_s$
        if $E_s$ is atomic
            Report single fault $E_s$, then exit
        else
            LocalizeFaults($E_s$)
\end{lstlisting}
     \caption{Multiple-Fault Localization Algorithm}
     \label{fig:algorithm-multiplefaults}
\end{figure}

%We extend \tool{}'s predicate switching algorithm to work on a more
%general case 

We extended the algorithm in Section~\ref{single-fault-section} for general cases
where a query may have one or more faults in its
predicates. \tool{} first rewrites the \query{WHERE} condition
in different forms so that the predicates with faults can be grouped
together as one combined predicate. In such cases, the multiple faults
can be seen as one fault in the combined predicate after grouping. It
then applies predicate switching for a single fault to identify the
group of predicates that is most likely to contain the faults.

Figure~\ref{fig:algorithm-multiplefaults} shows our algorithm.
%%% to localize multiple faults in predicates.
%%%in the \query{WHERE} expression of an SQL query. 
Given a logic expression $E$,
%%% that might contain faults,
\tool{} rewrites $E$ into one of the following forms (line 2):\\
\[
\small
E =
\begin{cases}
    E_1 \wedge E_2 \wedge \ldots E_n, \text{with } n > 1 & \text{(Conjunctive form)} \\
    E_1 \vee E_2 \vee \ldots E_n, \text{with } n > 1 & \text{(Disjunctive form)} \\
\end{cases}
\]
where each $E_i$ is another logic expression. If $E$ is an
\emph{atomic} predicate (i.e., $E$ cannot be written in either
conjunctive or disjunctive form), {\tool} can stop further
localization. For each predicate $E_i$ in $E$, it splits $E$ into two
parts: the first part contains only $E_i$, and the second part
($\overline{E_i}$) contains all the other predicates (line 4). Then,
it computes the score $R$ for $E_i$ and
$\overline{E_i}$ using predicate switching (line 5). If
the faults are all located in the set $\overline{E_i}$, the ratio
$R(E_i)$ over $R(\overline{E_i})$ will be low, indicating that $E_i$
is not likely the fault. Otherwise, a high ratio suggests that $E_i$
might be one of the faults in~$E$.



After computing the suspiciousness scores for all the predicates in
$E$, \tool{} selects the set $E_m$ of the predicates whose ratios
$R(E_i) / (R(\overline{E_i}) + 1)$ exceed a threshold $\sigma$
(line~7). If there are more than one predicate with a high ratio, it
reports the $R$ scores of all the predicates in $E$ and highlights the
predicates in $E_m$ as potentially containing multiple faults (lines
8-9).~If there is only one predicate $E_s$ with a high score, {\tool}
checks whether it is an atomic predicate (line 11). For an
atomic predicate $E_s$, it reports the scores of all the predicates in
$E$ and identifies $E_s$ as the likely single source of the error
(line 12). Otherwise, if $E_s$ can be further decomposed into a
conjunctive or disjunctive form, it recursively calls
\code{LocalizeFaults} to repeat the fault localization for
$E_s$ (line 14).

%invokes the
%recursive function 
