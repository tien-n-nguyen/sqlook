\section{Localizing Faulty SQL Queries}

%\begin{figure}[tbp]
%  \centering
%  \includegraphics[width=\columnwidth]{images/SQL-Transformed2.eps}
%  \caption{Instrumented PHP interpreter to monitor the execution of SQL queries}\label{fig:SQL-Transformed}
%\end{figure}

\begin{figure}[tbp]
  \centering
  \includegraphics[width=\columnwidth]{images/SQL-Transformed4.eps} %2
 \caption{Instrumented PHP interpreter to monitor the execution of SQL queries}\label{fig:SQL-Transformed}
\end{figure}

In a PHP Web application, program faults may be found in regular PHP
statements or those that interact with the database, called
\emph{database-interaction points} (e.g. line 4 of
Fig.~\ref{fig:example-phpcode}). The goal of this step is to
localize these faults, specifically to decide if a
database-interaction point contains a fault(s) at its \query{WHERE}
clause or not. To do that, {\tool} uses
%the statistical suspiciousness metric
Tarantula~\cite{tarantula05} to compute the suspiciousness scores for
all entities in PHP and in SQL \code{WHERE} clauses. To avoid the
issues as in Clark {\em et al.}~\cite{ga-ase11}'s approach, we have
following key design strategies:

(1) {\em Row-based test cases}. 
%As explained in Section 2.2., 
Instead of viewing the input and entire expected output from the
database as one test case, we analyze individual data records in
the actual and expected outputs to create row-based test cases. For
example, in Fig.~\ref{fig:example-output}, we have 4 row-based test
cases: 1) the input \code{\$age=25, \$gender=`Female',
\$country=`USA'} and the output of \code{Alice}'s record, 2) that
input and the absence of \code{Bob}'s, 3) that input and the output of
\code{Carol}'s, and 4) that input and the output of \code{Daniel}'s.

(2) {\em Monitoring the execution of PHP and SQL entities via
instrumentation.} The suspiciousness scores are given to PHP
statements and SQL \code{WHERE} clause(s). Instead of passing the
control to the database engine to execute an SQL command as in 
%Clark {\em et al.}
\cite{ga-ase11}, {\tool} instruments into an PHP interpreter the code
to execute the SQL command (Fig.~\ref{fig:SQL-Transformed}) and to
observe the evaluation of the \code{WHERE} clause with respect to
every row-based test case.  Because the \code{WHERE} clause's value
decides if the output of a row-based test case is present or not,
{\tool} needs to record if that clause is evaluated to \code{True} or
\code{False}. From that, it knows which part (\code{True} or
\code{False}) of~a \code{WHERE} clause is exercised for a row-based
test case. Thus, the suspiciousness metric can now be applied not only
to PHP code, but also to the deeper level of the two parts of a
\code{WHERE} clause.

(3) {\em Row-based Slicing across PHP and SQL.} Since {\tool} uses
row-based test cases, it needs to record the PHP statements that are
exercised in the execution of such a test case, i.e. the PHP
statements that are involved in the output of a data record (i.e. a
row). To do that, during the monitoring in step (2), it computes the
forward slice corresponding to each data row across the PHP statements
and the two parts of \code{WHERE}.
%To do that, {\tool} uses the statistical suspiciousness metric
%Tarantula~\cite{tarantula05} to compute the suspiciousness scores for
%all entities in PHP and in SQL \code{WHERE} clause(s). To avoid the
%issue(s) as in Clark {\em et al.}~\cite{ga-ase11}'s approach, we have
%following key design strategies:

%In a PHP web application, program faults may be found in regular PHP
%statements or those that interact with the database system (i.e.,
%\emph{database-interaction points}). To detect these faults, \tool{}
%uses a suspiciousness metric $Sus$, which can be any of the existing
%statistical suspiciousness metrics (such as
%Tarantula~\cite{tarantula05} and Ochiai~\cite{abreu-ochiai-07}), and
%computes the suspiciousness scores for all statements in a PHP program
%including database interaction points. However, as illustrated in the
%motivating example, the faults that exist in database-interaction
%points may not be identified by the metric $Sus$. To further handle
%this type of faults, \tool{} monitors the execution of SQL queries at
%database-interaction points where the conditions given by the
%\query{WHERE} parts of the SQL queries are evaluated to select data
%records from one or more database tables. If the evaluation results of
%the \query{WHERE} conditions do not meet the expected outcomes for
%certain records, \tool{} will increase the suspiciousness scores of
%those \query{WHERE} conditions to indicate that they are likely to
%contain faults. Unlike the technique proposed by Clark et
%al.~\cite{ga-ase11}, by monitoring the execution of SQL \query{WHERE}
%conditions, \tool{} can increase the suspiciousness scores of faulty
%conditions in SQL queries even when there exists only one unique SQL
%query per database-interaction point.

\subsection{Monitoring the Execution of SQL Queries}
\label{monitoring-section}



Figure~\ref{fig:SQL-Transformed} illustrates the execution of a PHP
program with SQL queries by a regular PHP interpreter. It contains the
source code to evaluate various expressions in a PHP program, e.g.
assignments, variables, function calls, etc. Among them, we focus on
the evaluation of the PHP
\code{mysql\_query} statement where database interactions take place
(the shaded part in the PHP interpreter's source code). The
interpreter evaluates the \code{mysql\_query} statement by sending the
SQL query to the database management system (DBMS) and retrieving its
returned result. Since the actual query execution is performed by the
DBMS, the interpreter does not have access to the internal operations
that evaluate the query's \query{WHERE} condition and extract data
records from database. Therefore, to monitor the evaluation of the
expressions in the
\query{WHERE} clause at run time, we instrument the original PHP
interpreter and replace the source code handling database queries with
our instrumented code that performs the query's operations
(Figure~\ref{fig:SQL-Transformed}).



%By evaluating the \query{WHERE} condition instead of delegating the evaluation to the DBMS, the instrumented interpreter now has the evaluation results of the \query{WHERE} expression for individual data records.

The operations that fulfill an SQL \query{SELECT} query consist of the
following: (a) retrieving data from one or more database tables
specified by the \query{FROM} part of the SQL query, (b) extracting
the data records that satisfy the criteria specified by the
\query{WHERE} condition of the SQL query, and (c) projecting the set
of columns (attributes) given in the \query{SELECT} part of query into
the final result set. As an example, the SQL query \code{SELECT}
\code{Name FROM Users WHERE Age $>$= 25 AND Gender = `Female' OR
Country $<>$ `USA'} retrieves the names of all the users from the
\code{`Users'} database table that meet the condition \code{Age $>$=
  25 AND Gender = `Female' OR Country $<>$ `USA'}.


Let us explain the instrumentation for our monitoring
process. Our instrumented interpreter re-implements these three
operations in four steps (shown in the instrumented code in
Figure~\ref{fig:SQL-Transformed}). The detailed instrumentation
execution is illustrated in Figure~\ref{fig:Instrumented-Code}. (In
our implementation, we instrument Quercus
(http://quercus.caucho.\-com/), a PHP interpreter, and use JSqlParser
(http://jsqlparser.source\-forge.\-net/) to parse SQL code. Specifically,
we have the following steps:

%; both are written in Java.)

%~\footnote{In our implementation, \tool{} instruments
%Quercus (http://quercus.caucho.com/), a PHP interpreter, and uses
%JSqlParser (http://jsqlparser.sourceforge.net/) to parse SQL queries;
%both are written in Java.}

\begin{figure*}[tbp]
  \centering
  \includegraphics[width=4.55in]{images/Instrumented-Code-2.eps}\\
  \caption{Instrumented code to monitor the execution of SQL queries~\cite{icsm13}}\label{fig:Instrumented-Code}
\end{figure*}

%\begin{figure*}[tbp]
%  \centering
%  \includegraphics[width=0.8\textwidth]{images/Instrumented-Code-1.eps}\\
%  \caption{Instrumented code to monitor the execution of SQL queries}\label{fig:Instrumented-Code}
%\end{figure*}

\vspace{0.04in}
\textbf{Step 1. Modifying an SQL query:}
The original SQL query is parsed and modified. Its
\query{WHERE} part is removed, and~the column set is changed into \code{`*'}
(i.e., all columns will be retrieved). The \code{ModifiedSql} variable
now contains the modified SQL query, which retrieves all data from one
or more database tables (the SQL query may specify a \query{JOIN}
operation on multiple tables). The \query{WHERE} clause and the column
set in the original SQL query are also extracted out (to the variables
\code{WhereExp} and \code{SelectedCols}, respectively) so that the
rows and columns can be filtered from the modified query's result in
the next steps. In Figure~\ref{fig:Instrumented-Code}, the dashed box
at the top shows an example of an SQL query that is the input of step
1. The values of the three variables \code{ModifiedSql},
\code{WhereExp}, and \code{SelectedCols} are shown in the dashed boxes
coming out of step 1, respectively.

% and will be used to illustrate the following steps.



\vspace{0.04in}
\textbf{Step 2. Executing the modified SQL query:}
The modified SQL query obtained from step 1 will be sent to the DBMS
to be executed there. Compared to the result of the original query,
the result of the modified query contains all data from the database
table(s) in which the rows and columns have not been filtered out
according to the specifications in the original query. In
Figure~\ref{fig:Instrumented-Code}, the dashed box coming out of step
2 (table \code{T}) shows the result after executing the
\code{ModifiedSql} query.

\vspace{0.04in}
\textbf{Step 3. Filtering rows:} In this step, \tool{} loops through
each row \code{R} in table \code{T} and evaluates the expression
\code{WhereExp} in the \query{WHERE} clause of the original SQL query~for row \code{R}.
If a row \code{R} satisfies the condition specified
by \code{WhereExp}, it will be extracted out, in the same way that a
DBMS processes the original SQL query. In
Figure~\ref{fig:Instrumented-Code}, the extracted rows
(\code{FilteredRows}) are the rows 2 and 3 of the \code{Users}
table. Importantly, by evaluating the individual predicates in the
\query{WHERE} expression, \tool{} is able to determine whether the
\query{WHERE} condition evaluates to \code{True} or \code{False} for a
given row.

\vspace{0.04in}
\textbf{Step 4. Filtering columns:} In the final step, \tool{}
extracts the column set specified by the original SQL query from the
data obtained from step 3. In Figure~\ref{fig:Instrumented-Code}, the
final result consists of the names \code{Bob} and \code{Carol},
which preserves the original query's result as if it was executed by
the DBMS.

%\vspace{0.06in}
%\noindent {\bf Evaluation Rules.}

\subsection{Evaluation Rules for Step 3}

Since SQL is a declarative language, when evaluating the SQL
expressions in step 3, \tool{} needs~to understand and implement their
semantics in the interpreter. Table~\ref{tab:transformation-rules}
shows the rules to evaluate an SQL expression $E$ with the data from a
given table row~$R$.

\begin{table}[t]
    \centering 
    %\scriptsize %\setlength{\tabcolsep}{3pt}
    \small
    \caption{Evaluation Rules for SQL expressions}\label{tab:transformation-rules}
\begin{tabular}{@{}cll@{}}
    \toprule
    \textbf{No.} & \textbf{SQL Expression E} & \textbf{Eval. Result on E for Row R} \\
    \midrule
    1.  & E ::= E1 AND E2     &   E(R) $\leftarrow$ E1(R) \&\& E2(R) \\
    \midrule
    2.  & E ::= E1 OR E2      &   E(R) $\leftarrow$ E1(R) $||$ E2(R) \\
    \midrule
    3.  & E ::= E1 BETWEEN    &   E(R) $\leftarrow$ E2(R) $<=$ E1(R)  \\
        & \hspace{30pt} E2 AND E3       & \hspace{30pt} \&\& E1(R) $<=$ E3(R) \\
    \midrule
    4.  & E ::= E' LIKE               &   $Pattern$.replace(\{`\%', `.*'\}, \{`\_', `.'\}) \\
        & \hspace{30pt} $Pattern$     &   E(R) $\leftarrow$ match($Pattern$, E'(R)) \\
    \midrule
    5.  & E ::= $Table$ AS $Alias$   & $Alias$.RealName $\leftarrow$ $Table$ \\
    \midrule
    6.  & E ::= $Table$.$Column$      &   if $Table$ is alias then \\
        &                               &   \hspace{15pt}$Table$ $\leftarrow$ $Table$.RealName \\
        &                              & E(R) $\leftarrow$ R.getCol($Table$.$Column$) \\
    \midrule
    7.  & E ::= E' IN (\{$E_i$\}),    &   E(R) $\leftarrow$ \{$E_i(R)$\}.contains(E'(R)) \\
        & \hspace{30pt} $i = 1 \ldots n$ \\
    \midrule
    8.  & E ::= EXISTS E'             &   E(R) $\leftarrow$ E'(R) != null \\
    \midrule
    9. & E ::= $SubSelect$          &   Send $SubSelect$ to DBMS \\
       &                            &   Store returned result in E(R) \\
    \bottomrule
\end{tabular}
\end{table}


\textbf{Rules 1 and 2:}
\code{E ::= E1 AND/OR E2}. For an SQL logic expression (e.g. \code{AND}
and \code{OR}), \tool{} evaluates the two sub-expressions
\code{E1} and \code{E2}, and returns the result based on the semantics
of the operator in the expression \code{E}. Other arithmetic
expressions such as addition, subtraction, and comparison expressions
(e.g., \code{`='} and \code{`$>$='}) are realized in a similar
manner.

\textbf{Rule 3:} \code{E ::= E1 BETWEEN E2 AND E3}. \tool{}
evaluates three expressions \code{E1}, \code{E2}, \code{E3}, and
checks the condition that the value of \code{E1(R)} is in the
range of \code{E2(R)} and \code{E3(R)}.

\textbf{Rule 4:} \code{E ::= E' LIKE $Pattern$}. \tool{}
implements the SQL regular expression $Pattern$ by determining if the
evaluation result from \code{E'(R)} matches that pattern.

\textbf{Rule 5:} \code{E ::= $Table$ AS $Alias$}. In an SQL query,
database tables can be referred to via aliases (e.g., the SQL query
\code{SELECT * FROM Users AS u WHERE u.Age $>=$ 25} assigns the alias
name \code{`u'} for the \code{Users} table). Thus, \tool{}
records the real table name of the alias for later reference.

\textbf{Rule 6:} \code{E ::= $Table$.$Column$}. For an SQL expression
to access a table column (e.g., \code{`u.Age'}), \tool{} first checks
if the table name is an alias name, in which case the real table name
is used. Then, \tool{} retrieves the corresponding table column from
the current row \code{R}. If the expression does not specify a table
name, the only table name in the query will be used.

%For example, given SQL query in the motivating example and the row
%\code{(1, Alice, 20, Female, USA)}, the evaluation result for column
%\code{Name} is \code{Alice}.

\textbf{Rule 7:} \code{E ::= E' IN (\{$E_i$\}), $i = 1 \ldots n$}.
\tool{} evaluates \code{E'} and the set of expressions \{$E_i$\}
and determines if the evaluation result of \code{E'} is contained
in the evaluation results of \{$E_i$\}. Note that \{$E_i$\} can be
another SQL \query{SELECT} query.

\textbf{Rule 8:} \code{E ::= EXISTS E'}. \tool{} evaluates \code{E'}
and checks if the result is non-empty (not null). Similar to
rule~7, \code{E'} is often an SQL sub-query.

% which is handled by the next evaluation rule.

\textbf{Rule 9:} \code{E ::= $SubSelect$}. For an expression specifying
a sub-query, \tool{} sends the sub-query to the DBMS to be
executed and retrieves its returned result. It monitors only the
extraction of table rows from the top-level query and transfers the
executions of all sub-queries (if any) to the DBMS.

%Although the SQL query in the motivating example is simple for
%illustrative purposes,

\tool{} is also able to handle more complex SQL
constructs such as \query{JOIN}, \query{ORDER BY}, \query{TOP},
%\query{ALL}, \query{ANY} and SQL functions such as \query{MAX},
\query{LEN}, and \query{UCASE}. It performs these operations by
either re-implementing them or delegating them to the DBMS (step 2 of
Figure~\ref{fig:Instrumented-Code} or rule 9 in Table
\ref{tab:transformation-rules}).

% Currently, \tool{} does not handle \query{UNION}. 


\subsection{Row-based Slicing across PHP and SQL}

\begin{figure}[tbp]
  \centering
  \includegraphics[width=0.95\columnwidth]{images/Execution-Trace-1.eps}\\
  \caption{Execution trace of row-based test cases}\label{fig:Execution-Trace}
\end{figure}



During the above monitoring process, {\tool} also records the PHP
statements and the parts (\code{True} and \code{False}) of
\code{WHERE} clause(s) that are exercised in the execution of a
row-based test case, thus, involved in the output of a data row. It
computes program slices corresponding to each data row.
%To gather more information about the fault, \tool{} considers each
%table row as a test case and computes the program slice corresponding
%to each test case. 
Figures~\ref{fig:Execution-Trace} and ~\ref{fig:Rowbased-Slicing}
illustrate the computation of the slices for the example in
Fig.~\ref{fig:example-phpcode} (with lines 5' and 7 added for
illustrative purposes). Fig.~\ref{fig:Execution-Trace} shows the
trace in the execution of our instrumented interpreter. The
nodes denote the PHP and SQL entities, and the arrows show the order
in the execution trace. Besides PHP statements, {\tool} needs to
consider two parts of a \code{WHERE} clause (see lines 4a and 4b).
%\tool{} introduces a new type of program
%entity, namely the state of the \query{WHERE} condition in an SQL
%query (lines 4a and 4b).
We use \code{1-Alice}, \code{2-Bob}, \code{3-Carol}, and
\code{4-Daniel} to denote the data records in the result set \code{T}
after step 2 in Fig.~\ref{fig:Instrumented-Code}, i.e., after
executing the modified SQL query.

As a PHP entity or SQL part is executed, \tool{} recognizes the
corresponding processed data row and includes that entity/part into
the slice of that row-based test case with its order in the execution
trace. For example, since the \query{WHERE} expression evaluates to
\code{True} for \code{Bob} and \code{Carol}, line 4a is included in
the slices of the test cases corresponding to those data rows
(see Fig.~\ref{fig:Rowbased-Slicing}). Similarly, line 4b is included in
the slices for \code{1-Alice} and \code{4-Daniel}, corresponding to
\code{WhereExp = False}. Since only \code{2-Bob} and \code{3-Carol}
are returned in the query's result, the PHP code on lines 5-6 performs
two iterations to print out each name. Whenever a PHP row-retrieving
function such as \code{mysql\_fetch\_array} is executed and returns a
\emph{non-}\code{null} table row (line 5), \tool{} includes 
in the slice of the corresponding row-based test case the PHP
statements having data dependencies with that row. Thus, lines 5, 5',
and 6 are included in \code{2-Bob}'s and \code{3-Carol}'s slices.
%the slices for \code{2-Bob} and \code{3-Carol}.
%executed twice (for \code{2-Bob} and \code{3-Carol})
%the statements that follow in the slice of the corresponding row-based
%test case. Therefore, statements 5-6 are included in the slices of
%\code{Bob} and \code{Carol}.
As line~5~is executed for \code{1-Alice} and
\code{4-Daniel}, the variable \code{\$row} is \code{null} and the
execution exits the \code{while} loop; i.e., no table row is accessed.
Thus, lines 5' and 6 are not included for \code{1-Alice}'s and
\code{4-Daniel}'s slices, while line 5 is included for all slices.~Since 
line 7 (or any next line) is exercised regardless of
data rows, it will be included in all slices of all test cases.
Similarly, the PHP statements before line 4 are exercised and included
in all slices.

%When a statement is not associated with any particular test case
%(e.g., statements 4, 7, and statement 5 in its last execution),
%\tool{} includes them in the slices of all test cases. The slices for
%these row-based test cases are shown in
%Figure~\ref{fig:Rowbased-Slicing}.

%tbp
\begin{figure}[t]
  \centering
  \includegraphics[width=0.81\columnwidth]{images/Row-based-Slicing-1.eps}\\ %0.84
  \caption{Row-based slicing across PHP and SQL}\label{fig:Rowbased-Slicing}
\end{figure}


\subsection{Computing Suspiciousness Scores}

\begin{figure}[t]
    \centering
    \setlength{\tabcolsep}{1pt}
    \renewcommand{\arraystretch}{1.1}
{\sffamily
 \footnotesize
\begin{tabular}{ll@{}ccccl@{}}
    \toprule
                   & & \multicolumn{4}{c}{Row-based test case} & Sus.\\
    \cmidrule{1-1}                                                                  \cmidrule{3-6}
    function display...(\$age, \$gender, \$country) \{                    & & 1-Alice             & 2-Bob             & 3-Carol   & 4-Daniel \\
    \ldots                                                                          & & $\bullet$     & $\bullet$     & $\bullet$     & $\bullet$   & 0.5\\
    4\hspace{5pt}\$result = mysql\_query(\$sql);                                   & & $\bullet$     & $\bullet$     & $\bullet$     & $\bullet$   & 0.5\\
    4a\hspace{30pt}WhereExp* = True             & &               & $\bullet$     & $\bullet$     &    & 0.25\\
    4b\hspace{30pt}WhereExp* = False                                           & & $\bullet$     &               &   & $\bullet$   & \textbf{1.0}\\
    5\hspace{5pt}while(\$row = mysql\_fetch\_array(\$result))\{                   & & $\bullet$  & $\bullet$     & $\bullet$     &  $\bullet$ & 0.5\\
    6\hspace{20pt}echo \$row[`Name'] . `$<$br /$>$';                                    & &         & $\bullet$     & $\bullet$     &  & 0.25\\
    \midrule
    Pass/Fail Status                                                                & & F             & F             & P             & F \\
    \bottomrule
    \addlinespace
    \multicolumn{6}{l}{Test case with \$age = 25, \$gender = `Female', \$country = `USA'}\\
    \multicolumn{6}{l}{* WhereExp: Age $>$= 25 AND Gender = `Female' OR Country $<>$ `USA'}
\end{tabular}
}
    \caption{Suspiciousness scores computed by \tool{} for Fig.~\ref{fig:example-phpcode}}\label{fig:suspiciousness-improved}
\end{figure}

%\begin{figure*}[t]
%    \centering
%    \footnotesize
%    %\setlength{\tabcolsep}{6pt}
%    \renewcommand{\arraystretch}{1.1}
%{\sffamily
% \scriptsize
%\begin{tabular}{llccccl}
%    \toprule
%    Test case with \$age = 25, \$gender = `Female', \$country = `USA'               & & \multicolumn{4}{c}{Row-based test case} & Sus.\\
%    \cmidrule{1-1}                                                                  \cmidrule{3-6}
%    function displaySearchResults(\$age, \$gender, \$country) \{                    & & 1-Alice             & 2-Bob             & 3-Carol   & 4-Daniel \\
%    \ldots                                                                          & & $\bullet$     & $\bullet$     & $\bullet$     & $\bullet$   & 0.5\\
%    4\hspace{10pt}\$result = mysql\_query(\$sql);                                   & & $\bullet$     & $\bullet$     & $\bullet$     & $\bullet$   & 0.5\\
%    4a\hspace{30pt}WhereExp* = True             & &               & $\bullet$     & $\bullet$     &    & 0.25\\
%    4b\hspace{30pt}WhereExp* = False                                           & & $\bullet$     &               &   & $\bullet$   & \textbf{1.0}\\
%    5\hspace{10pt}while(\$row = mysql\_fetch\_array(\$result)) \{                   & & $\bullet$  & $\bullet$     & $\bullet$     &  $\bullet$ & 0.5\\
%    6\hspace{20pt}echo \$row[`Name'] . `$<$br /$>$';                                    & &         & $\bullet$     & $\bullet$     &  & 0.25\\
%    \midrule
%    Pass/Fail Status                                                                & & F             & F             & P             & F \\
%    \bottomrule
%    \addlinespace
%    \multicolumn{6}{l}{* WhereExp: Age $>$= 25 AND Gender = `Female' OR Country $<>$ `USA'}
%\end{tabular}
%}
%    \caption{Suspiciousness scores computed by \tool{} for the PHP function in Figure~\ref{fig:example-phpcode}}\label{fig:suspiciousness-improved}
%\end{figure*}

Based on the monitoring results, \tool{} computes the suspiciousness
scores for all program entities, including the \query{WHERE}
conditions of SQL queries and other statements in the PHP
program. Figure~\ref{fig:suspiciousness-improved} illustrates the
score computation for the example in
Figure~\ref{fig:example-phpcode}. A test case is marked as
\code{Passed(P)} if the presence (or absence) of the
corresponding record in the actual output is as expected; otherwise,
it is marked as \code{Failed(F)}. For example, \code{Alice} does not
appear in the actual output, which is not expected; thus record 1 is a
\code{Failed} test case. Similarly, \code{Bob} is included in the
actual output while it should not, thus record 2 is also a
\code{Failed} test case. The only \code{Passed} test case is record 3,
where \code{Carol} is output as expected. The bullets for the 
statements indicate whether the entities are included in the slice for
the corresponding test case, which is established when \tool{}
monitors the execution trace of row-based test cases. 
The column \code{Sus.} shows Tarantula suspiciousness scores. 

%%for the corresponding program entities using Tarantula metric.

% (as explained in the motivating example).

As seen in Figure~\ref{fig:suspiciousness-improved}, the \code{False}
part of the \query{WHERE} expression on line 4b has a high
suspiciousness score (1.0), indicating that the program state when the
\query{WHERE} of the SQL query is evaluated to \code{False} is likely
incorrect, (i.e., the result set does not contain the corresponding
record whereas the record is expected to be included). Also, the
suspiciousness score of the
\query{WHERE} expression corresponding to the \code{False} case is
higher than the score of any other program entity, which suggests that
the predicates in the \query{WHERE} clause of the SQL query are most
likely to contain an error. In this example, the fault is located at
the last predicate of the \query{WHERE} condition (the operator in
\code{Country $<>$ `\$country'} should be
\code{`='}). 
The suspiciousness scores computed by \tool{} are, thus, useful in
localizing faulty SQL queries.  

%Next, we will explain how {\tool} localizes faulty predicates in a
%\code{WHERE} clause.


