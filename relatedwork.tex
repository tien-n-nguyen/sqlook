\section{Related Work}

A related work to {\tool} is from Clark {\em et al.}~\cite{ga-ase11},
%who introduced the first, state-of-the-art technique for 
In their {\em database-aware} fault localization for Web applications,
they extended Tarantula~\cite{tarantula05} to monitor at run-time the
host program's statements, SQL statements and attributes.
%They extend the idea of the statistical fault localization method in
%Tarantula~\cite{tarantula05} to monitor at runtime the host program's
%statement, the SQL statements, and SQL attribute tuples, and to assign
%them suspicious scores. A static SQL statement in the host program
%will be executed and become multiple SQL queries in different
%executions. 
They assume that those multiple SQL queries must be involved with a
different set of database attributes. As explained in Sections 1-2,
it is common in practice that those multiple SQL queries at run-time
are involved with a fixed set of attributes, and they vary only at the
literal values.
%If one of those unique SQL queries with a unique set of attributes is
%executed by more failing test cases than passing ones, it more likely
%contributes to the fault and is assigned higher score. However, as
%shown in Section~2, it is common that those multiple SQL queries at
%run-time are involved with a fixed set of database attributes, and
%they vary only at the literal values. 
Because there is one unique set of attributes, their method would not
be able to assign the higher score for that SQL statement (Sections
1-2).  {\tool} is also related to SQLook~\cite{icsm13}. However, there
are significant advances from {\tool} over SQLook. First, while SQLook
is focused only on bugs in PHP, {\tool} also identifies faulty
predicates in an SQL query after deciding that a WHERE clause is at
fault. Second, we peformed an empirical study on SQL-related bugs in
PHP to motivate our approach. Third, we developed a novel row-based
slicing across PHP and SQL. Finally, we conducted a more systematic
evaluation in detecting faulty SQL predicates.

ALTAR~\cite{altar-icst17} combines row-based dynamic slicing and delta
debugging to localize faults in SQL predicates. They introduce a
exoneration-based technique to isolate individual faulty clauses
within WHERE predicates. It uses row-based dynamic slicing to discover
suspicious clauses in WHERE predicates and removes non-faulty
clauses using a technique inspired by delta debugging. Later version
of ALTAR~\cite{altar-jss19} improves the exoneration-based technique
to make it more time efficient. In comparison, {\tool} uses predicate
switching that could be more effective than delta debugging for
predicates. ALTAR does not localize bugs within PHP code.

There exist fault-localization methods for single-language,
{\em data-centric} programs~\cite{dor-issta08,litvak10,saha11}, which
primarily interact with databases to get the data contents, process
and output them. Saha {\em et al.}~\cite{saha11} introduced a method
to localize faults in programs in ABAP (an SQL-like language). In
{\tool}, we apply the idea of using individual rows as test cases as
in their method.
%They developed a {\em key-based} slicing algorithm to break the trace
%into multiple slices where each slice maps to an entry in the output
%collection. That is, it takes into account the relevance of execution
%of statements to specific database keys that were used in creating
%those slices.  Then {\em semantic difference} between the slices that
%correspond to correct entries and those that correspond to incorrect
%ones will be computed to identify fault statements. 
However, they do not aim to support {\em multi-lingual} programs with
database code embedded in a host program. 
%
That method belongs to a class of techniques using dynamic
slicing~\cite{dor-issta08,litvak10,zhang-pldi04,cleve-icsm10}.
The fault is likely to be in the statements in the
program slice of incorrectly computed variables.
%%%, but not in the ones of correctly computed variables.
%agrawal-90, cleve-wcre08
%Hainaut and Cleve~\cite{cleve-icsm10,cleve-wcre08} capture dynamic
%traces via dynamic slicing and analysis on data-centric, embedded code
%and SQL logs in order to resolve the input queries being passed to a
%database in data-intensive systems. Their goal is more toward program
%evolution comprehension and reverse engineering. Other approaches for
%fault-localization in data-centric applications utilize static
%slicing. 
%PanayaAI~\cite{dor-issta08} computes statically a forward slice from
%select statements and infers how the returned data is manipulated in
%an ABAP program. 
Field-sensitive analysis is used to compute program dependence
in ERP system domain~\cite{litvak10}.  Other methods rely on
the differences between passing and failing
runs~\cite{dallmeier-ecoop05,mani-ase10}.

{\tool} is extended to support for SQL from statistical
fault localization
methods~\cite{tarantula05,abreu-ochiai-07,liblit-pldi05}.
%sober05, jones02
%The idea is that the counts of the exercising of different code
%elements in passing and failing runs can help localize faults. 
Such spectrum can be based on several program features such as branch,
complete-path, data-dependence, output, and execution
trace~\cite{harrold00,santelices09,zhang-fse09,apollo-tse10},
%apollo10
%Other techniques analyze 
statistical correlations between failure and CFG's
predicates~\cite{liblit-pldi05}, paths~\cite{holmes-icse09},
time spectra~\cite{timespectra08}, code
changes~\cite{stoerzer-fse06}. They are not database-aware.

%ren-issta07

%roposed an approach to localize faults via automated predicate
%switching. To localize a bug, searching for arbitrary changes to the
%program state is expensive due to the extremely large search
%space. Their idea is that the outcome of a branch is only true/false,
%the number of modified states resulting by predicate switching is far
%less than those possible through arbitrary state
%changes~\cite{zhang-icse06}. They showed that

Other methods modify program states at selected points in the
execution of a failing test. If the fault still occurs, it is likely
the bug~\cite{cleve-icse05,zhang-icse06,jeffrey-issta08}.  Zhang {\em
et al.}~\cite{zhang-icse06} localize faults by switching a predicate's
outcome at runtime. 
%%They showed that to do so can bring the execution to successful, thus,
%%help localize a bug. 
{\tool} localizes faults in the predicates of
a \code{where} statement. 
However, their method cannot handle the cases of multiple faults.
%{\tool} can also handle the cases of multiple faults in those
%predicates. 
Mutation testing~\cite{andrews06} has been applied to modify
an SQL program to assess the adequacy of a test suite~\cite{tuya07}.

% and to study the path divergence of faulty programs~\cite{tuya07}.

%do06, wong10

%tuya06
