\section{Empirical Evaluation}

%Let us present our empirical evaluation on \tool{}'s
%accuracy in localizing faults in Web applications. 

We conducted our empirical evaluation on {\tool}'s accuracy in fault
localization with three subject systems from \code{sourceforge.net}
with the sizes ranging from 19 KLOC to 156KLOC
(Table~\ref{tab:subject-systems}).  \code{AddressBook} is an
application to manage contact information and user
groups. \code{SchoolMate} is a web system for managing students,
teachers, and classes in a school. The largest subject system,
\code{ZenCart}, is an e-commerce application. These subject systems
have been used in prior work on fault detection for Web-based
applications~\cite{icsm13,apollo10}.
%All of these applications have interactions with a database through
%SQL queries.

%%%For each system, the number of SQL queries is shown in the column \code{Query}.

%database-interaction points is shown in the last column \code{DB-I}.

%sub

\subsection{Localizing Database-Related Faults}

%\vspace{0.04in}
%\noindent {\bf Experiment Setup.} 

%In the first experiment, for each system, we manually and randomly
%seeded two types of database-related faults: (1) SQL faults in the
%\query{WHERE} clause of SQL queries, and (2) PHP faults that affect
%the output of certain rows retrieved from a database query. Each
%mutant program has a single fault. Then, we created a failed test case
%that was resulted from the fault in either an SQL query or the PHP
%code. Table~\ref{tab:eval-database-aware} shows the result. Column
%\code{Mutants} shows the number of created mutants.
%Column \code{SQL faults/\% Rank} shows the percentage of statements in
%the execution trace that a~fixer need not examine by using
%\tool{}'s ranked list of suspicious statements. For example, for PHP
%faults, (s)he does not have to examine 86-98\% of the statements in
%the execution trace. Since the faulty SQL query is ranked at the top
%by \tool{}, column \code{SQL faults/\% Rank} shows 100\% for all three
%systems.

In the first experiment, we manually 
%and randomly
seeded two types of database-related faults: (1) SQL faults in the
\query{WHERE} clause of SQL queries, and (2) PHP faults that affect
the output data
%of certain rows 
retrieved from a database query. For (1), we mutated the operators of
the predicates in the \query{WHERE} clauses, while for (2), we used
the same mutation strategy as in the experiment in Clark {\em et
al.}~\cite{ga-ase11}. Each mutant program has a single fault. Then, we
created a failed test case that was resulted from the fault in either
an SQL query or the PHP code.

Table~\ref{tab:eval-database-aware} shows the result. Column
\code{Mutants} shows the number of created mutants.  Column \code{SQL
  faults/\% Rank} shows the percentage of statements in the execution
trace that a developer need not examine by using \tool{}'s ranked list of
suspicious statements. For example, for PHP faults, (s)he does not
have to examine 86-98\% of the statements in the execution
trace. Since the faulty SQL query is ranked at the top by \tool{},
column \code{SQL faults/\% Rank} shows 100\% for all three systems.

\begin{table}[t]
    \centering
%    \scriptsize
    \caption{Subject Systems}\label{tab:subject-systems}
\begin{tabular}{@{}lrrrr@{}}
    \toprule
    \textbf{System} & \textbf{Version} & \textbf{Files} & \textbf{LOC} & \textbf{Query} \\
    \midrule
    AddressBook (AB)    & 6.2.12    & 103   & 19K   & 52  \\ %184 (avg LOC) \\
    SchoolMate (SM)     & 1.5.4     & 63    & 50K   & 295 \\ %127 (avg LOC) \\
    ZenCart (ZC)        & 1.3.9     & 1,118  & 156K  & 2,171   \\ %140 (avg LOC) \\
    \bottomrule
\end{tabular}
\end{table}

\begin{table}[t]
    \centering
    \small
    \caption{Database-aware Fault Localization Results}\label{tab:eval-database-aware}
\begin{tabular}{@{}lcccccc@{}}
    \addlinespace
    \toprule
        & \multicolumn{2}{@{}c@{}}{\textbf{SQL faults}} & & \multicolumn{2}{@{}c@{}}{\textbf{PHP faults}} \\
    \cmidrule{2-3} \cmidrule{5-6}
    \textbf{System} & \textbf{Mutants} & \textbf{\% Rank} & & \textbf{Mutants} & \textbf{\% Rank} \\
    \midrule
    AddressBook   & 30  & 100\%     & &  9    & 98\% \\
    SchoolMate    & 54  & 100\%     & & 15    & 86\% \\
    ZenCart       & 91  & 100\%     & & 24    & 90\% \\
    \bottomrule
\end{tabular}
\end{table}

\begin{table}[t]
\centering
\small
\caption{SQL queries with Unique Set of Attributes}
\label{tab:sql}
\begin{tabular}{l|r|r|r|r||r|r}
  \hline
  % after \\: \hline or \cline{col1-col2} \cline{col3-col4} ...
  \#predicates & {\bf 0-1} & {\bf 2-3} & {\bf 4-7} & {\bf 8-10} & {\bf Checked} & {\bf Unique} \\
  \hline
  AddressBook & 17  & 2   & 15 &  0 & 34 & 29 \\
  SchoolMate  & 181 & 25  & 3  &  0 & 36 & 36 \\
  ZenCart     & 755 & 431 & 135 & 8 & 36 & 36 \\
  \hline
  Total       &   &  &  & & 106 & 101 \\
  \hline
\end{tabular}
\end{table}

\vspace{0.05in}
\noindent {\bf Comparison.} 
Since \tool{} uses information on individual rows in the test case,
\tool{} is able to rank the likelihood of faulty entities with one
test case only. In contrast, the state-of-the-art approach, Clark {\em
et al.}~\cite{ga-ase11}, was designed to require multiple test cases
to localize faults. For comparison, we took 106 randomly sampled SQL
queries and manually examined if a query involves a unique set of
attributes and varies only at the literal values. To do that, we first
divided the queries into groups according to the numbers of their
predicates. The number of sampled queries in each group is
proportional to its size. In Table~\ref{tab:sql}, the first columns
show the numbers of SQL queries with the corresponding numbers of
predicates. Columns \code{Checked} and \code{Unique} show the numbers
of queries that were checked and have unique sets of attributes,
respectively. As seen, among 106 random query samples, 101 of them
involve a unique set of attributes. Clark {\em et al.}~\cite{ga-ase11}
could not give those SQL statements higher suspicious scores even with
multiple test cases, thus, could not locate those 101 faults. In
contrast, {\tool} was able to rank all of them at the top position
of the resulting list.


\subsection{Localizing Faults in SQL's Predicates}

%\subsection{Experiment Setup}

%subject systems table

%%\vspace{0.04in}
%%\noindent {\bf Experiment Setup.}

In this experiment, we conducted two more studies. In our first study,
we used the same single-fault mutants for SQL queries as in the
previous experiment.  Each mutant program contains one single fault in
a predicate of a given SQL query. In our second study, each mutant
contains multiple faults.  In each study, we applied our two
algorithms to localize the seeded faults: (1) \tool{}'s single-fault
algorithm to detect SQL predicate faults assuming that there is a
single fault in the query (Section~\ref{single-fault-section}), and
(2) \tool{}'s multi-fault algorithm to detect SQL predicate faults
with no prior assumption about the number of faults
(Section~\ref{multi-fault-section}). The output of each algorithm is a
ranked list of the predicates in the SQL query, with the
highest-ranked predicate being the most likely one to contain the
fault. To evaluate {\tool}'s accuracy, we counted the number of times
that a faulty predicate appears in the top-ranked list of faulty
predicates returned by {\tool}.

\paragraph{Results on Localizing Single Faults}

%\vspace{0.04in} {\bf A. Results on Localizing Single Faults}

Table~\ref{tab:eval-single-faults} displays the evaluation result of
our single-fault and multi-fault localization algorithms. Column
\code{\#M} gives the number of mutants with single faults for each
system. Under column \code{Single-fault ranks}, the values in the five
sub-columns show the number of times the seeded fault appears in the
first to the fifth position of the resulting ranked list of
\tool{}'s single-fault algorithm.
%%Since the SQL queries that we used in our study have from 2-5
%%predicates, we
Table~\ref{tab:eval-single-faults} shows the top-1 to top-5 results.
In \code{AddressBook}, \tool{} correctly localized 27 out
of the 30 seeded faults with a single recommendation. With a ranked
list of 3 recommendations, it can correctly locate all 30 faults.

%one fault is ranked second in the list and two other faults appear
%in the third position.

\begin{table}[t]
    \centering
    \small
    \caption{Results on Single Seeded Faults in SQL queries}\label{tab:eval-single-faults}
    \setlength{\tabcolsep}{3.7pt}
\begin{tabular}{@{}lcccccccccccc@{}}
    \toprule
        &   & \multicolumn{5}{@{}c@{}}{\textbf{Single-fault ranks}} & & \multicolumn{5}{@{}c@{}}{\textbf{Multi-fault ranks}} \\
    \cmidrule{3-7} \cmidrule{9-13}
    \textbf{Sys.} & \textbf{\#M} & \textbf{1} & \textbf{2} & \textbf{3} & \textbf{4} & \textbf{5} & & \textbf{1} & \textbf{2} & \textbf{3} & \textbf{4} & \textbf{5} \\
    \midrule
    AB     & 30     & 27    & 1     & 2     & -     & -     &   & 27    & 1     & 2     & -     & - \\
    SM     & 54     & 43    & 11    & -     & -    & -     &   & 41    & 12    & 1     & -     & - \\
    ZC     & 91     & 77    & 8     & 4     & 2     & -     &   & 77    & 8     & 4     & 2     & - \\
    \midrule
%    \%     &        & 84    & 11    & 3     & 1     & 0     &   & 83    & 12    & 4     & 1     & 0 \\
    \multicolumn{2}{c}{\% coverage}  & 84    & 96    & 99     & 100     & 100     &   & 83    & 95    & 99     & 100     & 100 \\
    \bottomrule
\end{tabular}
\end{table}

Column \code{Multi-fault ranks} (Table~\ref{tab:eval-single-faults})
shows the number of times a fault appears in the corresponding
position in the ranked list of \tool{}'s multi-fault
algorithm.
%As seen, the results are {\em comparable} to those produced by the
%single-fault algorithm.
In \code{SchoolMate}, the single-fault algorithm performed slightly
better than multiple-fault algorithm.
%This is because the multi-fault algorithm does not have the prior
%knowledge that there is only one single fault in the program, while
%the other algorithm is specialized toward single faults.
The last row in Table~\ref{tab:eval-single-faults} (\code{\%
coverage}) shows the percentage the faults that are covered by the
corresponding top-ranked list. Overall, around 84\% of the seeded
faults are correctly identified by \tool{} with a single
recommendation.  The majority of the faults (96\%) can be found in the
top-2 results from two algorithms.

\paragraph{Results on Localizing Multiple Faults}

\begin{table}[t]
    \centering
%    \scriptsize
    \caption{Results on Multiple Seeded Faults in SQL queries}\label{tab:eval-multi-faults}
\begin{tabular}{@{}lcccccccccc@{}}
    \toprule
        &   & \multicolumn{4}{@{}c@{}}{\textbf{Single-fault ranks}} & & \multicolumn{4}{@{}c@{}}{\textbf{Multi-fault ranks}} \\
    \cmidrule{3-6} \cmidrule{8-11}
    \textbf{Sys.} & \textbf{\#M} & \textbf{2} & \textbf{3} & \textbf{4} & \textbf{5} & & \textbf{2} & \textbf{3} & \textbf{4} & \textbf{5} \\
    \midrule
    AB     & 14     & 9     & 3     & 2     & 0     &   & 10     & 2     & 2     & 0 \\
    SM     & 27     & 13    & 12    & 2     & 0     &   & 18     & 8     & 1     & 0 \\
    ZC     & 38     & 19    & 9     & 2     & 8     &   & 19     & 9     & 2     & 8 \\
    \midrule
%    \%     &        & 52    & 30    & 8     & 10    &   & 59    & 24     & 6     & 10 \\
    \multicolumn{2}{c}{\% coverage}  & 52    & 82    & 90     & 100    &   & 60    & 84     & 90     & 100 \\
    \bottomrule
\end{tabular}
\end{table}

To evaluate how well \tool{} performs on multiple seeded faults, we
conducted another study in which we seeded two errors for each
mutant. Table~\ref{tab:eval-multi-faults} shows the results.
% of \tool{}'s fault localization.
The rows and columns are similar to those in
Table~\ref{tab:eval-single-faults} except that column \code{top-1} is
not applicable since there are two errors. We consider that the faults
in a mutant are localized in the top-$n$ resulting list if the top-$n$
list covers both faulty predicates, and one of the faulty predicates
is ranked at the $n$ position. As seen, in \code{SchoolMate},
\tool{}'s multi-fault algorithm performed better than the single-fault
algorithm. Overall, 60\% of mutants have their faults detected in the
top-2 list, and \tool{}'s top-3 list covers both seeded faults for
84\% of mutant programs.

\noindent {\bf Time Efficiency.} In our experiments, we also
measured running time for the localization of each faulty
program. Each run took less than 1s without counting database
accessing time.

%84\% of the faults.

%In our implementation, we chose $Sus$ to be the Tarantula metric~\cite{tarantula05}. $Sus$ can be any other suspiciousness metric used in existing coverage-based statistical fault localization methods.
%Threshold $\sigma$ = ?

