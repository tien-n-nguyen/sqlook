\section{Introduction}

Web applications have played important roles in several aspects of our
society. In dynamic Web, a program, that is written in a host
language, e.g., PHP or ASP, interacts with a database engine to
retrieve, process, and present the data in a Web browser. In such a
program, there are statements that are used to construct a {\em query}
from string literals, variables' values, the values returned from
function calls, etc. A popular query language supported by several
database engines is SQL. The statements in a PHP/ASP program that
connect and communicate those queries to the database engine are
referred to as {\em database-interaction points}~\cite{ga-ase11}. The
results returned from a database are processed in displayed in a Web
browser.


%A dynamic Web application is often written in a language such as PHP
%or ASP that communicates with the databases to retrieve dynamic
%data, and then processes and displays them on the client-side
%browsers.
%In such a program, there exist program statements that are responsible
%for interacting with the databases, which are called {\em
%database-interaction points}. Before an interaction point, the program
%constructs a string {\em query} from string literals, variables'
%values, functions' returned values, etc. That query is written in a
%query language supported by the database (e.g. SQL). The returned
%result will be processed and displayed on client-side browsers.

%During the execution of the Web program, the string query is
%constructed and passed to the database to be executed there and the
%returned results will be stored via a variable(s) in the program for
%further manipulation or printing.

Automated fault localization is important in helping developers save
time and effort to fix the faults. However, localizing faults in such
a dynamic Web application is challenging due to the interaction
between the host program and the database engine.
%As in other types of application, dynamic Web applications have failures
%as well.
Several researchers have reported the faults caused by the data
communication between the application and the database
engine~\cite{ga-ase11,brooks-icst09}. While several automated
approaches have been introduced for traditional program
code~\cite{abreu-ochiai-07, tarantula05,liblit-pldi05} and
data-centric applications (i.e. single-language database
programs)~\cite{dor-issta08,litvak10,saha11}, very few approaches have
been proposed for {\em database-aware} fault
localization~\cite{ga-ase11,icsm13}, which needs to consider the data
communication between the host program in PHP/ASP and the database via
the queries. While Clark {\em et al.}~\cite{ga-ase11} require users to
provide the passing/failing test cases that must produce {\em
  different SQL queries} with unique structures, SQLook~\cite{icsm13}
cannot locate the detailed faulty predicates within the SQL queries.
These requirements are too strict since in reality, all
passing/failing test cases often create SQL queries with the same
structure and only literal values vary for each query.


%It was reported that there are common program failures in a dynamic
%Web application that~are caused by the interaction and passing of data
%between the application itself and the
%database~\cite{ga-ase11,brooks-icst09}. Localizing faults is a crucial
%task in software maintenance to improve software~quality.
%Thus, many researchers developed {\em automated fault localization
%  methods} for traditional, non-database
%applications~\cite{abreu-ochiai-07, tarantula05,liblit-pldi05} and for
%data-centric applications (i.e. single-language database
%programs)~\cite{dor-issta08,litvak10,saha11}. However, little
%attention has been paid to research in {\em database-aware} fault
%localization for dynamic Web applications, i.e., taking into account
%the interaction between the applications and databases via queries.

%----
%Clark {\em et al.}~\cite{ga-ase11} introduced an approach for
%database-aware fault localization in a dynamic Web application, which
%is based on Tarantula~\cite{tarantula05} to assign each statement a
%suspiciousness score computed based on the percentage of
%passing/failing test cases executing that statement. However, it has
%key restrictions on its effectiveness. First, it requires users~to
%provide the passing/failing test cases that must produce {\em
%  different unique queries} at run-time, i.e. produce different unique
%SQL structures in which one of the structures is~exercised by all
%failing test cases. It uses Tarantula to locate the faulty SQL
%statement corresponding to that structure. These requirements are too
%strict since in reality, all passing/failing test cases often create
%SQL queries with the same structure and only literal values vary for
%each query. Second, it cannot locate detailed faults {\em within} the
%queries, except for faulty SQL attributes.
%----

%(i.e. erroneous data columns referred by SQL queries).

%Clark {\em et al.}~\cite{ga-ase11} introduce the first approach for
%database-aware fault localization in a dynamic Web
%application. However, it has key restrictions. First, it requires
%users to provide the passing/failing test cases that must produce {\em
%unique queries} at run-time (i.e. SQL commands involving in unique
%sets of data table columns/attributes). The provided failing test
%cases must also correspond to one unique SQL command. These
%restrictions are very impractical because in reality, all
%passing/failing test cases often produce SQL commands with the fixed
%set of attributes and varied literal values. Thus, their approach is
%not practical because it considers that all test cases produce a
%unique SQL command. Second, it can not locate detailed faults {\em
%within} the SQL queries, except the faulty SQL attributes
%(i.e. erroneous data columns referred to by SQL queries).

%To address those issues, we introduce {\tool}, a novel method for
%database-aware fault localization in dynamic PHP Web
%applications with SQL support. We focus on the output errors caused
%by incorrect SQL queries with erroneous \code{WHERE} clauses or by the
%manipulation of the queries' result in PHP. 
%%%When the predicates in \code{WHERE} clause are faulty, it is very
%%%likely that the output is incorrect.

In this paper, we introduce {\tool}, a novel approach to
database-aware fault localization for dynamic PHP-based Web
applications. {\tool} can locate faulty PHP statements that handle
the SQL queries as well as the faulty SQL queries with erroneous
condition clauses.
%
{\tool} works in two phases. First, it localizes the faulty SQL
statements in PHP code. To achieve that, we make a test case from each
individual row in a database table. We use the presence/absence of
each row and its expected value to create a test case, instead of
using the entire output of the records in a database as a test case.
%use a {\em row-based test case technique} in which instead of
%considering the entire output of data records as a test case, we
%leverage the presence/absence of individual data rows and their
%expected values to create more test cases, called {\em row-based test
%  cases}. %Specifically,
%an input of the PHP program and a present/absent data
%row in the output forms a row-based test case. If such
If the presence/absence of a row in the database is expected,
that test case is considered as a passing one, and as a failing one
otherwise.

Importantly, we also use an instrumentation technique to monitor the
evaluation of the predicates of each condition clause (i.e.,
\code{WHERE} clause), rather than yielding the control of the query
execution to the database engine.
%
%Importantly, instead of passing the control to the database engine to
%execute a SQL command as in~Clark {\em et al.}  \cite{ga-ase11},
We instrument into a PHP interpreter the code for query execution to
monitor the evaluation of the predicates to decide whether each
\code{WHERE} clause has impact on the output of individual rows in the
database table. For each \code{WHERE} part, we assign a suspiciousness
score using a spectrum formula accordingly to the numbers of the
execution of that \code{WHERE} part in the passing and failing test
cases as in a normal spectrum-based fault localization
method~\cite{abreu-ochiai-07}. From that suspiciousness score, we
perform {\em row-based slicing} from SQL \code{WHERE} parts to the
statements in PHP code in the basis of individual rows in the database
table, in order to record the PHP statements that are executed to
produce the output of the rows.
%{\em instrument into a PHP interpreter} the code~to execute the SQL
%query and to {\em monitor} the evaluation of the predicates of a
%\code{WHERE} clause to determine if they affect the output of
%individual data records.
%Then, based on whether the \code{WHERE} parts are exercised frequently
%by passing/failing test cases, we apply Tarantula
%metric~\cite{tarantula05} to give a suspiciousness score for each
%\code{WHERE} part.
%
%Since our row-based test cases are for individual data rows, to
%compute the scores for PHP statements, {\tool} performs {\em row-based
%  slicing} across PHP statements and SQL parts to record the PHP
%statements that are exercised in the output of a row.

After deciding that a \code{WHERE} clause in an SQL query potentially
has a fault, {\tool} continues to localize specific predicates likely
responsible for the fault. To do that, when monitoring 
the execution of an SQL query, it also records the values of the
predicates in the \code{WHERE} clause.~It~then applies a predicate
switching technique in which if the boolean value of a predicate after
switching (i.e., \code{True} becomes \code{False},~or vice versa)
leads to a different result of the \query{WHERE} clause, which makes a
\emph{failed} test case become successful, the predicate's original
value is likely incorrect. The predicates whose switched values
change the results of more failed test cases are given higher
suspiciousness scores.


%
Our empirical evaluation shows that {\tool} can achieve higher
accuracy than the state-of-the-art approach. For single-fault
scenarios, around 84\% of the seeded faults are correctly identified
by {\tool} with a single recommendation. For multiple-fault scenarios,
60\% of the cases have their faults detected in the top-2 list.  Our
empirical evaluation shows that {\tool} achieves higher accuracy than
Clark {\em et al.}~\cite{ga-ase11}.


Our key contributions include:

1. A novel database-aware fault localization method for dynamic
Web applications to locate the faults in PHP statements and the
predicates of the \code{WHERE} expressions of SQL queries.
%The prototype tool provides a ranked list of suspicious PHP
%entities and the SQL predicates,

%2. A prototype database-aware fault-localization tool, {\tool} that
%provides a ranked list of suspicious PHP entities and the predicates
%in SQL queries,

2. An empirical evaluation to show {\tool}'s accuracy.

% and usefulness of {\tool} in helping developers in fault
%localization.

%Section~2 presents a motivating example. Details on {\tool} are
%described in Sections 3 and 4. Section 5 is for our
%evaluation. Related work is in Section 6. Conclusions appear~last.


