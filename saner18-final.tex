
%% bare_conf.tex
%% V1.4b
%% 2015/08/26
%% by Michael Shell
%% See:
%% http://www.michaelshell.org/
%% for current contact information.
%%
%% This is a skeleton file demonstrating the use of IEEEtran.cls
%% (requires IEEEtran.cls version 1.8b or later) with an IEEE
%% conference paper.
%%
%% Support sites:
%% http://www.michaelshell.org/tex/ieeetran/
%% http://www.ctan.org/pkg/ieeetran
%% and
%% http://www.ieee.org/

%%*************************************************************************
%% Legal Notice:
%% This code is offered as-is without any warranty either expressed or
%% implied; without even the implied warranty of MERCHANTABILITY or
%% FITNESS FOR A PARTICULAR PURPOSE! 
%% User assumes all risk.
%% In no event shall the IEEE or any contributor to this code be liable for
%% any damages or losses, including, but not limited to, incidental,
%% consequential, or any other damages, resulting from the use or misuse
%% of any information contained here.
%%
%% All comments are the opinions of their respective authors and are not
%% necessarily endorsed by the IEEE.
%%
%% This work is distributed under the LaTeX Project Public License (LPPL)
%% ( http://www.latex-project.org/ ) version 1.3, and may be freely used,
%% distributed and modified. A copy of the LPPL, version 1.3, is included
%% in the base LaTeX documentation of all distributions of LaTeX released
%% 2003/12/01 or later.
%% Retain all contribution notices and credits.
%% ** Modified files should be clearly indicated as such, including  **
%% ** renaming them and changing author support contact information. **
%%*************************************************************************


% *** Authors should verify (and, if needed, correct) their LaTeX system  ***
% *** with the testflow diagnostic prior to trusting their LaTeX platform ***
% *** with production work. The IEEE's font choices and paper sizes can   ***
% *** trigger bugs that do not appear when using other class files.       ***                          ***
% The testflow support page is at:
% http://www.michaelshell.org/tex/testflow/



\documentclass[conference]{IEEEtran}
% Some Computer Society conferences also require the compsoc mode option,
% but others use the standard conference format.
%
% If IEEEtran.cls has not been installed into the LaTeX system files,
% manually specify the path to it like:
% \documentclass[conference]{../sty/IEEEtran}





% Some very useful LaTeX packages include:
% (uncomment the ones you want to load)


% *** MISC UTILITY PACKAGES ***
%
%\usepackage{ifpdf}
% Heiko Oberdiek's ifpdf.sty is very useful if you need conditional
% compilation based on whether the output is pdf or dvi.
% usage:
% \ifpdf
%   % pdf code
% \else
%   % dvi code
% \fi
% The latest version of ifpdf.sty can be obtained from:
% http://www.ctan.org/pkg/ifpdf
% Also, note that IEEEtran.cls V1.7 and later provides a builtin
% \ifCLASSINFOpdf conditional that works the same way.
% When switching from latex to pdflatex and vice-versa, the compiler may
% have to be run twice to clear warning/error messages.






% *** CITATION PACKAGES ***
%
%\usepackage{cite}
% cite.sty was written by Donald Arseneau
% V1.6 and later of IEEEtran pre-defines the format of the cite.sty package
% \cite{} output to follow that of the IEEE. Loading the cite package will
% result in citation numbers being automatically sorted and properly
% "compressed/ranged". e.g., [1], [9], [2], [7], [5], [6] without using
% cite.sty will become [1], [2], [5]--[7], [9] using cite.sty. cite.sty's
% \cite will automatically add leading space, if needed. Use cite.sty's
% noadjust option (cite.sty V3.8 and later) if you want to turn this off
% such as if a citation ever needs to be enclosed in parenthesis.
% cite.sty is already installed on most LaTeX systems. Be sure and use
% version 5.0 (2009-03-20) and later if using hyperref.sty.
% The latest version can be obtained at:
% http://www.ctan.org/pkg/cite
% The documentation is contained in the cite.sty file itself.






% *** GRAPHICS RELATED PACKAGES ***
%
\ifCLASSINFOpdf
  % \usepackage[pdftex]{graphicx}
  % declare the path(s) where your graphic files are
  % \graphicspath{{../pdf/}{../jpeg/}}
  % and their extensions so you won't have to specify these with
  % every instance of \includegraphics
  % \DeclareGraphicsExtensions{.pdf,.jpeg,.png}
\else
  % or other class option (dvipsone, dvipdf, if not using dvips). graphicx
  % will default to the driver specified in the system graphics.cfg if no
  % driver is specified.
  % \usepackage[dvips]{graphicx}
  % declare the path(s) where your graphic files are
  % \graphicspath{{../eps/}}
  % and their extensions so you won't have to specify these with
  % every instance of \includegraphics
  % \DeclareGraphicsExtensions{.eps}
\fi
% graphicx was written by David Carlisle and Sebastian Rahtz. It is
% required if you want graphics, photos, etc. graphicx.sty is already
% installed on most LaTeX systems. The latest version and documentation
% can be obtained at: 
% http://www.ctan.org/pkg/graphicx
% Another good source of documentation is "Using Imported Graphics in
% LaTeX2e" by Keith Reckdahl which can be found at:
% http://www.ctan.org/pkg/epslatex
%
% latex, and pdflatex in dvi mode, support graphics in encapsulated
% postscript (.eps) format. pdflatex in pdf mode supports graphics
% in .pdf, .jpeg, .png and .mps (metapost) formats. Users should ensure
% that all non-photo figures use a vector format (.eps, .pdf, .mps) and
% not a bitmapped formats (.jpeg, .png). The IEEE frowns on bitmapped formats
% which can result in "jaggedy"/blurry rendering of lines and letters as
% well as large increases in file sizes.
%
% You can find documentation about the pdfTeX application at:
% http://www.tug.org/applications/pdftex





% *** MATH PACKAGES ***
%
%\usepackage{amsmath}
% A popular package from the American Mathematical Society that provides
% many useful and powerful commands for dealing with mathematics.
%
% Note that the amsmath package sets \interdisplaylinepenalty to 10000
% thus preventing page breaks from occurring within multiline equations. Use:
%\interdisplaylinepenalty=2500
% after loading amsmath to restore such page breaks as IEEEtran.cls normally
% does. amsmath.sty is already installed on most LaTeX systems. The latest
% version and documentation can be obtained at:
% http://www.ctan.org/pkg/amsmath





% *** SPECIALIZED LIST PACKAGES ***
%
%\usepackage{algorithmic}
% algorithmic.sty was written by Peter Williams and Rogerio Brito.
% This package provides an algorithmic environment fo describing algorithms.
% You can use the algorithmic environment in-text or within a figure
% environment to provide for a floating algorithm. Do NOT use the algorithm
% floating environment provided by algorithm.sty (by the same authors) or
% algorithm2e.sty (by Christophe Fiorio) as the IEEE does not use dedicated
% algorithm float types and packages that provide these will not provide
% correct IEEE style captions. The latest version and documentation of
% algorithmic.sty can be obtained at:
% http://www.ctan.org/pkg/algorithms
% Also of interest may be the (relatively newer and more customizable)
% algorithmicx.sty package by Szasz Janos:
% http://www.ctan.org/pkg/algorithmicx




% *** ALIGNMENT PACKAGES ***
%
%\usepackage{array}
% Frank Mittelbach's and David Carlisle's array.sty patches and improves
% the standard LaTeX2e array and tabular environments to provide better
% appearance and additional user controls. As the default LaTeX2e table
% generation code is lacking to the point of almost being broken with
% respect to the quality of the end results, all users are strongly
% advised to use an enhanced (at the very least that provided by array.sty)
% set of table tools. array.sty is already installed on most systems. The
% latest version and documentation can be obtained at:
% http://www.ctan.org/pkg/array


% IEEEtran contains the IEEEeqnarray family of commands that can be used to
% generate multiline equations as well as matrices, tables, etc., of high
% quality.




% *** SUBFIGURE PACKAGES ***
%\ifCLASSOPTIONcompsoc
%  \usepackage[caption=false,font=normalsize,labelfont=sf,textfont=sf]{subfig}
%\else
%  \usepackage[caption=false,font=footnotesize]{subfig}
%\fi
% subfig.sty, written by Steven Douglas Cochran, is the modern replacement
% for subfigure.sty, the latter of which is no longer maintained and is
% incompatible with some LaTeX packages including fixltx2e. However,
% subfig.sty requires and automatically loads Axel Sommerfeldt's caption.sty
% which will override IEEEtran.cls' handling of captions and this will result
% in non-IEEE style figure/table captions. To prevent this problem, be sure
% and invoke subfig.sty's "caption=false" package option (available since
% subfig.sty version 1.3, 2005/06/28) as this is will preserve IEEEtran.cls
% handling of captions.
% Note that the Computer Society format requires a larger sans serif font
% than the serif footnote size font used in traditional IEEE formatting
% and thus the need to invoke different subfig.sty package options depending
% on whether compsoc mode has been enabled.
%
% The latest version and documentation of subfig.sty can be obtained at:
% http://www.ctan.org/pkg/subfig




% *** FLOAT PACKAGES ***
%
%\usepackage{fixltx2e}
% fixltx2e, the successor to the earlier fix2col.sty, was written by
% Frank Mittelbach and David Carlisle. This package corrects a few problems
% in the LaTeX2e kernel, the most notable of which is that in current
% LaTeX2e releases, the ordering of single and double column floats is not
% guaranteed to be preserved. Thus, an unpatched LaTeX2e can allow a
% single column figure to be placed prior to an earlier double column
% figure.
% Be aware that LaTeX2e kernels dated 2015 and later have fixltx2e.sty's
% corrections already built into the system in which case a warning will
% be issued if an attempt is made to load fixltx2e.sty as it is no longer
% needed.
% The latest version and documentation can be found at:
% http://www.ctan.org/pkg/fixltx2e


%\usepackage{stfloats}
% stfloats.sty was written by Sigitas Tolusis. This package gives LaTeX2e
% the ability to do double column floats at the bottom of the page as well
% as the top. (e.g., "\begin{figure*}[!b]" is not normally possible in
% LaTeX2e). It also provides a command:
%\fnbelowfloat
% to enable the placement of footnotes below bottom floats (the standard
% LaTeX2e kernel puts them above bottom floats). This is an invasive package
% which rewrites many portions of the LaTeX2e float routines. It may not work
% with other packages that modify the LaTeX2e float routines. The latest
% version and documentation can be obtained at:
% http://www.ctan.org/pkg/stfloats
% Do not use the stfloats baselinefloat ability as the IEEE does not allow
% \baselineskip to stretch. Authors submitting work to the IEEE should note
% that the IEEE rarely uses double column equations and that authors should try
% to avoid such use. Do not be tempted to use the cuted.sty or midfloat.sty
% packages (also by Sigitas Tolusis) as the IEEE does not format its papers in
% such ways.
% Do not attempt to use stfloats with fixltx2e as they are incompatible.
% Instead, use Morten Hogholm'a dblfloatfix which combines the features
% of both fixltx2e and stfloats:
%
% \usepackage{dblfloatfix}
% The latest version can be found at:
% http://www.ctan.org/pkg/dblfloatfix




% *** PDF, URL AND HYPERLINK PACKAGES ***
%
%\usepackage{url}
% url.sty was written by Donald Arseneau. It provides better support for
% handling and breaking URLs. url.sty is already installed on most LaTeX
% systems. The latest version and documentation can be obtained at:
% http://www.ctan.org/pkg/url
% Basically, \url{my_url_here}.




% *** Do not adjust lengths that control margins, column widths, etc. ***
% *** Do not use packages that alter fonts (such as pslatex).         ***
% There should be no need to do such things with IEEEtran.cls V1.6 and later.
% (Unless specifically asked to do so by the journal or conference you plan
% to submit to, of course. )

\usepackage{amsmath}
\usepackage{booktabs}
%\usepackage{ctable}
%\usepackage{enumerate}
%\usepackage{epsf}
%\usepackage{fvrb-ex}
\usepackage{graphicx}
%\usepackage{latex8}
%\usepackage{latexsym}
\usepackage{listings}
\usepackage{multirow}
%\usepackage{rotating}
%\usepackage{times}
%\usepackage{tweaklist}
%\usepackage{ulem}
%\usepackage{url}
%\usepackage{hyperref}
\usepackage{textcomp}

\usepackage{balance}

%\newcommand{\tool}{DRC}
%\newcommand{\code}[1]{{\footnotesize\textsf{#1}}}

\newcommand{\tool}{SQLook}
\newcommand{\code}[1]{{\footnotesize\textsf{#1}}}
\newcommand{\query}[1]{{\scriptsize\textsf{#1}}}

%\newcommand{definition}{Definition}
%\newtheorem{theorem}{Theorem}

\newtheorem{definition}{Definition}
\newtheorem{theorem}{Theorem}
\newtheorem{proof}{Proof}

\lstset{
    language={}, emph={},
	mathescape=false, escapechar=@,
	basicstyle=\scriptsize\sffamily,
    numberstyle=\scriptsize\sffamily,
    emphstyle=\bfseries,
    numbers=left, stepnumber=1,
    frame=single, xleftmargin=15pt, xrightmargin=4pt, framexleftmargin=0pt,  framexrightmargin=0pt, numbersep=7pt, % xleftmargin=11pt
    columns=flexible, breaklines=true, showspaces=false, showstringspaces=true, showtabs=false, tabsize=4
}


% correct bad hyphenation here
\hyphenation{op-tical net-works semi-conduc-tor}

\IEEEoverridecommandlockouts

\begin{document}
%
% paper title
% Titles are generally capitalized except for words such as a, an, and, as,
% at, but, by, for, in, nor, of, on, or, the, to and up, which are usually
% not capitalized unless they are the first or last word of the title.
% Linebreaks \\ can be used within to get better formatting as desired.
% Do not put math or special symbols in the title.
%\title{Bare Demo of IEEEtran.cls\\ for IEEE Conferences}

\title{SQL-Aware Fault Localization in\\ PHP-Based Web Applications}

\author{\IEEEauthorblockN{Hung Viet Nguyen$^*$\thanks{$^*$The research work was done while the author was 
a Ph.D. student with Dr. Tien N. Nguyen}}
\IEEEauthorblockA{Google, USA}
%hungnv@iastate.edu}
\and
\IEEEauthorblockN{Hoan Anh Nguyen}
\IEEEauthorblockA{Iowa State University, USA\\
hoan@iastate.edu}
\and
\IEEEauthorblockN{Tien N. Nguyen}
\IEEEauthorblockA{University of Texas at Dallas, USA\\
tien.n.nguyen@utdallas.edu}
}


% author names and affiliations
% use a multiple column layout for up to three different
% affiliations
%\author{\IEEEauthorblockN{Hung Viet Nguyen, Hoan Anh Nguyen, Tung Thanh Nguyen, and Tien N. Nguyen}
%\IEEEauthorblockA{Electrical and Computer Engineering Department\\
%Iowa State University}
%Email: \{hungnv, hoan, tung, anhnt, tien\}@iastate.edu}
%}

%\author{\IEEEauthorblockN{First-Name Last-Name}
%\IEEEauthorblockA{Department Name\\
%University Name}
%Email: \{hungnv, hoan, tung, anhnt, tien\}@iastate.edu}
%Email: first.last@organization.edu
%}

%\author{\IEEEauthorblockN{Michael Shell}
%\IEEEauthorblockA{School of Electrical and\\Computer Engineering\\
%Georgia Institute of Technology\\
%Atlanta, Georgia 30332--0250\\
%Email: http://www.michaelshell.org/contact.html}
%\and
%\IEEEauthorblockN{Homer Simpson}
%\IEEEauthorblockA{Twentieth Century Fox\\
%Springfield, USA\\
%Email: homer@thesimpsons.com}
%\and
%\IEEEauthorblockN{James Kirk\\ and Montgomery Scott}
%\IEEEauthorblockA{Starfleet Academy\\
%San Francisco, California 96678--2391\\
%Telephone: (800) 555--1212\\
%Fax: (888) 555--1212}}


% author names and affiliations
% use a multiple column layout for up to three different

% conference papers do not typically use \thanks and this command
% is locked out in conference mode. If really needed, such as for
% the acknowledgment of grants, issue a \IEEEoverridecommandlockouts
% after \documentclass

% for over three affiliations, or if they all won't fit within the width
% of the page, use this alternative format:
% 
%\author{\IEEEauthorblockN{Michael Shell\IEEEauthorrefmark{1},
%Homer Simpson\IEEEauthorrefmark{2},
%James Kirk\IEEEauthorrefmark{3}, 
%Montgomery Scott\IEEEauthorrefmark{3} and
%Eldon Tyrell\IEEEauthorrefmark{4}}
%\IEEEauthorblockA{\IEEEauthorrefmark{1}School of Electrical and Computer Engineering\\
%Georgia Institute of Technology,
%Atlanta, Georgia 30332--0250\\ Email: see http://www.michaelshell.org/contact.html}
%\IEEEauthorblockA{\IEEEauthorrefmark{2}Twentieth Century Fox, Springfield, USA\\
%Email: homer@thesimpsons.com}
%\IEEEauthorblockA{\IEEEauthorrefmark{3}Starfleet Academy, San Francisco, California 96678-2391\\
%Telephone: (800) 555--1212, Fax: (888) 555--1212}
%\IEEEauthorblockA{\IEEEauthorrefmark{4}Tyrell Inc., 123 Replicant Street, Los Angeles, California 90210--4321}}




% use for special paper notices
%\IEEEspecialpapernotice{(Invited Paper)}




% make the title area
\maketitle

% As a general rule, do not put math, special symbols or citations
% in the abstract
%\begin{abstract}
%The abstract goes here.
%\end{abstract}

% no keywords

\begin{abstract}
Localizing software defects helps developers save time and effort
in their software maintainance tasks.
%  
In a PHP-based Web application, fault localization is not
straightforward due to the dynamic nature of code generation and the
complex interactions with the backend database engine(s). A fault
might occur in the PHP code or within the SQL query in the host
program that is sent to be to executed in the database engine.
%
This paper presents {\tool}, a novel database-aware fault localization
method/tool that is able to locate output faults in PHP code as well
as in the predicates of the \code{WHERE} expressions in SQL queries in
a PHP-based, dynamic Web application.
%
In {\tool}, we instrument an PHP interpreter to monitor the execution
of an SQL query and the evaluation of the SQL predicates to decide
whether a predicate affects the output of an individual data record.
To do so, 
% PHP interpreter is instrumented to execute an SQL query and to
% monitor the evaluation of those SQL predicates to determine if they
% affect the output process of individual data records.
we perform row-based slicing across PHP code and SQL queries to record
the program entities responsible for the output of each row in a
database table.
%
To locate faulty predicates, we use predicate switching to identify
the suspicious predicates in an SQL query.
%
Our empirical evaluation shows that {\tool} can achieve higher
accuracy than the state-of-the-art approach. For single-fault
scenarios, around 84\% of the seeded faults are correctly identified
by {\tool} with a single recommendation. For multiple-fault scenarios,
60\% of the cases have their faults detected in the top-2 list.
\end{abstract}


\begin{IEEEkeywords}
Fault Localization; PHP Web applications; 
\end{IEEEkeywords}

\section{Introduction}

Web applications have played important roles in several aspects of our
society. In dynamic Web, a program, that is written in a host
language, e.g., PHP or ASP, interacts with a database engine to
retrieve, process, and present the data in a Web browser. In such a
program, there are statements that are used to construct a {\em query}
from string literals, variables' values, the values returned from
function calls, etc. A popular query language supported by several
database engines is SQL. The statements in a PHP/ASP program that
connect and communicate those queries to the database engine are
referred to as {\em database-interaction points}~\cite{ga-ase11}. The
results returned from a database are processed in displayed in a Web
browser.


%A dynamic Web application is often written in a language such as PHP
%or ASP that communicates with the databases to retrieve dynamic
%data, and then processes and displays them on the client-side
%browsers.
%In such a program, there exist program statements that are responsible
%for interacting with the databases, which are called {\em
%database-interaction points}. Before an interaction point, the program
%constructs a string {\em query} from string literals, variables'
%values, functions' returned values, etc. That query is written in a
%query language supported by the database (e.g. SQL). The returned
%result will be processed and displayed on client-side browsers.

%During the execution of the Web program, the string query is
%constructed and passed to the database to be executed there and the
%returned results will be stored via a variable(s) in the program for
%further manipulation or printing.

Automated fault localization is important in helping developers save
time and effort to fix the faults. However, localizing faults in such
a dynamic Web application is challenging due to the interaction
between the host program and the database engine.
%As in other types of application, dynamic Web applications have failures
%as well.
Several researchers have reported the faults caused by the data
communication between the application and the database
engine~\cite{ga-ase11,brooks-icst09}. While several automated
approaches have been introduced for traditional program
code~\cite{abreu-ochiai-07, tarantula05,liblit-pldi05} and
data-centric applications (i.e. single-language database
programs)~\cite{dor-issta08,litvak10,saha11}, very few approaches have
been proposed for {\em database-aware} fault
localization~\cite{ga-ase11,icsm13}, which needs to consider the data
communication between the host program in PHP/ASP and the database via
the queries. While Clark {\em et al.}~\cite{ga-ase11} require users to
provide the passing/failing test cases that must produce {\em
  different SQL queries} with unique structures, SQLook~\cite{icsm13}
cannot locate the detailed faulty predicates within the SQL queries.
These requirements are too strict since in reality, all
passing/failing test cases often create SQL queries with the same
structure and only literal values vary for each query.


%It was reported that there are common program failures in a dynamic
%Web application that~are caused by the interaction and passing of data
%between the application itself and the
%database~\cite{ga-ase11,brooks-icst09}. Localizing faults is a crucial
%task in software maintenance to improve software~quality.
%Thus, many researchers developed {\em automated fault localization
%  methods} for traditional, non-database
%applications~\cite{abreu-ochiai-07, tarantula05,liblit-pldi05} and for
%data-centric applications (i.e. single-language database
%programs)~\cite{dor-issta08,litvak10,saha11}. However, little
%attention has been paid to research in {\em database-aware} fault
%localization for dynamic Web applications, i.e., taking into account
%the interaction between the applications and databases via queries.

%----
%Clark {\em et al.}~\cite{ga-ase11} introduced an approach for
%database-aware fault localization in a dynamic Web application, which
%is based on Tarantula~\cite{tarantula05} to assign each statement a
%suspiciousness score computed based on the percentage of
%passing/failing test cases executing that statement. However, it has
%key restrictions on its effectiveness. First, it requires users~to
%provide the passing/failing test cases that must produce {\em
%  different unique queries} at run-time, i.e. produce different unique
%SQL structures in which one of the structures is~exercised by all
%failing test cases. It uses Tarantula to locate the faulty SQL
%statement corresponding to that structure. These requirements are too
%strict since in reality, all passing/failing test cases often create
%SQL queries with the same structure and only literal values vary for
%each query. Second, it cannot locate detailed faults {\em within} the
%queries, except for faulty SQL attributes.
%----

%(i.e. erroneous data columns referred by SQL queries).

%Clark {\em et al.}~\cite{ga-ase11} introduce the first approach for
%database-aware fault localization in a dynamic Web
%application. However, it has key restrictions. First, it requires
%users to provide the passing/failing test cases that must produce {\em
%unique queries} at run-time (i.e. SQL commands involving in unique
%sets of data table columns/attributes). The provided failing test
%cases must also correspond to one unique SQL command. These
%restrictions are very impractical because in reality, all
%passing/failing test cases often produce SQL commands with the fixed
%set of attributes and varied literal values. Thus, their approach is
%not practical because it considers that all test cases produce a
%unique SQL command. Second, it can not locate detailed faults {\em
%within} the SQL queries, except the faulty SQL attributes
%(i.e. erroneous data columns referred to by SQL queries).

%To address those issues, we introduce {\tool}, a novel method for
%database-aware fault localization in dynamic PHP Web
%applications with SQL support. We focus on the output errors caused
%by incorrect SQL queries with erroneous \code{WHERE} clauses or by the
%manipulation of the queries' result in PHP. 
%%%When the predicates in \code{WHERE} clause are faulty, it is very
%%%likely that the output is incorrect.

In this paper, we introduce {\tool}, a novel approach to
database-aware fault localization for dynamic PHP-based Web
applications. {\tool} can locate faulty PHP statements that handle
the SQL queries as well as the faulty SQL queries with erroneous
condition clauses.
%
{\tool} works in two phases. First, it localizes the faulty SQL
statements in PHP code. To achieve that, we make a test case from each
individual row in a database table. We use the presence/absence of
each row and its expected value to create a test case, instead of
using the entire output of the records in a database as a test case.
%use a {\em row-based test case technique} in which instead of
%considering the entire output of data records as a test case, we
%leverage the presence/absence of individual data rows and their
%expected values to create more test cases, called {\em row-based test
%  cases}. %Specifically,
%an input of the PHP program and a present/absent data
%row in the output forms a row-based test case. If such
If the presence/absence of a row in the database is expected,
that test case is considered as a passing one, and as a failing one
otherwise.

Importantly, we also use an instrumentation technique to monitor the
evaluation of the predicates of each condition clause (i.e.,
\code{WHERE} clause), rather than yielding the control of the query
execution to the database engine.
%
%Importantly, instead of passing the control to the database engine to
%execute a SQL command as in~Clark {\em et al.}  \cite{ga-ase11},
We instrument into a PHP interpreter the code for query execution to
monitor the evaluation of the predicates to decide whether each
\code{WHERE} clause has impact on the output of individual rows in the
database table. For each \code{WHERE} part, we assign a suspiciousness
score using a spectrum formula accordingly to the numbers of the
execution of that \code{WHERE} part in the passing and failing test
cases as in a normal spectrum-based fault localization
method~\cite{abreu-ochiai-07}. From that suspiciousness score, we
perform {\em row-based slicing} from SQL \code{WHERE} parts to the
statements in PHP code in the basis of individual rows in the database
table, in order to record the PHP statements that are executed to
produce the output of the rows.
%{\em instrument into a PHP interpreter} the code~to execute the SQL
%query and to {\em monitor} the evaluation of the predicates of a
%\code{WHERE} clause to determine if they affect the output of
%individual data records.
%Then, based on whether the \code{WHERE} parts are exercised frequently
%by passing/failing test cases, we apply Tarantula
%metric~\cite{tarantula05} to give a suspiciousness score for each
%\code{WHERE} part.
%
%Since our row-based test cases are for individual data rows, to
%compute the scores for PHP statements, {\tool} performs {\em row-based
%  slicing} across PHP statements and SQL parts to record the PHP
%statements that are exercised in the output of a row.

After deciding that a \code{WHERE} clause in an SQL query potentially
has a fault, {\tool} continues to localize specific predicates likely
responsible for the fault. To do that, when monitoring 
the execution of an SQL query, it also records the values of the
predicates in the \code{WHERE} clause.~It~then applies a predicate
switching technique in which if the boolean value of a predicate after
switching (i.e., \code{True} becomes \code{False},~or vice versa)
leads to a different result of the \query{WHERE} clause, which makes a
\emph{failed} test case become successful, the predicate's original
value is likely incorrect. The predicates whose switched values
change the results of more failed test cases are given higher
suspiciousness scores.


We conduct several experiments to evaluate {\tool}. We use the dataset
that has been used in the prior work in fault localization for Web
applications with the total of 225K LOCs and 2,518 SQL queries.  Our
empirical evaluation shows that {\tool} can achieve higher accuracy
than the state-of-the-art approach in Clark {\em et
  al.}~\cite{ga-ase11}. For single-fault scenarios, around 84\% of the
seeded faults are correctly identified by {\tool} with a single
recommendation. For multiple-fault scenarios, 60\% of the cases have
their faults detected in the top-2 list.


In brief, the key contributions of this paper include:

1. A novel database-aware fault localization method for dynamic
Web applications to locate the faults in PHP statements and the
predicates of the \code{WHERE} expressions of SQL queries.
%The prototype tool provides a ranked list of suspicious PHP
%entities and the SQL predicates,

2. A prototype database-aware fault-localization tool, {\tool} that
provides a ranked list of suspicious PHP entities and the predicates
in SQL queries,

3. An empirical evaluation to show {\tool}'s accuracy, and usefulness
of {\tool} in helping developers in fault localization.

%Section~2 presents a motivating example. Details on {\tool} are
%described in Sections 3 and 4. Section 5 is for our
%evaluation. Related work is in Section 6. Conclusions appear~last.



\section{Motivation}

Let us start with an example to motivate our
approach. 

\subsection{Motivating Example}

\begin{table}
    \centering
    %    \footnotesize
    \small
    \caption{The \textsf{Users} database table~\cite{icsm13}}\label{tab:users-table}
\begin{tabular}{lllll}
  \addlinespace
  \toprule
  % after \\: \hline or \cline{col1-col2} \cline{col3-col4} ...
  \textbf{ID} & \textbf{Name} & \textbf{Age} & \textbf{Gender} & \textbf{Country} \\
  \midrule
  1 & Alice & 20 & Female & USA \\
  2 & Bob & 20 & Male & Canada \\
  3 & Carol & 25 & Female & Canada \\
  4 & Daniel & 30 & Male & USA \\
  \bottomrule
\end{tabular}
\end{table}

Table~\ref{tab:users-table} shows an example of a database
table with four records on four persons. The table \code{Users} has
four rows corresponding to four data records. The columns of the table
represent the {\em attributes} of each record. In this example, the
data record shows the \code{ID}, \code{Name}, \code{Age},
\code{Gender}, and \code{Country} for each person. For example, the
first row shows the information for a person with the name of
\code{Alice}. In a dynamic Web application, the data can be retrieved
and updated from PHP code with the embedded SQL queries as we will
explain next. For example, one can construct the following SQL query
in a PHP program as in \code{``SELECT Name FROM Users WHERE Age =
  20''}. The result is a {\em result set} containing the names of all
users whose age is 20.

%Web applications are often developed with dynamic data. The data is
%usually stored in databases, such as in the table shown in
%Table~\ref{tab:users-table}. The \code{Users} database table
%contains four rows (\emph{data records}). The five columns of the
%table indicate five \emph{attributes} associated with each record. For
%example, the first record has the information on a user with
%the attributes \code{ID=1, Name=Alice, Age=20, Gender=Female}, and
%\code{Country=USA}. This data can then be accessed or updated from the
%Web application through SQL queries. For instance, the SQL query
%\code{``SELECT Name FROM Users WHERE Age = 20''} returns a
%\emph{result set} containing the names of all users whose age is
%20.

As a PHP-based Web application communicates with a database, a fault
might occur at the code responsible for such
interaction. Figure~\ref{fig:example-phpcode} shows the PHP code that
contains a fault in its SQL query at line 3, resulting the incorrect
output. The goal of the function in Figure~\ref{fig:example-phpcode}
is to display the records of the persons whose ages are greater than
the specified age, and the genders and countries match with the
specified ones. However, the developer made a mistake at the PHP code
to construct the SQL query: the operator in the last predicate
\code{Country $<>$ `\$country'} should be \code{`='}, instead of
\code{`$<>$'}. The string representing this SQL query is constructed
from the string literals at line 3 and the arguments \code{age},
\code{gender}, and \code{country}, after the code responsible for the
preparation of the connection to the database at lines 1--2. The query
is sent to the database server to be executed at line 4 via the PHP
function \code{mysql\_query}. The result set is stored in the variable
\code{result}. Because the query is incorrect, the output result is
not correct as a consequence (lines 5--7 are used to loop through the
result set and display the users' names matched from the search).

%As a Web application interacts with a database through SQL queries,
%database-specific failures can occur. Fig.~\ref{fig:example-phpcode}
%shows an example of a PHP function that produces incorrect output
%values due to an error in its SQL query. The purpose of the function
%is to display the names of the users from the \code{Users} table
%(Table~\ref{tab:users-table}) that satisfy a searching criteria (by
%\code{age, gender}, or \code{country}). First, the connection to the
%database is established (lines 1-2,
%Fig.~\ref{fig:example-phpcode}). The \code{\$sql} variable (line 3)
%contains the SQL query that retrieves the users' names for a given
%search input. This query is then sent to the database server to be
%executed via the PHP function \code{mysql\_query} (line 4), and the
%returned result set is stored in the variable
%%\code{\$result}. Finally, the code on lines 5-7 is used to loop
%through the records in the result set and display the corresponding
%names of the users found via the search. Note that this function
%contains an SQL fault: on line 3, the operator in the last predicate
%\code{Country $<>$ `\$country'} of the SQL query should be \code{`='}
%instead of \code{`$<>$'}.

\begin{figure}[t]
    {\small\sffamily
    %\setlength{\tabcolsep}{6pt}
    \renewcommand{\arraystretch}{1.3}
    {\normalfont\normalsize A PHP function with an SQL query error on line 3:}\\
\begin{tabular}{@{}p{\columnwidth}@{}}
    \toprule
    function displaySearchResults(\$age, \$gender, \$country) \{ \\
    1\hspace{5pt}\$con = mysql\_connect(`localhost', `admin', `password'); \\
    2\hspace{5pt}mysql\_select\_db(`my\_database', \$con); \\
    3\hspace{5pt}\$sql = ``SELECT Name FROM Users WHERE Age $>=$ \$age \\
    \hspace{20pt}AND Gender = `\$gender' OR \textbf{Country $<>$ `\$country'} ''; \\
    4\hspace{5pt}\$result = mysql\_query(\$sql); \\
    5\hspace{5pt}while(\$row = mysql\_fetch\_array(\$result)) \{ \\
    6\hspace{15pt}echo \$row[`Name'] . `$<$br $/>$'; \\
    7\hspace{5pt}\} \} \\
    \bottomrule
\end{tabular}

\vspace{5pt}
{\normalfont\normalsize Expected SQL query on line 3:}\\
\begin{tabular}{@{}p{\columnwidth}@{}}
    \toprule
    3\hspace{5pt}\$sql = ``SELECT Name FROM Users WHERE Age $>=$ \$age \\
    \hspace{20pt}AND Gender = `\$gender' OR \textbf{Country = `\$country'} ''; \\
    \bottomrule
\end{tabular}}
    \caption{An PHP function with an SQL query error~\cite{icsm13}}\label{fig:example-phpcode}
\end{figure}

\begin{figure}[t]
    \centering
    %    \footnotesize
    \small
\begin{minipage}[t]{0.985\columnwidth}
Search Input: \textsf{\$age=25, \$gender=`Female', \$country=`USA'}
\end{minipage}

\vspace{4pt}
\begin{minipage}[t]{0.63\columnwidth}
Actual SQL query:
\begin{lstlisting}[numbers=none,xleftmargin=4pt]
SELECT Name FROM Users WHERE
    Age >= 25 AND Gender = `Female'
        OR @\textbf{Country$<>$`USA'}@
\end{lstlisting}
\end{minipage}
\hspace{0.03\columnwidth}
\begin{minipage}[t]{0.30\columnwidth}
Actual output:
\begin{lstlisting}[numbers=none,xleftmargin=4pt]
Bob
Carol
\end{lstlisting}
\end{minipage}

\begin{minipage}[t]{0.63\columnwidth}
Expected SQL query:
\begin{lstlisting}[numbers=none,xleftmargin=4pt]
SELECT Name FROM Users WHERE
    Age >= 25 AND Gender = `Female'
        OR @\textbf{Country=`USA'}@
\end{lstlisting}
\end{minipage}
\hspace{0.03\columnwidth}
\begin{minipage}[t]{0.30\columnwidth}
Expected output:
\begin{lstlisting}[numbers=none,xleftmargin=4pt]
Alice
Carol
Daniel
\end{lstlisting}
\end{minipage}
%
%\vspace{-7pt}
    \caption{Output of the PHP function in Figure~\ref{fig:example-phpcode}~\cite{icsm13}}\label{fig:example-output}
\end{figure}

%for a given test case

Due to that error, the output result is
unexpected. Figure~\ref{fig:example-output} shows actual and expected
SQL queries and their corresponding output results given the search
input \code{\$age=25, \$gender=`Female', \$country=`USA'}. Because
the SQL query was incorrectly constructed at the last predicate, the
actual output is not the expected one. The records of Alice and Daniel
are supposed to be retrieved, however, were left out. The record of
Bob was returned and included in the output, while it must not.

%Due to that error in the SQL query, the function does~not display the
%expected results. Fig.~\ref{fig:example-output} shows the actual and
%expected output values given the search input \code{\$age=25,
%\$gender=`Female', \$country=`USA'}. In the correct SQL query, the
%three-predicate condition determining the values in the returned
%result set is: \code{Age $>=$ 25} \code{AND} \code{Gender = `Female'}
%\code{OR} \code{Country = `USA'}. Since the actual query has a fault
%in the last predicate, there is a mismatch between its actual and
%expected outputs. As seen, Alice's and Daniel's names are expected to
%be found in the result, but are not displayed. Meanwhile, Bob's name
%is included in the actual output although it must not.

\begin{figure}[t]
    \centering
    %    \scriptsize
    \small
    \setlength{\tabcolsep}{2pt}
    \renewcommand{\arraystretch}{1.1}
{\sffamily
\begin{tabular}{@{}llllll@{}}
    \toprule
                                                                                    & \multicolumn{4}{c}{Test Cases} & Sus.\\
                                                                                    \cmidrule{2-5}
    function displaySearchResults(\$age, \$gender, \$country) \{                    & 1         & 2         & \ldots         & $n$ \\
    \ldots                                                                          & $\bullet$ & $\bullet$ & \ldots    & $\bullet$ & 0.5\\
    4\hspace{5pt}\$result = mysql\_query(\$sql);                                   & $\bullet$ & $\bullet$ & \ldots    & $\bullet$ & 0.5\\
    4$\ast$\hspace{10pt}SELECT...Age$>$=? AND Gender=? OR Country$<>$?                  & $\bullet$ & $\bullet$ & \ldots    & $\bullet$ & \textbf{0.5}\\
    5\hspace{5pt}while(\$row = mysql\_fetch\_array(\$result)) \{                    & $\bullet$ & $\bullet$ & \ldots    & $\bullet$ & 0.5\\
    6\hspace{20pt}echo \$row[`Name'] . `$<$br /$>$';                                    & $\bullet$ & $\bullet$ & \ldots    & $\bullet$ & 0.5\\
    \midrule
    Pass/Fail Status                                                                & F       & P     & \ldots    & F \\
    \bottomrule
\end{tabular}
}
    \caption{Suspiciousness scores of the PHP statements and SQL queries in Figure~\ref{fig:example-phpcode} computed using the technique by Clark et al. and Tarantula metric~\cite{icsm13}}\label{fig:suspiciousness}
\end{figure}

%Given the mismatch between the actual output and the expected one, 

%Tien

It is challenging to localize such fault causing that errorenous
output. The error was made in the construction of the string
representing the SQL query within the PHP code. In other words, the
error occurs within the embedded string in the PHP code.  Even more
challenging is the fact that the execution of the query is done at the
database server, which is separate from where the PHP program is
executed to create the SQL query. In brief, to localize this kind of
faults/bugs, a tool needs to be {\em database-aware}, i.e., need to
consider the communications between PHP-based Web application and the
database via the queries.


%Given such mismatch, it is not obvious which part of the program
%accounts for the error. In this example, the fault lies at a predicate
%of an SQL query, whereas the Web application is written in
%PHP. Moreover, the actual execution of the SQL query occurs at the
%database server, which is separate from the main PHP program where the
%SQL query is created. Thus, the process of localizing this type of
%fault needs to be \emph{database-aware}, namely taking into account
%the interaction between the Web application and the database via
%queries.

Although there exist many fault localization techniques for
single-language programs (e.g. \cite{abreu-ochiai-07},
\cite{tarantula05}), \cite{liblit-pldi05}), and data-centric programs
(e.g. \cite{dor-issta08}, \cite{litvak10}, \cite{saha11}), little
attention has been given to database-specific faults in a multilingual
Web application.
%
A state-of-the-art approach in database-aware fault localization for
Web applications is Clark {\em et al.}~\cite{ga-ase11}. Their tool
monitors the execution of the PHP statements in the main program and
that of the generated SQL queries. It then assigns the
\emph{suspiciousness} scores to the PHP statements as well as the SQL
queries using the idea from the spectrum-based fault localization
approaches (Tarantula~\cite{tarantula05} and
Ochiai~\cite{abreu-ochiai-07}).
%monitors SQL queries generated at runtime and compute the
%\emph{suspiciousness} scores for those SQL queries and their
%attributes, as well as the statements in the main program.
%
The rationale in computing suspiciousness scores is to contrast the
runtime behaviors of correct and incorrect executions of the program.
If a program entity (program statement or predicate) is exercised by
more failing test cases than passing ones, it is more likely to be
responsible for the failure, and thus assigned with a higher
suspiciousness~score.

Let us explain how Clark {\em et al.}'s method~\cite{ga-ase11} works
in the database-aware fashion. It assigns the suspiciousness scores to
the program entities including PHP statements, SQL queries, and SQL
attributes.  Fig.~\ref{fig:suspiciousness} illustrates the computation
of suspiciousness scores for our running example. Line 4 shows
the PHP statement with \code{mysql\_query} while line 4$\ast$ displays
the SQL query produced by the code at line 4 at run-time. We use
the question marks to denote the literal numbers or strings.
%
%Line 4$\ast$ shows the SQL query executed by the PHP
%statement \code{mysql\_query} on line 4 at runtime, with the question
%marks indicating literal values (numbers/strings).
%
If multiple {\em unique SQL queries}, each with a different set of
attributes, are produced by the PHP code at line 4, their tool will assign
the suspiciousness scores to individual SQL queries.
%If the PHP \code{mysql\_query} statement executes multiple
%\emph{unique} SQL queries (each with a different set of attributes) in
%different executions, the suspiciousness scores of the individual
%queries will be computed.

%Compared with the previous fault localization approaches, Clark {\em
%et al.}'s method~\cite{ga-ase11} is database-aware in that it
%considers SQL queries or SQL attributes as program entities and also
%computes their 
%suspiciousness scores. For example, Fig.~\ref{fig:suspiciousness}
%illustrates the computation of suspiciousness scores for the program
%entities in Fig.~\ref{fig:example-phpcode} including SQL
%queries. Line 4$\ast$ shows the SQL query executed by the PHP
%statement \code{mysql\_query} on line 4 at runtime, with the question
%marks indicating literal values (numbers/strings). If the PHP
%\code{mysql\_query} statement executes multiple \emph{unique} SQL
%queries (each with a different set of attributes) in different
%executions, the suspiciousness scores of the individual queries will
%be computed.

The key idea of that method is that if some SQL queries with sets of
attributes are exercised in the execution of the passing test cases
and some other queries with different sets of attributes are executed
in those of failing ones, we can distinguish them and assign more
suspiciousness scores for the queries corresponding to the failing
ones. However, the SQL queries (executed by the PHP statement
\code{mysql\_query}) in different executions often have the same set
of attributes with the same structure. They differ from one another in
the literal values in the predicates. For example, the queries in the
running example would have the same structure \code{``SELECT Name FROM
  Users WHERE Age $>$= ? AND Gender = ? OR Country $<>$ ?''}, and the
same set of attributes \code{\{Name, Age, Gender, Country\}}. The
literals for ages, genders, and countries might vary for different
queries. Because the queries all have the same structure with
different literals, the numbers of passing and failing test cases
going through each of the queries are the same. Thus, they will be
given the same suspiciousness scores as the PHP statement
\code{mysql\_query}.

%That computation is based on the idea that if some of unique
%SQL queries are executed by more failing test cases than passing ones,
%they will have higher scores. This strategy is useful when there are
%multiple unique SQL queries that expose different behaviors in passing
%and failing test cases. However, in practice, a PHP
%\code{mysql\_query} statement often executes only one SQL
%query, in which the set of attributes is fixed, and the concrete SQL
%queries in different executions have the same structure and vary only
%at the literal values. This phenomenon is also reported by the
%authors~\cite{ga-ase11}. As an illustration, the unique query
%in this example (line 4$\ast$ of Fig.~\ref{fig:suspiciousness}) is
%\code{``SELECT Name FROM Users WHERE Age $>$= ? AND Gender = ? OR
%Country $<>$ ?''}, with the set of attributes \code{\{Name, Age,
%Gender, Country\}}.

Figure~\ref{fig:suspiciousness} shows the coverage information and the
suspiciousness scores given by Clark {\em et al.}~\cite{ga-ase11}.
The rows correspond to the statements and queries, and the columns are
for the test cases. A dark circle shows the execution of a statement
in a test case. The last row shows the status of passing and failing
of the test cases. The suspiciousness scores for the rows are computed
using the Tarantula (\cite{tarantula05}) metric:
\begin{equation}
\small
\label{eq:tarantula}
    S(e) = \dfrac{\tfrac{Failed(e)}{TotalFailed}}{\tfrac{Passed(e)}{TotalPassed} + \tfrac{Failed(e)}{TotalFailed}}
\end{equation}
where $Passed(e)$ is the number of passing test cases that execute
$e$, $Failed(e)$ is the number of failing test cases that execute $e$,
and $TotalPassed$ and $TotalFailed$ are the respective total numbers
of passing and failing test cases.

%Since there is one unique SQL query executed by the PHP
%\code{mysql\_query} statement, the coverage of the SQL query in the
%passing and failing test cases is the same as the \code{mysql\_query}
%statement, and therefore its suspiciousness score does not provide
%further information about the location of the fault. In
%Fig.~\ref{fig:suspiciousness}, the bullets indicate the program
%entities that are exercised by a given test case. At the bottom row,
%the letters \code{P} and \code{F} specify a passing and failing test
%case, respectively. Column \code{Sus.} shows the suspiciousness score
%$S(e)$ for a program entity $e$ using the Tarantula
%(\cite{tarantula05}) metric:
%\begin{equation}
%\small
%\label{eq:tarantula}
%    S(e) = \dfrac{\tfrac{Failed(e)}{TotalFailed}}{\tfrac{Passed(e)}{TotalPassed} + \tfrac{Failed(e)}{TotalFailed}}
%\end{equation}
%where $Passed(e)$ is the number of passing test cases that execute
%$e$, $Failed(e)$ is the number of failing test cases that execute $e$,
%and $TotalPassed$ and $TotalFailed$ are the respective total numbers
%of passing and failing test cases.

As seen in Figure~\ref{fig:suspiciousness}, because the numbers of
passing/failing test cases going through the PHP statements and SQL
queries are the same for the statements and queries, they are all
given the same score (0.5). In fact, for any spectrum-based fault
localization methods would give the same suspiciousness score for
those lines in this example because same set of program entities is
executed in every passing or failing test case, regardless of the test
suite. Therefore, the faulty queries are not idenfied. In other words,
to localize this kind of faults, we need to develop a {\em
  database-aware} fault localization method that considers the
interaction between the PHP code and the database via the queries.

%Although there is a fault in the SQL query, its suspiciousness score
%(0.5) is the same as the PHP \code{mysql\_query} statement that
%executes it. In fact, all the program entities always have the same
%score even when a different suspiciousness metric is used, since the
%same set of program entities is executed in every passing or failing
%test case (regardless of the test suite). From those suspiciousness
%scores, no further information about the fault is gained. This
%limitation motivated us to develop a new method to better localize
%database-specific faults.

\subsection{Approach Overview}

The example illustrates an \emph{incorrect output failure} that may
occur as a Web application interacts with a database. The incorrect
%values in the
returned results are often caused by the way the data records are
selected from a database table, which is specified by the
\query{WHERE} clause of an SQL \query{SELECT} query. In this paper, we
target incorrect output failures due to errors in the \query{WHERE}
part of SQL queries. Another type of database-specific failure is
\emph{execution failure}, which occurs if the SQL query has incorrect
syntax or specifies an invalid operation with the database (e.g. the
query refers to a non-existent database table or attribute). While
execution failures can be fixed by checking the syntax of queries and
ensuring their conformance to the database schemas, incorrect output
failures are more difficult to resolve since they are caused by
semantic errors in the queries.

To learn more on SQL queries, we conducted an exploratory study. We
collected three open-source dynamic Web applications from SourceForge
(\code{AddressBook}, \code{SchoolMate}, and \code{ZenCart}) with a
total of 1,284 PHP files and 225 KLOCs. We wrote a tool to analyze
those files and found 2,518 SQL queries to the databases with 2,672
predicates in \code{WHERE} clauses. There are almost 2 SQL queries in
a PHP file. A query could have up to 10 predicates. There are 304
queries with 3-10 predicates. There are 5-25 lines of PHP code for
each SQL query. Thus, if a tool such as in Clark {\em et
al.}~\cite{ga-ase11} reports to developers that a query is faulty
without details on specific predicates, it would be not efficient for
them to locate the fault. Those numbers motivated us to develop a tool
to help localize the defects due to errors in the predicates of
\code{WHERE} clauses.

In designing our solution, we leverage the fact that the output of a
query reveals information about the correctness of individual records
in the output. If the actual output of the program does not match with
the expected one, we can further analyze individual records in the
actual output to determine which records result in the mismatch. For
example, the actual and expected outputs for the given test case in
Fig.~\ref{fig:example-output} reveal that \code{Carol} is a correct
record whereas \code{Bob} is an incorrect one. To locate the fault,
a developer would then examine the three predicates in the SQL query
that returns \code{Bob}. (S)he would recognize that two
predicates \code{Age $>$= 25} and \code{Gender = `Female'} evaluate to
\code{false} since Bob's age is \code{20} and his gender is
\code{`Male'}. Thus, the first two predicates do not contribute to the
output of \code{Bob}. Only the last predicate whose value is true
determines its presence. Thus, the last predicate is most suspicious.
%The developer could conclude that it most likely causes the fault.

% incorrect output.
%In fact, it is the source of the fault in this example.

%Note that we assume the table \code{Users} and the table column \code{Name} are correct; otherwise, it would have led to an execution failure, which is not the focus of this paper.

Based on the above ideas, we develop {\tool}, a database-aware fault
localization method that works at two levels:

(1) \tool{} localizes the {\em faulty SQL query} by monitoring the
output of individual records and computing the query's suspiciousness
score based on the correctness of these records;

(2) Given an SQL query that is likely to be faulty, \tool{} examines
the predicates in the {\em \query{WHERE} part} of the SQL query to
identify which predicate is responsible for the incorrect output.
%Next section will explain the step (1).

%, we present these two
% methods.

\section{Localizing Faulty SQL Queries}

%\begin{figure}[tbp]
%  \centering
%  \includegraphics[width=\columnwidth]{images/SQL-Transformed2.eps}
%  \caption{Instrumented PHP interpreter to monitor the execution of SQL queries}\label{fig:SQL-Transformed}
%\end{figure}

\begin{figure}[tbp]
  \centering
  \includegraphics[width=\columnwidth]{images/SQL-Transformed4.eps} %2
 \caption{Instrumented PHP interpreter to monitor the execution of SQL queries~\cite{icsm13}}\label{fig:SQL-Transformed}
\end{figure}


In this section, we describe the steps in {\tool} to localize the
faulty SQL queries. In addition to assign the suspiciousness scores
for the PHP statements, we also aim to assign scores to SQL
\code{WHERE} clauses of the SQL queries that are handled at the
database-interaction points. Those points are the PHP statements
responsible for the interaction with the database engines, e.g., line
4 of Figure~\ref{fig:example-phpcode}.  To achieve that, we have the
following key design ideas:

%In a PHP Web application, program faults may be found in regular PHP
%statements or those that interact with the database, called
%\emph{database-interaction points} (e.g. line 4 of
%Figure~\ref{fig:example-phpcode}). The goal of this step is to
%localize these faults, specifically to decide if a
%database-interaction point contains a fault(s) at its \query{WHERE}
%clause or not. To do that, {\tool} uses Tarantula~\cite{tarantula05}
%to compute the suspiciousness scores for all entities in PHP and in
%SQL \code{WHERE} clauses. To avoid the issues as in Clark {\em et
%  al.}~\cite{ga-ase11}'s approach, we have following key design
%strategies:



(1) {\em Row-based test cases}. Instead of using entire database table
and its expected output as one single test case, we use each
individual record and its presence/absence in the output as a test
case. For example, in the running example, we can create four
row-based test cases: 1) the input is the triple \code{\$age=25,
  \$gender=`Female', \$country=`USA'}, and the output is the name
\code{Alice}, 2) the same input and the output is the absence of the
name \code{Bob}, 3) the same input and the output is the name
\code{Carol}, and 4) the same input and the output is the name
\code{Daniel}.

%Instead of viewing the input and entire expected output from the
%database as one test case, we analyze individual data records in the
%actual and expected outputs to create row-based test cases. For
%example, in Figure~\ref{fig:example-output}, we have four row-based
%test cases: 1) the input \code{\$age=25, \$gender=`Female',
%  \$country=`USA'} and the output of \code{Alice}'s record, 2) that
%input and the absence of \code{Bob}'s, 3) that input and the output of
%\code{Carol}'s, and 4) that input and the output of \code{Daniel}'s.

(2) {\em Monitoring the execution of PHP and SQL entities via
  instrumentation.} We instrument into an PHP interpreter the code to
execute the SQL query and to record the evaluation of the \code{WHERE}
clause for each row-based test case. {\tool} will record the
evaluation of a \code{WHERE} clause to \code{True} or \code{False} and
the corresponding presence or absence of the output for a row-based
test case. It then records the \code{True} or \code{False} part of the
\code{WHERE} clause being exercised for that row-based test case.
Therefore, the suspiciousness score can now be applied not only to PHP
code, but also to the deeper level of the \code{True}/\code{False}
parts of a \code{WHERE} clause. In brief, we now can assign the
suspiciousness scores to PHP statements and \code{WHERE} clauses.


%The suspiciousness scores are given to PHP statements and SQL
%\code{WHERE} clause(s). Instead of passing the control to the database
%engine to execute an SQL command as in \cite{ga-ase11}, {\tool}
%instruments into an PHP interpreter the code to execute the SQL
%command (Figure~\ref{fig:SQL-Transformed}) and to observe the
%evaluation of the \code{WHERE} clause with respect to every row-based
%test case. Because the \code{WHERE} clause's value decides if the
%output of a row-based test case is present or not, {\tool} needs to
%record if that clause is evaluated to \code{True} or
%\code{False}. From that, it knows which part (\code{True} or
%\code{False}) of~a \code{WHERE} clause is exercised for a row-based
%test case. Thus, the suspiciousness metric can now be applied not only
%to PHP code, but also to the deeper level of the two parts of a
%\code{WHERE} clause.

(3) {\em Row-based Slicing across PHP and SQL.} Since {\tool} uses
row-based test cases, it needs to record the PHP statements that are
exercised in the execution of such a test case, i.e. the PHP
statements that are involved in the output of a data record (i.e. a
row). To do that, during the monitoring in step (2), it computes the
forward slice corresponding to each data row across the PHP statements
and the two parts (\code{True}/\code{False}) of the \code{WHERE}
clause.
%To do that, {\tool} uses the statistical suspiciousness metric
%Tarantula~\cite{tarantula05} to compute the suspiciousness scores for
%all entities in PHP and in SQL \code{WHERE} clause(s). To avoid the
%issue(s) as in Clark {\em et al.}~\cite{ga-ase11}'s approach, we have
%following key design strategies:

%In a PHP web application, program faults may be found in regular PHP
%statements or those that interact with the database system (i.e.,
%\emph{database-interaction points}). To detect these faults, \tool{}
%uses a suspiciousness metric $Sus$, which can be any of the existing
%statistical suspiciousness metrics (such as
%Tarantula~\cite{tarantula05} and Ochiai~\cite{abreu-ochiai-07}), and
%computes the suspiciousness scores for all statements in a PHP program
%including database interaction points. However, as illustrated in the
%motivating example, the faults that exist in database-interaction
%points may not be identified by the metric $Sus$. To further handle
%this type of faults, \tool{} monitors the execution of SQL queries at
%database-interaction points where the conditions given by the
%\query{WHERE} parts of the SQL queries are evaluated to select data
%records from one or more database tables. If the evaluation results of
%the \query{WHERE} conditions do not meet the expected outcomes for
%certain records, \tool{} will increase the suspiciousness scores of
%those \query{WHERE} conditions to indicate that they are likely to
%contain faults. Unlike the technique proposed by Clark et
%al.~\cite{ga-ase11}, by monitoring the execution of SQL \query{WHERE}
%conditions, \tool{} can increase the suspiciousness scores of faulty
%conditions in SQL queries even when there exists only one unique SQL
%query per database-interaction point.

\subsection{Monitoring the Execution of SQL Queries}



Figure~\ref{fig:SQL-Transformed} illustrates the execution of a PHP
program with SQL queries by a regular PHP interpreter. It contains the
source code to evaluate various expressions in a PHP program, e.g.
assignments, variables, function calls, etc. Among them, we focus on
the evaluation of the PHP
\code{mysql\_query} statement where database interactions take place
(the shaded part in the PHP interpreter's source code). The
interpreter evaluates the \code{mysql\_query} statement by sending the
SQL query to the database management system (DBMS) and retrieving its
returned result. Since the actual query execution is performed by the
DBMS, the interpreter does not have access to the internal operations
that evaluate the query's \query{WHERE} condition and extract data
records from database. Therefore, to monitor the evaluation of the
expressions in the
\query{WHERE} clause at run time, we instrument the original PHP
interpreter and replace the source code handling database queries with
our instrumented code that performs the query's operations
(Figure~\ref{fig:SQL-Transformed}).



%By evaluating the \query{WHERE} condition instead of delegating the evaluation to the DBMS, the instrumented interpreter now has the evaluation results of the \query{WHERE} expression for individual data records.

The operations that fulfill an SQL \query{SELECT} query consist of the
following: (a) retrieving data from one or more database tables
specified by the \query{FROM} part of the SQL query, (b) extracting
the data records that satisfy the criteria specified by the
\query{WHERE} condition of the SQL query, and (c) projecting the set
of columns (attributes) given in the \query{SELECT} part of query into
the final result set. As an example, the SQL query \code{SELECT}
\code{Name FROM Users WHERE Age $>$= 25 AND Gender = `Female' OR
Country $<>$ `USA'} retrieves the names of all the users from the
\code{`Users'} database table that meet the condition \code{Age $>$=
  25 AND Gender = `Female' OR Country $<>$ `USA'}.


Let us explain the instrumentation for our monitoring
process. Our instrumented interpreter re-implements these three
operations in four steps (shown in the instrumented code in
Figure~\ref{fig:SQL-Transformed}). The detailed instrumentation
execution is illustrated in Figure~\ref{fig:Instrumented-Code}. (In
our implementation, we instrument Quercus
(http://quercus.caucho.\-com/), a PHP interpreter, and use JSqlParser
(http://jsqlparser.source\-forge.\-net/) to parse SQL code. Specifically,
we have the following steps:

%; both are written in Java.)

%~\footnote{In our implementation, \tool{} instruments
%Quercus (http://quercus.caucho.com/), a PHP interpreter, and uses
%JSqlParser (http://jsqlparser.sourceforge.net/) to parse SQL queries;
%both are written in Java.}

\begin{figure*}[tbp]
  \centering
  \includegraphics[width=4.55in]{images/Instrumented-Code-2.eps}\\
  \caption{Instrumented code to monitor the execution of SQL queries~\cite{icsm13}}\label{fig:Instrumented-Code}
\end{figure*}

%\begin{figure*}[tbp]
%  \centering
%  \includegraphics[width=0.8\textwidth]{images/Instrumented-Code-1.eps}\\
%  \caption{Instrumented code to monitor the execution of SQL queries}\label{fig:Instrumented-Code}
%\end{figure*}

\vspace{0.04in}
\textbf{Step 1. Modifying an SQL query:}
The original SQL query is parsed and modified. Its
\query{WHERE} part is removed, and~the column set is changed into \code{`*'}
(i.e., all columns will be retrieved). The \code{ModifiedSql} variable
now contains the modified SQL query, which retrieves all data from one
or more database tables (the SQL query may specify a \query{JOIN}
operation on multiple tables). The \query{WHERE} clause and the column
set in the original SQL query are also extracted out (to the variables
\code{WhereExp} and \code{SelectedCols}, respectively) so that the
rows and columns can be filtered from the modified query's result in
the next steps. In Figure~\ref{fig:Instrumented-Code}, the dashed box
at the top shows an example of an SQL query that is the input of step
1. The values of the three variables \code{ModifiedSql},
\code{WhereExp}, and \code{SelectedCols} are shown in the dashed boxes
coming out of step 1, respectively.

% and will be used to illustrate the following steps.



\vspace{0.04in}
\textbf{Step 2. Executing the modified SQL query:}
The modified SQL query obtained from step 1 will be sent to the DBMS
to be executed there. Compared to the result of the original query,
the result of the modified query contains all data from the database
table(s) in which the rows and columns have not been filtered out
according to the specifications in the original query. In
Figure~\ref{fig:Instrumented-Code}, the dashed box coming out of step
2 (table \code{T}) shows the result after executing the
\code{ModifiedSql} query.

\vspace{0.04in}
\textbf{Step 3. Filtering rows:} In this step, \tool{} loops through
each row \code{R} in table \code{T} and evaluates the expression
\code{WhereExp} in the \query{WHERE} clause of the original SQL query~for row \code{R}.
If a row \code{R} satisfies the condition specified
by \code{WhereExp}, it will be extracted out, in the same way that a
DBMS processes the original SQL query. In
Figure~\ref{fig:Instrumented-Code}, the extracted rows
(\code{FilteredRows}) are the rows 2 and 3 of the \code{Users}
table. Importantly, by evaluating the individual predicates in the
\query{WHERE} expression, \tool{} is able to determine whether the
\query{WHERE} condition evaluates to \code{True} or \code{False} for a
given row.

\vspace{0.04in}
\textbf{Step 4. Filtering columns:} In the final step, \tool{}
extracts the column set specified by the original SQL query from the
data obtained from step 3. In Figure~\ref{fig:Instrumented-Code}, the
final result consists of the names \code{Bob} and \code{Carol},
which preserves the original query's result as if it was executed by
the DBMS.

%\vspace{0.06in}
%\noindent {\bf Evaluation Rules.}

\subsection{Evaluation Rules for Step 3}

Since SQL is a declarative language, when evaluating the SQL
expressions in step 3, \tool{} needs~to understand and implement their
semantics in the interpreter. Table~\ref{tab:transformation-rules}
shows the rules to evaluate an SQL expression $E$ with the data from a
given table row~$R$.

\input{rules.tex}

\textbf{Rules 1 and 2:}
\code{E ::= E1 AND/OR E2}. For an SQL logic expression (e.g. \code{AND}
and \code{OR}), \tool{} evaluates the two sub-expressions
\code{E1} and \code{E2}, and returns the result based on the semantics
of the operator in the expression \code{E}. Other arithmetic
expressions such as addition, subtraction, and comparison expressions
(e.g., \code{`='} and \code{`$>$='}) are realized in a similar
manner.

\textbf{Rule 3:} \code{E ::= E1 BETWEEN E2 AND E3}. \tool{}
evaluates three expressions \code{E1}, \code{E2}, \code{E3}, and
checks the condition that the value of \code{E1(R)} is in the
range of \code{E2(R)} and \code{E3(R)}.

\textbf{Rule 4:} \code{E ::= E' LIKE $Pattern$}. \tool{}
implements the SQL regular expression $Pattern$ by determining if the
evaluation result from \code{E'(R)} matches that pattern.

\textbf{Rule 5:} \code{E ::= $Table$ AS $Alias$}. In an SQL query,
database tables can be referred to via aliases (e.g., the SQL query
\code{SELECT * FROM Users AS u WHERE u.Age $>=$ 25} assigns the alias
name \code{`u'} for the \code{Users} table). Thus, \tool{}
records the real table name of the alias for later reference.

\textbf{Rule 6:} \code{E ::= $Table$.$Column$}. For an SQL expression
to access a table column (e.g., \code{`u.Age'}), \tool{} first checks
if the table name is an alias name, in which case the real table name
is used. Then, \tool{} retrieves the corresponding table column from
the current row \code{R}. If the expression does not specify a table
name, the only table name in the query will be used.

%For example, given SQL query in the motivating example and the row
%\code{(1, Alice, 20, Female, USA)}, the evaluation result for column
%\code{Name} is \code{Alice}.

\textbf{Rule 7:} \code{E ::= E' IN (\{$E_i$\}), $i = 1 \ldots n$}.
\tool{} evaluates \code{E'} and the set of expressions \{$E_i$\}
and determines if the evaluation result of \code{E'} is contained
in the evaluation results of \{$E_i$\}. Note that \{$E_i$\} can be
another SQL \query{SELECT} query.

\textbf{Rule 8:} \code{E ::= EXISTS E'}. \tool{} evaluates \code{E'}
and checks if the result is non-empty (not null). Similar to
rule~7, \code{E'} is often an SQL sub-query.

% which is handled by the next evaluation rule.

\textbf{Rule 9:} \code{E ::= $SubSelect$}. For an expression specifying
a sub-query, \tool{} sends the sub-query to the DBMS to be
executed and retrieves its returned result. It monitors only the
extraction of table rows from the top-level query and transfers the
executions of all sub-queries (if any) to the DBMS.

%Although the SQL query in the motivating example is simple for
%illustrative purposes,

\tool{} is also able to handle more complex SQL
constructs such as \query{JOIN}, \query{ORDER BY}, \query{TOP},
%\query{ALL}, \query{ANY} and SQL functions such as \query{MAX},
\query{LEN}, and \query{UCASE}. It performs these operations by
either re-implementing them or delegating them to the DBMS (step 2 of
Figure~\ref{fig:Instrumented-Code} or rule 9 in Table
\ref{tab:transformation-rules}).

% Currently, \tool{} does not handle \query{UNION}. 


\subsection{Row-based Slicing across PHP and SQL}

\begin{figure}[tbp]
  \centering
  \includegraphics[width=0.95\columnwidth]{images/Execution-Trace-1.eps}\\
  \caption{Execution trace of row-based test cases}\label{fig:Execution-Trace}
\end{figure}



During the above monitoring process, {\tool} also records the PHP
statements and the parts (\code{True} and \code{False}) of
\code{WHERE} clause(s) that are exercised in the execution of a
row-based test case, thus, involved in the output of a data row. It
computes program slices corresponding to each data row.
%To gather more information about the fault, \tool{} considers each
%table row as a test case and computes the program slice corresponding
%to each test case. 
Figures~\ref{fig:Execution-Trace} and ~\ref{fig:Rowbased-Slicing}
illustrate the computation of the slices for the example in
Fig.~\ref{fig:example-phpcode} (with lines 5' and 7 added for
illustrative purposes). Fig.~\ref{fig:Execution-Trace} shows the
trace in the execution of our instrumented interpreter. The
nodes denote the PHP and SQL entities, and the arrows show the order
in the execution trace. Besides PHP statements, {\tool} needs to
consider two parts of a \code{WHERE} clause (see lines 4a and 4b).
%\tool{} introduces a new type of program
%entity, namely the state of the \query{WHERE} condition in an SQL
%query (lines 4a and 4b).
We use \code{1-Alice}, \code{2-Bob}, \code{3-Carol}, and
\code{4-Daniel} to denote the data records in the result set \code{T}
after step 2 in Fig.~\ref{fig:Instrumented-Code}, i.e., after
executing the modified SQL query.

As a PHP entity or SQL part is executed, \tool{} recognizes the
corresponding processed data row and includes that entity/part into
the slice of that row-based test case with its order in the execution
trace. For example, since the \query{WHERE} expression evaluates to
\code{True} for \code{Bob} and \code{Carol}, line 4a is included in
the slices of the test cases corresponding to those data rows
(see Fig.~\ref{fig:Rowbased-Slicing}). Similarly, line 4b is included in
the slices for \code{1-Alice} and \code{4-Daniel}, corresponding to
\code{WhereExp = False}. Since only \code{2-Bob} and \code{3-Carol}
are returned in the query's result, the PHP code on lines 5-6 performs
two iterations to print out each name. Whenever a PHP row-retrieving
function such as \code{mysql\_fetch\_array} is executed and returns a
\emph{non-}\code{null} table row (line 5), \tool{} includes 
in the slice of the corresponding row-based test case the PHP
statements having data dependencies with that row. Thus, lines 5, 5',
and 6 are included in \code{2-Bob}'s and \code{3-Carol}'s slices.
%the slices for \code{2-Bob} and \code{3-Carol}.
%executed twice (for \code{2-Bob} and \code{3-Carol})
%the statements that follow in the slice of the corresponding row-based
%test case. Therefore, statements 5-6 are included in the slices of
%\code{Bob} and \code{Carol}.
As line~5~is executed for \code{1-Alice} and
\code{4-Daniel}, the variable \code{\$row} is \code{null} and the
execution exits the \code{while} loop; i.e., no table row is accessed.
Thus, lines 5' and 6 are not included for \code{1-Alice}'s and
\code{4-Daniel}'s slices, while line 5 is included for all slices.~Since 
line 7 (or any next line) is exercised regardless of
data rows, it will be included in all slices of all test cases.
Similarly, the PHP statements before line 4 are exercised and included
in all slices.

%When a statement is not associated with any particular test case
%(e.g., statements 4, 7, and statement 5 in its last execution),
%\tool{} includes them in the slices of all test cases. The slices for
%these row-based test cases are shown in
%Figure~\ref{fig:Rowbased-Slicing}.

%tbp
\begin{figure}[t]
  \centering
  \includegraphics[width=0.81\columnwidth]{images/Row-based-Slicing-1.eps}\\ %0.84
  \caption{Row-based slicing across PHP and SQL}\label{fig:Rowbased-Slicing}
\end{figure}


\subsection{Computing Suspiciousness Scores}

\begin{figure}[t]
    \centering
    \footnotesize
    \setlength{\tabcolsep}{1pt}
    \renewcommand{\arraystretch}{1.1}
{\sffamily
 \scriptsize
\begin{tabular}{ll@{}ccccl@{}}
    \toprule
                   & & \multicolumn{4}{c}{Row-based test case} & Sus.\\
    \cmidrule{1-1}                                                                  \cmidrule{3-6}
    function display...(\$age, \$gender, \$country) \{                    & & 1-Alice             & 2-Bob             & 3-Carol   & 4-Daniel \\
    \ldots                                                                          & & $\bullet$     & $\bullet$     & $\bullet$     & $\bullet$   & 0.5\\
    4\hspace{5pt}\$result = mysql\_query(\$sql);                                   & & $\bullet$     & $\bullet$     & $\bullet$     & $\bullet$   & 0.5\\
    4a\hspace{30pt}WhereExp* = True             & &               & $\bullet$     & $\bullet$     &    & 0.25\\
    4b\hspace{30pt}WhereExp* = False                                           & & $\bullet$     &               &   & $\bullet$   & \textbf{1.0}\\
    5\hspace{5pt}while(\$row = mysql\_fetch\_array(\$result))\{                   & & $\bullet$  & $\bullet$     & $\bullet$     &  $\bullet$ & 0.5\\
    6\hspace{20pt}echo \$row[`Name'] . `$<$br /$>$';                                    & &         & $\bullet$     & $\bullet$     &  & 0.25\\
    \midrule
    Pass/Fail Status                                                                & & F             & F             & P             & F \\
    \bottomrule
    \addlinespace
    \multicolumn{6}{l}{Test case with \$age = 25, \$gender = `Female', \$country = `USA'}\\
    \multicolumn{6}{l}{* WhereExp: Age $>$= 25 AND Gender = `Female' OR Country $<>$ `USA'}
\end{tabular}
}
    \caption{Suspiciousness scores computed by \tool{} for Fig.~\ref{fig:example-phpcode}}\label{fig:suspiciousness-improved}
\end{figure}

%\begin{figure*}[t]
%    \centering
%    \footnotesize
%    %\setlength{\tabcolsep}{6pt}
%    \renewcommand{\arraystretch}{1.1}
%{\sffamily
% \scriptsize
%\begin{tabular}{llccccl}
%    \toprule
%    Test case with \$age = 25, \$gender = `Female', \$country = `USA'               & & \multicolumn{4}{c}{Row-based test case} & Sus.\\
%    \cmidrule{1-1}                                                                  \cmidrule{3-6}
%    function displaySearchResults(\$age, \$gender, \$country) \{                    & & 1-Alice             & 2-Bob             & 3-Carol   & 4-Daniel \\
%    \ldots                                                                          & & $\bullet$     & $\bullet$     & $\bullet$     & $\bullet$   & 0.5\\
%    4\hspace{10pt}\$result = mysql\_query(\$sql);                                   & & $\bullet$     & $\bullet$     & $\bullet$     & $\bullet$   & 0.5\\
%    4a\hspace{30pt}WhereExp* = True             & &               & $\bullet$     & $\bullet$     &    & 0.25\\
%    4b\hspace{30pt}WhereExp* = False                                           & & $\bullet$     &               &   & $\bullet$   & \textbf{1.0}\\
%    5\hspace{10pt}while(\$row = mysql\_fetch\_array(\$result)) \{                   & & $\bullet$  & $\bullet$     & $\bullet$     &  $\bullet$ & 0.5\\
%    6\hspace{20pt}echo \$row[`Name'] . `$<$br /$>$';                                    & &         & $\bullet$     & $\bullet$     &  & 0.25\\
%    \midrule
%    Pass/Fail Status                                                                & & F             & F             & P             & F \\
%    \bottomrule
%    \addlinespace
%    \multicolumn{6}{l}{* WhereExp: Age $>$= 25 AND Gender = `Female' OR Country $<>$ `USA'}
%\end{tabular}
%}
%    \caption{Suspiciousness scores computed by \tool{} for the PHP function in Figure~\ref{fig:example-phpcode}}\label{fig:suspiciousness-improved}
%\end{figure*}

Based on the monitoring results, \tool{} computes the suspiciousness
scores for all program entities, including the \query{WHERE}
conditions of SQL queries and other statements in the PHP
program. Fig.~\ref{fig:suspiciousness-improved} illustrates the
score computation for the example in
Fig.~\ref{fig:example-phpcode}. A test case is marked as
\code{Passed(P)} if the presence (or absence) of the
corresponding record in the actual output is as expected; otherwise,
it is marked as \code{Failed(F)}. For example, \code{Alice} does not
appear in the actual output, which is not expected; thus record 1 is a
\code{Failed} test case. Similarly, \code{Bob} is included in the
actual output while it should not, thus record 2 is also a
\code{Failed} test case. The only \code{Passed} test case is record 3,
where \code{Carol} is output as expected. The bullets for the 
statements indicate whether the entities are included in the slice for
the corresponding test case, which is established when \tool{}
monitors the execution trace of row-based test cases. 
The column \code{Sus.} shows Tarantula suspiciousness scores. 

%%for the corresponding program entities using Tarantula metric.

% (as explained in the motivating example).

As seen in Fig.~\ref{fig:suspiciousness-improved}, the \code{False}
part of the \query{WHERE} expression on line 4b has a high
suspiciousness score (1.0), indicating that the program state when the
\query{WHERE} of the SQL query is evaluated to \code{False} is likely
incorrect, (i.e., the result set does not contain the corresponding
record whereas the record is expected to be included). Also, the
suspiciousness score of the
\query{WHERE} expression corresponding to the \code{False} case is
higher than the score of any other program entity, which suggests that
the predicates in the \query{WHERE} clause of the SQL query are most
likely to contain an error. In this example, the fault is located at
the last predicate of the \query{WHERE} condition (the operator in
\code{Country $<>$ `\$country'} should be
\code{`='} instead of \code{`$<>$'}). 
The suspiciousness scores computed by \tool{} are, thus, useful in
localizing faulty SQL queries.  

%Next, we will explain how {\tool} localizes faulty predicates in a
%\code{WHERE} clause.




\section{Localizing Faulty Predicates}
% in SQL Queries}

After an SQL query was identified as the likely cause of the
incorrect output, \tool{} continues to examine the predicates in its
\query{WHERE} clause to further localize the fault. 
%Since the incorrect output is caused by the incorrect evaluation
%results of the \query{WHERE} condition for some data record(s),
It records the values of the predicates of the \code{WHERE}
clause during the execution of an SQL query (Step 3, Section
3.1). Then, given a failed test case, it detects the predicates
whose values result in the incorrect output. To do this, it uses
a \emph{predicate switching} technique with the following key idea: If
the boolean value of a predicate after switching (i.e., \code{True}
becomes \code{False}, or vice versa) leads to a different result of
the \query{WHERE} expression, which makes a \emph{failed} test case
become a \emph{passed} one, the original value of the predicate is
likely to be incorrect. That is, such a predicate is highly
associated with the erroneous output. Among all predicates in the
\query{WHERE} clause, those whose switched values change the results
of the most failed test cases will be the most likely ones to contain
the fault(s).

\subsection{Localizing a Faulty Predicate}
\label{single-fault-section}

%s with a Single Fault}

\begin{table}
    \small
    \setlength{\tabcolsep}{1.5pt}
    \centering
    \caption{\tool{}'s fault localization using predicate switching}\label{tab:predicate-table}
\begin{tabular}{clllcccc}
  \addlinespace
  \toprule
  \textbf{Row} & \textbf{P1} & \textbf{P2} & \textbf{P3} & \textbf{P1 AND } & \textbf{Act.} & \textbf{Exp.} & \textbf{Passed/} \\
  &    &   &   & {\bf P2 OR P3}  &  {\bf Output} & {\bf Output} & {\bf Failed}\\ 
  \midrule
  1 & False ($\times$) & True & False ($\times$) & False & (none) & Alice & Failed \\
  2 & False & False & True ($\times$) & True & Bob & (none) & Failed \\
  3 & True & True & True & True & Carol & Carol & Passed \\
  4 & True & False ($\times$) & False ($\times$) & False & (none) & Daniel & Failed \\
  \midrule
  $R$ score & 1 & 1 & 3 \\
  \bottomrule
  \addlinespace
  \multicolumn{8}{l}{P1: Age $>$= 25 \hspace{1em} P2: Gender = `Female' \hspace{1em} P3: Country $<>$ `USA'}
\end{tabular}
\end{table}

%\begin{table*}
%    \footnotesize
%    \centering
%    \caption{Illustration of \tool{}'s fault localization using predicate switching}\label{tab:predicate-table}
%\begin{tabular}{clllcccc}
%  \addlinespace
%  \toprule
%  \textbf{Row} & \textbf{P1} & \textbf{P2} & \textbf{P3} & \textbf{P1 AND P2 OR P3} & \textbf{Act. Output} & \textbf{Exp. Output} & \textbf{Passed/Failed} \\
%  \midrule
%  1 & False ($\times$) & True & False ($\times$) & False & (none) & Alice & Failed \\
%  2 & False & False & True ($\times$) & True & Bob & (none) & Failed \\
%  3 & True & True & True & True & Carol & Carol & Passed \\
%  4 & True & False ($\times$) & False ($\times$) & False & (none) & Daniel & Failed \\
%  \midrule
%  $R$ score & 1 & 1 & 3 \\
%  \bottomrule
%  \addlinespace
%  \multicolumn{8}{l}{P1: Age $>$= 25 \hspace{1em} P2: Gender = `Female' \hspace{1em} P3: Country $<>$ `USA'}
%\end{tabular}
%\end{table*}

Let us explain our technique to localize a faulty predicate and
then our generalization to support the localizing of multiple faulty
ones.  The technique for localizing a single faulty predicate is used
only internally by {\tool}, while the general algorithm is for localizing
any numbers of faulty predicates. Table~\ref{tab:predicate-table}
(called a \emph{predicate table}) illustrates our predicate switching
technique to localize the faulty predicate for the running example.
% in Section 2.
%, assuming that there is \emph{a single fault} in the SQL query. 
The predicate table contains the evaluation results of individual
(atomic) predicates in the \query{WHERE} clause. This information is
recorded at run time as the instrumented PHP interpreter evaluates the
predicates in the
\query{WHERE} condition (see Section~\ref{monitoring-section}). In
Table~\ref{tab:predicate-table}, the three predicates \code{Age $>$=
25}, \code{Gender = `Female'}, and \code{Country $<>$ `USA'} are denoted
by \code{P1}, \code{P2}, and {P3}, respectively. As seen, the
\query{WHERE} expression (\code{P1 AND P2 OR P3}) for rows 2 and 3
evaluates to \code{True}, thus the execution returns
\code{Bob} and \code{Carol} (column \code{Act. Output}). In contrast,
the \query{WHERE} expression for rows 1 and 4 evaluates to
\code{False}, thus \code{Alice} and \code{Daniel} are not included in
the returned result. The columns \code{Exp. Output} and
\code{Passed/Failed} show the expected output and the passed/failed
status of the respective row-based test case.

%If the actual output (column \code{Act. Output}) is the same as the expected output (column \code{Exp. Output}), the corresponding row (test case) will be marked as \code{Passed}; otherwise, it will be marked as \code{Failed} (column \code{Passed/Failed}).

Given the table row of a \emph{failed} test case, \tool{} attempts to
switch the current boolean value for one predicate at a time and
re-evaluates the \query{WHERE} clause with the predicate's new
value. If the value of the \query{WHERE} clause changes (from
\code{True} to \code{False} or vice versa), meaning that the given row
is now a \emph{passed} test case, \tool{} records this event to
compute the likelihood that the predicate contains a fault. In
Table~\ref{tab:predicate-table}, those predicates and their
corresponding rows are marked with the notation $\times$. In the first
row, if either \code{P1} or \code{P3} is switched from \code{False} to
\code{True}, the value of the
\query{WHERE} condition \code{P1 AND P2 OR P3} will change to
\code{True}, and therefore \code{Alice} will be output as expected. 
In contrast, changing the value of \code{P2} from \code{True} to
\code{False} does not affect the original output result. After
applying predicate switching to all predicates and table rows,
\tool{} computes the suspiciousness score $R$ for a predicate by
summing up the total of times its switched value changes a failed test
case~to a passed one. In Table~\ref{tab:predicate-table}, \code{P3}'s
$R$ score is 3 whereas \code{P1} and \code{P2}'s scores are 1.
Thus, the predicate having the highest suspiciousness score
(\code{P3}) is the one likely containing the fault.

%Therefore, the real fault at the predicate \code{P3} can be localized using the \tool{}'s predicate switching technique.

%\begin{figure}[t]
%    \centering
%\begin{lstlisting} [
%    emph={foreach, in, if, then},
%	mathescape=true,
%    xleftmargin=11pt
%]
%$R(P_i) \leftarrow 0, \forall $ Predicate $P_i$ in $WhereExp$
%foreach Failed Test Case $R$
%    foreach Predicate $P_i$ in $WhereExp$
%        $P_i(R)$.switchBooleanValue() // $P_i(R)$ = NOT $P_i(R)$
%        result $\leftarrow$ evaluateExp($WhereExp$, $R$)
%        if result = ExpectedResult($WhereExp$, $R$) then
%            $R(P_i)$ = $R(P_i)$ + 1
%        $P_i(R)$.restoreBooleanValue()
%\end{lstlisting}
%    \caption{Predicate switching algorithm to localize single predicate faults}\label{fig:algorithm-singfaults}
%\end{figure}
%
%Figure~\ref{fig:algorithm-singfaults} summarizes \tool{}'s predicate switching technique to localize the fault in a predicate of an SQL query. For each failed test case $R$ and a predicate $P_i$, \tool{} first switches the current value of $P_i$ at row $R$ (line 4) and re-evaluates the \query{WHERE} expression (line 5). If the value of the \query{WHERE} expression after switching is the same as the expected result for row $R$, \tool{} increases the $R$ score of $P_i$ by one (lines 6-7). The value of $P_i(R)$ will then be restored to its original value before \tool{} repeats the same process on other predicates and rows (line 8).


%%To show the usefulness of \tool{}'s predicate switching technique, 
Let us provide the following theorems to give the theoretical evidence
that for an SQL query with a single predicate fault, the faulty
predicate always has the highest suspiciousness score; moreover, given
sufficient test cases, its score is at least double the scores of
other predicates.

\begin{theorem}
\label{thm1}
{\em Let $P_1$, $P_2$, \ldots, $P_n$ be the predicates in the \query{WHERE}
expression of an SQL query. Suppose that there is a single fault in
one of the predicates, namely $P_*$, then $R(P_*) \ge R(P_i), \forall
i = 1 \ldots n$ and $P_i \neq P_*$.}
\end{theorem}

\begin{proof}
Let $m$ be the number of failed test cases in the predicate table. For
a failed test case, switching the value of the faulty predicate $P_*$
makes the test case passed, since the failed test case is caused by
the fault in $P_*$. Thus, the $R$ score for $P_*$ is increased by
one for every failed test case. Overall, the score of
$P_*$ is $R(P_*) = m$ (1). In contrast, changing the values of other
predicates may or may not affect the original outcome since they are
not the real cause of the error. Thus, the $R$ score of a non-faulty
predicate is either not increased or increased by one for every failed
test case. It follows that $R(P_i) \le m, \forall i = 1 \ldots n$ and
$P_i \neq P_*$ (2). From (1) and (2), we have $R(P_*) \ge
R(P_i), \forall i = 1 \ldots n$ and $P_i \neq P_*$.
\end{proof}

\input{solution-predicatefaults-proof} 


\subsection{Localizing Predicates with Multiple Faults}

\begin{figure}[t]
    \centering
\begin{lstlisting} [
    emph={function, foreach, do, if, else, end},
	mathescape=true
]
function LocalizeFaults(LogicExpression $E$)
    Rewrite $E$ as $E_1 \oplus E_2 \ldots E_n$ // $\oplus$ is either $\wedge$ or $\vee$.
    foreach $E_i$ do
        Rewrite $E$ as $E_i \oplus \overline{E_i}$, with $\overline{E_i} = E \setminus E_i$
        Compute $R(E_i)$ and $R(\overline{E_i})$ by predicate switching
    end
    $E_m = \{E_i: R(E_i) / (R(\overline{E_i}) + 1) \ge \sigma\}$
    if $|E_m|$ > 1
        Report multiple faults $E_m$, then exit
    if $E_m$ contains only one expression $E_s$
        if $E_s$ is atomic
            Report single fault $E_s$, then exit
        else
            LocalizeFaults($E_s$)
\end{lstlisting}
     \caption{Multiple-Fault Localization Algorithm}
     \label{fig:algorithm-multiplefaults}
\end{figure}

%We extend \tool{}'s predicate switching algorithm to work on a more
%general case 

We extended the algorithm in Section IV.A for general cases
where a query may have one or more faults in its
predicates. \tool{} first rewrites the \query{WHERE} condition
in different forms so that the predicates with faults can be grouped
together as one combined predicate. In such cases, the multiple faults
can be seen as one fault in the combined predicate after grouping. It
then applies predicate switching for a single fault to identify the
group of predicates that is most likely to contain the faults.

Fig.~\ref{fig:algorithm-multiplefaults} shows our algorithm.
%%% to localize multiple faults in predicates.
%%%in the \query{WHERE} expression of an SQL query. 
Given a logic expression $E$,
%%% that might contain faults,
\tool{} rewrites $E$ into one of the following forms (line 2):\\
\[
\small
E =
\begin{cases}
    E_1 \wedge E_2 \wedge \ldots E_n, \text{with } n > 1 & \text{(Conjunctive form)} \\
    E_1 \vee E_2 \vee \ldots E_n, \text{with } n > 1 & \text{(Disjunctive form)} \\
\end{cases}
\]
where each $E_i$ is another logic expression. If $E$ is an
\emph{atomic} predicate (i.e., $E$ cannot be written in either
conjunctive or disjunctive form), {\tool} can stop further
localization. For each predicate $E_i$ in $E$, it splits $E$ into two
parts: the first part contains only $E_i$, and the second part
($\overline{E_i}$) contains all the other predicates (line 4). Then,
it computes the score $R$ for $E_i$ and
$\overline{E_i}$ using predicate switching (line 5). If
the faults are all located in the set $\overline{E_i}$, the ratio
$R(E_i)$ over $R(\overline{E_i})$ will be low, indicating that $E_i$
is not likely the fault. Otherwise, a high ratio suggests that $E_i$
might be one of the faults in~$E$.



After computing the suspiciousness scores for all the predicates in
$E$, \tool{} selects the set $E_m$ of the predicates whose ratios
$R(E_i) / (R(\overline{E_i}) + 1)$ exceed a threshold $\sigma$
(line~7). If there are more than one predicate with a high ratio, it
reports the $R$ scores of all the predicates in $E$ and highlights the
predicates in $E_m$ as potentially containing multiple faults (lines
8-9).~If there is only one predicate $E_s$ with a high score, {\tool}
checks whether it is an atomic predicate (line 11). For an
atomic predicate $E_s$, it reports the scores of all the predicates in
$E$ and identifies $E_s$ as the likely single source of the error
(line 12). Otherwise, if $E_s$ can be further decomposed into a
conjunctive or disjunctive form, it recursively calls
\code{LocalizeFaults} to repeat the fault localization for
$E_s$ (line 14).

%invokes the
%recursive function 
  
\section{Empirical Evaluation}

%Let us present our empirical evaluation on \tool{}'s
%accuracy in localizing faults in Web applications. 

We conducted our empirical evaluation on {\tool}'s accuracy in fault
localization with three subject systems from \code{sourceforge.net}
with the sizes ranging from 19 KLOC to 156KLOC
(Table~\ref{tab:subject-systems}).  \code{AddressBook} is an
application to manage contact information and user
groups. \code{SchoolMate} is a web system for managing students,
teachers, and classes in a school. The largest subject system,
\code{ZenCart}, is an e-commerce application. These subject systems
have been used in prior work on fault detection for Web-based
applications~\cite{icsm13,apollo10}.
%All of these applications have interactions with a database through
%SQL queries.

%%%For each system, the number of SQL queries is shown in the column \code{Query}.

%database-interaction points is shown in the last column \code{DB-I}.

%sub

\subsection{Localizing Database-Related Faults}

%\vspace{0.04in}
%\noindent {\bf Experiment Setup.} 

%In the first experiment, for each system, we manually and randomly
%seeded two types of database-related faults: (1) SQL faults in the
%\query{WHERE} clause of SQL queries, and (2) PHP faults that affect
%the output of certain rows retrieved from a database query. Each
%mutant program has a single fault. Then, we created a failed test case
%that was resulted from the fault in either an SQL query or the PHP
%code. Table~\ref{tab:eval-database-aware} shows the result. Column
%\code{Mutants} shows the number of created mutants.
%Column \code{SQL faults/\% Rank} shows the percentage of statements in
%the execution trace that a~fixer need not examine by using
%\tool{}'s ranked list of suspicious statements. For example, for PHP
%faults, (s)he does not have to examine 86-98\% of the statements in
%the execution trace. Since the faulty SQL query is ranked at the top
%by \tool{}, column \code{SQL faults/\% Rank} shows 100\% for all three
%systems.

In the first experiment, we manually 
%and randomly
seeded two types of database-related faults: (1) SQL faults in the
\query{WHERE} clause of SQL queries, and (2) PHP faults that affect
the output data
%of certain rows 
retrieved from a database query. For (1), we mutated the operators of
the predicates in the \query{WHERE} clauses, while for (2), we used
the same mutation strategy as in the experiment in Clark {\em et
al.}~\cite{ga-ase11}. Each mutant program has a single fault. Then, we
created a failed test case that was resulted from the fault in either
an SQL query or the PHP code.

Table~\ref{tab:eval-database-aware} shows the result. Column
\code{Mutants} shows the number of created mutants.  Column \code{SQL
  faults/\% Rank} shows the percentage of statements in the execution
trace that a developer need not examine by using \tool{}'s ranked list of
suspicious statements. For example, for PHP faults, (s)he does not
have to examine 86-98\% of the statements in the execution
trace. Since the faulty SQL query is ranked at the top by \tool{},
column \code{SQL faults/\% Rank} shows 100\% for all three systems.

\begin{table}[t]
    \centering
%    \scriptsize
    \caption{Subject Systems}\label{tab:subject-systems}
\begin{tabular}{@{}lrrrr@{}}
    \toprule
    \textbf{System} & \textbf{Version} & \textbf{Files} & \textbf{LOC} & \textbf{Query} \\
    \midrule
    AddressBook (AB)    & 6.2.12    & 103   & 19K   & 52  \\ %184 (avg LOC) \\
    SchoolMate (SM)     & 1.5.4     & 63    & 50K   & 295 \\ %127 (avg LOC) \\
    ZenCart (ZC)        & 1.3.9     & 1,118  & 156K  & 2,171   \\ %140 (avg LOC) \\
    \bottomrule
\end{tabular}
\end{table}

\begin{table}[t]
    \centering
    \small
    \caption{Database-aware Fault Localization Results}\label{tab:eval-database-aware}
\begin{tabular}{@{}lcccccc@{}}
    \addlinespace
    \toprule
        & \multicolumn{2}{@{}c@{}}{\textbf{SQL faults}} & & \multicolumn{2}{@{}c@{}}{\textbf{PHP faults}} \\
    \cmidrule{2-3} \cmidrule{5-6}
    \textbf{System} & \textbf{Mutants} & \textbf{\% Rank} & & \textbf{Mutants} & \textbf{\% Rank} \\
    \midrule
    AddressBook   & 30  & 100\%     & &  9    & 98\% \\
    SchoolMate    & 54  & 100\%     & & 15    & 86\% \\
    ZenCart       & 91  & 100\%     & & 24    & 90\% \\
    \bottomrule
\end{tabular}
\end{table}

\begin{table}[t]
\centering
\small
\caption{SQL queries with Unique Set of Attributes}
\label{tab:sql}
\begin{tabular}{l|r|r|r|r||r|r}
  \hline
  % after \\: \hline or \cline{col1-col2} \cline{col3-col4} ...
  \#predicates & {\bf 0-1} & {\bf 2-3} & {\bf 4-7} & {\bf 8-10} & {\bf Checked} & {\bf Unique} \\
  \hline
  AddressBook & 17  & 2   & 15 &  0 & 34 & 29 \\
  SchoolMate  & 181 & 25  & 3  &  0 & 36 & 36 \\
  ZenCart     & 755 & 431 & 135 & 8 & 36 & 36 \\
  \hline
  Total       &   &  &  & & 106 & 101 \\
  \hline
\end{tabular}
\end{table}

\vspace{0.05in}
\noindent {\bf Comparison.} 
Since \tool{} uses information on individual rows in the test case,
\tool{} is able to rank the likelihood of faulty entities with one
test case only. In contrast, the state-of-the-art approach, Clark {\em
et al.}~\cite{ga-ase11}, was designed to require multiple test cases
to localize faults. For comparison, we took 106 randomly sampled SQL
queries and manually examined if a query involves a unique set of
attributes and varies only at the literal values. To do that, we first
divided the queries into groups according to the numbers of their
predicates. The number of sampled queries in each group is
proportional to its size. In Table~\ref{tab:sql}, the first columns
show the numbers of SQL queries with the corresponding numbers of
predicates. Columns \code{Checked} and \code{Unique} show the numbers
of queries that were checked and have unique sets of attributes,
respectively. As seen, among 106 random query samples, 101 of them
involve a unique set of attributes. Clark {\em et al.}~\cite{ga-ase11}
could not give those SQL statements higher suspicious scores even with
multiple test cases, thus, could not locate those 101 faults. In
contrast, {\tool} was able to rank all of them at the top position
of the resulting list.


\subsection{Localizing Faults in SQL's Predicates}

%\subsection{Experiment Setup}

%subject systems table

%%\vspace{0.04in}
%%\noindent {\bf Experiment Setup.}

In this experiment, we conducted two more studies. In our first study,
we used the same single-fault mutants for SQL queries as in the
previous experiment.  Each mutant program contains one single fault in
a predicate of a given SQL query. In our second study, each mutant
contains multiple faults.  In each study, we applied our two
algorithms to localize the seeded faults: (1) \tool{}'s single-fault
algorithm to detect SQL predicate faults assuming that there is a
single fault in the query (Section~\ref{single-fault-section}), and
(2) \tool{}'s multi-fault algorithm to detect SQL predicate faults
with no prior assumption about the number of faults
(Section~\ref{multi-fault-section}). The output of each algorithm is a
ranked list of the predicates in the SQL query, with the
highest-ranked predicate being the most likely one to contain the
fault. To evaluate {\tool}'s accuracy, we counted the number of times
that a faulty predicate appears in the top-ranked list of faulty
predicates returned by {\tool}.

\paragraph{Results on Localizing Single Faults}

%\vspace{0.04in} {\bf A. Results on Localizing Single Faults}

Table~\ref{tab:eval-single-faults} displays the evaluation result of
our single-fault and multi-fault localization algorithms. Column
\code{\#M} gives the number of mutants with single faults for each
system. Under column \code{Single-fault ranks}, the values in the five
sub-columns show the number of times the seeded fault appears in the
first to the fifth position of the resulting ranked list of
\tool{}'s single-fault algorithm.
%%Since the SQL queries that we used in our study have from 2-5
%%predicates, we
Table~\ref{tab:eval-single-faults} shows the top-1 to top-5 results.
In \code{AddressBook}, \tool{} correctly localized 27 out
of the 30 seeded faults with a single recommendation. With a ranked
list of 3 recommendations, it can correctly locate all 30 faults.

%one fault is ranked second in the list and two other faults appear
%in the third position.

\begin{table}[t]
    \centering
    \small
    \caption{Results on Single Seeded Faults in SQL queries}\label{tab:eval-single-faults}
    \setlength{\tabcolsep}{3.7pt}
\begin{tabular}{@{}lcccccccccccc@{}}
    \toprule
        &   & \multicolumn{5}{@{}c@{}}{\textbf{Single-fault ranks}} & & \multicolumn{5}{@{}c@{}}{\textbf{Multi-fault ranks}} \\
    \cmidrule{3-7} \cmidrule{9-13}
    \textbf{Sys.} & \textbf{\#M} & \textbf{1} & \textbf{2} & \textbf{3} & \textbf{4} & \textbf{5} & & \textbf{1} & \textbf{2} & \textbf{3} & \textbf{4} & \textbf{5} \\
    \midrule
    AB     & 30     & 27    & 1     & 2     & -     & -     &   & 27    & 1     & 2     & -     & - \\
    SM     & 54     & 43    & 11    & -     & -    & -     &   & 41    & 12    & 1     & -     & - \\
    ZC     & 91     & 77    & 8     & 4     & 2     & -     &   & 77    & 8     & 4     & 2     & - \\
    \midrule
%    \%     &        & 84    & 11    & 3     & 1     & 0     &   & 83    & 12    & 4     & 1     & 0 \\
    \multicolumn{2}{c}{\% coverage}  & 84    & 96    & 99     & 100     & 100     &   & 83    & 95    & 99     & 100     & 100 \\
    \bottomrule
\end{tabular}
\end{table}

Column \code{Multi-fault ranks} (Table~\ref{tab:eval-single-faults})
shows the number of times a fault appears in the corresponding
position in the ranked list of \tool{}'s multi-fault
algorithm.
%As seen, the results are {\em comparable} to those produced by the
%single-fault algorithm.
In \code{SchoolMate}, the single-fault algorithm performed slightly
better than multiple-fault algorithm.
%This is because the multi-fault algorithm does not have the prior
%knowledge that there is only one single fault in the program, while
%the other algorithm is specialized toward single faults.
The last row in Table~\ref{tab:eval-single-faults} (\code{\%
coverage}) shows the percentage the faults that are covered by the
corresponding top-ranked list. Overall, around 84\% of the seeded
faults are correctly identified by \tool{} with a single
recommendation.  The majority of the faults (96\%) can be found in the
top-2 results from two algorithms.

\paragraph{Results on Localizing Multiple Faults}

\begin{table}[t]
    \centering
%    \scriptsize
    \caption{Results on Multiple Seeded Faults in SQL queries}\label{tab:eval-multi-faults}
\begin{tabular}{@{}lcccccccccc@{}}
    \toprule
        &   & \multicolumn{4}{@{}c@{}}{\textbf{Single-fault ranks}} & & \multicolumn{4}{@{}c@{}}{\textbf{Multi-fault ranks}} \\
    \cmidrule{3-6} \cmidrule{8-11}
    \textbf{Sys.} & \textbf{\#M} & \textbf{2} & \textbf{3} & \textbf{4} & \textbf{5} & & \textbf{2} & \textbf{3} & \textbf{4} & \textbf{5} \\
    \midrule
    AB     & 14     & 9     & 3     & 2     & 0     &   & 10     & 2     & 2     & 0 \\
    SM     & 27     & 13    & 12    & 2     & 0     &   & 18     & 8     & 1     & 0 \\
    ZC     & 38     & 19    & 9     & 2     & 8     &   & 19     & 9     & 2     & 8 \\
    \midrule
%    \%     &        & 52    & 30    & 8     & 10    &   & 59    & 24     & 6     & 10 \\
    \multicolumn{2}{c}{\% coverage}  & 52    & 82    & 90     & 100    &   & 60    & 84     & 90     & 100 \\
    \bottomrule
\end{tabular}
\end{table}

To evaluate how well \tool{} performs on multiple seeded faults, we
conducted another study in which we seeded two errors for each
mutant. Table~\ref{tab:eval-multi-faults} shows the results.
% of \tool{}'s fault localization.
The rows and columns are similar to those in
Table~\ref{tab:eval-single-faults} except that column \code{top-1} is
not applicable since there are two errors. We consider that the faults
in a mutant are localized in the top-$n$ resulting list if the top-$n$
list covers both faulty predicates, and one of the faulty predicates
is ranked at the $n$ position. As seen, in \code{SchoolMate},
\tool{}'s multi-fault algorithm performed better than the single-fault
algorithm. Overall, 60\% of mutants have their faults detected in the
top-2 list, and \tool{}'s top-3 list covers both seeded faults for
84\% of mutant programs.

\noindent {\bf Time Efficiency.} In our experiments, we also
measured running time for the localization of each faulty
program. Each run took less than 1s without counting database
accessing time.

%84\% of the faults.

%In our implementation, we chose $Sus$ to be the Tarantula metric~\cite{tarantula05}. $Sus$ can be any other suspiciousness metric used in existing coverage-based statistical fault localization methods.
%Threshold $\sigma$ = ?


\section{Related Work}

A related work to {\tool} is from Clark {\em et al.}~\cite{ga-ase11},
%who introduced the first, state-of-the-art technique for 
In their {\em database-aware} fault localization for Web applications,
they extended Tarantula~\cite{tarantula05} to monitor at run-time the
host program's statements, SQL statements and attributes.
%They extend the idea of the statistical fault localization method in
%Tarantula~\cite{tarantula05} to monitor at runtime the host program's
%statement, the SQL statements, and SQL attribute tuples, and to assign
%them suspicious scores. A static SQL statement in the host program
%will be executed and become multiple SQL queries in different
%executions. 
They assume that those multiple SQL queries must be involved with a
different set of database attributes. As explained in Sections 1-2,
it is common in practice that those multiple SQL queries at run-time
are involved with a fixed set of attributes, and they vary only at the
literal values.
%If one of those unique SQL queries with a unique set of attributes is
%executed by more failing test cases than passing ones, it more likely
%contributes to the fault and is assigned higher score. However, as
%shown in Section~2, it is common that those multiple SQL queries at
%run-time are involved with a fixed set of database attributes, and
%they vary only at the literal values. 
Because there is one unique set of attributes, their method would not
be able to assign the higher score for that SQL statement (Sections
1-2).  {\tool} is also related to SQLook~\cite{icsm13}. However, there
are significant advances from {\tool} over SQLook. First, while SQLook
is focused only on bugs in PHP, {\tool} also identifies faulty
predicates in an SQL query after deciding that a WHERE clause is at
fault. Second, we peformed an empiricalstudy on SQL-related bugs in
PHP to motivate our approach. Third, we developed a novel row-based
slicing across PHP and SQL. Finally, we conducted a more systematic
evaluation in detecting faulty SQL predicates.

ALTAR~\cite{altar-icst17} combines row-based dynamic slicing and delta
debugging to localize faults in SQL predicates. They introduce a
exoneration-based technique to isolate individual faulty clauses
within WHERE predicates. It uses row-based dynamic slicing to discover
suspicious clauses in WHERE predicates and removes non-faulty
clauses using a technique inspired by delta debugging. Later version
of ALTAR~\cite{altar-jss19} improves the exoneration-based technique
to make it more time efficient. In comparison, {\tool} uses predicate
switching that could be more effective than delta debugging for
predicates. ALTAR does not localize bugs within PHP code.

There exist fault-localization methods for single-language,
{\em data-centric} programs~\cite{dor-issta08,litvak10,saha11}, which
primarily interact with databases to get the data contents, process
and output them. Saha {\em et al.}~\cite{saha11} introduced a method
to localize faults in programs in ABAP (an SQL-like language). In
{\tool}, we apply the idea of using individual rows as test cases as
in their method.
%They developed a {\em key-based} slicing algorithm to break the trace
%into multiple slices where each slice maps to an entry in the output
%collection. That is, it takes into account the relevance of execution
%of statements to specific database keys that were used in creating
%those slices.  Then {\em semantic difference} between the slices that
%correspond to correct entries and those that correspond to incorrect
%ones will be computed to identify fault statements. 
However, they do not aim to support {\em multi-lingual} programs with
database code embedded in a host program. 
%
That method belongs to a class of techniques using dynamic
slicing~\cite{dor-issta08,litvak10,zhang-pldi04,cleve-icsm10}.
The fault is likely to be in the statements in the
program slice of incorrectly computed variables.
%%%, but not in the ones of correctly computed variables.
%agrawal-90, cleve-wcre08
%Hainaut and Cleve~\cite{cleve-icsm10,cleve-wcre08} capture dynamic
%traces via dynamic slicing and analysis on data-centric, embedded code
%and SQL logs in order to resolve the input queries being passed to a
%database in data-intensive systems. Their goal is more toward program
%evolution comprehension and reverse engineering. Other approaches for
%fault-localization in data-centric applications utilize static
%slicing. 
%PanayaAI~\cite{dor-issta08} computes statically a forward slice from
%select statements and infers how the returned data is manipulated in
%an ABAP program. 
Field-sensitive analysis is used to compute program dependence
in ERP system domain~\cite{litvak10}.  Other methods rely on
the differences between passing and failing
runs~\cite{dallmeier-ecoop05,mani-ase10}.

{\tool} is extended to support for SQL from statistical
fault localization
methods~\cite{tarantula05,abreu-ochiai-07,liblit-pldi05}.
%sober05, jones02
%The idea is that the counts of the exercising of different code
%elements in passing and failing runs can help localize faults. 
Such spectrum can be based on several program features such as branch,
complete-path, data-dependence, output, and execution
trace~\cite{harrold00,santelices09,zhang-fse09,apollo-tse10},
%apollo10
%Other techniques analyze 
statistical correlations between failure and CFG's
predicates~\cite{liblit-pldi05}, paths~\cite{holmes-icse09},
time spectra~\cite{timespectra08}, code
changes~\cite{stoerzer-fse06}. They are not database-aware.

%ren-issta07

%roposed an approach to localize faults via automated predicate
%switching. To localize a bug, searching for arbitrary changes to the
%program state is expensive due to the extremely large search
%space. Their idea is that the outcome of a branch is only true/false,
%the number of modified states resulting by predicate switching is far
%less than those possible through arbitrary state
%changes~\cite{zhang-icse06}. They showed that

Other methods modify program states at selected points in the
execution of a failing test. If the fault still occurs, it is likely
the bug~\cite{cleve-icse05,zhang-icse06,jeffrey-issta08}.  Zhang {\em
et al.}~\cite{zhang-icse06} localize faults by switching a predicate's
outcome at runtime. 
%%They showed that to do so can bring the execution to successful, thus,
%%help localize a bug. 
{\tool} localizes faults in the predicates of
a \code{where} statement. 
However, their method cannot handle the cases of multiple faults.
%{\tool} can also handle the cases of multiple faults in those
%predicates. 
Mutation testing~\cite{andrews06} has been applied to modify
an SQL program to assess the adequacy of a test suite~\cite{tuya07}.

% and to study the path divergence of faulty programs~\cite{tuya07}.

%do06, wong10

%tuya06

\section{Conclusions}

%This paper presents {\tool}, 

Lcalizing faults in a dynamic Web application is challenging due to
its dynamic nature and the interactions between the application and
the databases. In this paper, we present {\tool}, our database-aware
fault localization method for Web programs, is able to detect output
faults in PHP statements and in the predicates of \code{WHERE} clauses
in SQL queries. In {\tool}, an PHP interpreter is instrumented to
execute the query and monitor the evaluation of the predicates to
decide if they affect the output of individual records. It performs
row-based slicing across PHP and SQL to record the entities exercised
in the output process of each row. Finally, {\tool}'s predicate
switching identifies suspicious predicates. Our empirical evaluation
shows that {\tool} can achieve higher accuracy than the
state-of-the-art approach. For single-fault scenarios, around 84\% of
the seeded faults are correctly identified by {\tool} with a single
recommendation. For multiple-fault scenarios, 60\% of the cases have
their faults detected in the top-2 list.


% and help in reducing the debugging efforts.




% For peer review papers, you can put extra information on the cover
% page as needed:
% \ifCLASSOPTIONpeerreview
% \begin{center} \bfseries EDICS Category: 3-BBND \end{center}
% \fi
%
% For peerreview papers, this IEEEtran command inserts a page break and
% creates the second title. It will be ignored for other modes.
\IEEEpeerreviewmaketitle



%\section{Introduction}
% no \IEEEPARstart
%This demo file is intended to serve as a ``starter file''
%for IEEE conference papers produced under \LaTeX\ using
%IEEEtran.cls version 1.8b and later.
% You must have at least 2 lines in the paragraph with the drop letter
% (should never be an issue)
%I wish you the best of success.

%\hfill mds
 
%\hfill August 26, 2015

%\subsection{Subsection Heading Here}
%Subsection text here.


%\subsubsection{Subsubsection Heading Here}
%Subsubsection text here.


% An example of a floating figure using the graphicx package.
% Note that \label must occur AFTER (or within) \caption.
% For figures, \caption should occur after the \includegraphics.
% Note that IEEEtran v1.7 and later has special internal code that
% is designed to preserve the operation of \label within \caption
% even when the captionsoff option is in effect. However, because
% of issues like this, it may be the safest practice to put all your
% \label just after \caption rather than within \caption{}.
%
% Reminder: the "draftcls" or "draftclsnofoot", not "draft", class
% option should be used if it is desired that the figures are to be
% displayed while in draft mode.
%
%\begin{figure}[!t]
%\centering
%\includegraphics[width=2.5in]{myfigure}
% where an .eps filename suffix will be assumed under latex, 
% and a .pdf suffix will be assumed for pdflatex; or what has been declared
% via \DeclareGraphicsExtensions.
%\caption{Simulation results for the network.}
%\label{fig_sim}
%\end{figure}

% Note that the IEEE typically puts floats only at the top, even when this
% results in a large percentage of a column being occupied by floats.


% An example of a double column floating figure using two subfigures.
% (The subfig.sty package must be loaded for this to work.)
% The subfigure \label commands are set within each subfloat command,
% and the \label for the overall figure must come after \caption.
% \hfil is used as a separator to get equal spacing.
% Watch out that the combined width of all the subfigures on a 
% line do not exceed the text width or a line break will occur.
%
%\begin{figure*}[!t]
%\centering
%\subfloat[Case I]{\includegraphics[width=2.5in]{box}%
%\label{fig_first_case}}
%\hfil
%\subfloat[Case II]{\includegraphics[width=2.5in]{box}%
%\label{fig_second_case}}
%\caption{Simulation results for the network.}
%\label{fig_sim}
%\end{figure*}
%
% Note that often IEEE papers with subfigures do not employ subfigure
% captions (using the optional argument to \subfloat[]), but instead will
% reference/describe all of them (a), (b), etc., within the main caption.
% Be aware that for subfig.sty to generate the (a), (b), etc., subfigure
% labels, the optional argument to \subfloat must be present. If a
% subcaption is not desired, just leave its contents blank,
% e.g., \subfloat[].


% An example of a floating table. Note that, for IEEE style tables, the
% \caption command should come BEFORE the table and, given that table
% captions serve much like titles, are usually capitalized except for words
% such as a, an, and, as, at, but, by, for, in, nor, of, on, or, the, to
% and up, which are usually not capitalized unless they are the first or
% last word of the caption. Table text will default to \footnotesize as
% the IEEE normally uses this smaller font for tables.
% The \label must come after \caption as always.
%
%\begin{table}[!t]
%% increase table row spacing, adjust to taste
%\renewcommand{\arraystretch}{1.3}
% if using array.sty, it might be a good idea to tweak the value of
% \extrarowheight as needed to properly center the text within the cells
%\caption{An Example of a Table}
%\label{table_example}
%\centering
%% Some packages, such as MDW tools, offer better commands for making tables
%% than the plain LaTeX2e tabular which is used here.
%\begin{tabular}{|c||c|}
%\hline
%One & Two\\
%\hline
%Three & Four\\
%\hline
%\end{tabular}
%\end{table}


% Note that the IEEE does not put floats in the very first column
% - or typically anywhere on the first page for that matter. Also,
% in-text middle ("here") positioning is typically not used, but it
% is allowed and encouraged for Computer Society conferences (but
% not Computer Society journals). Most IEEE journals/conferences use
% top floats exclusively. 
% Note that, LaTeX2e, unlike IEEE journals/conferences, places
% footnotes above bottom floats. This can be corrected via the
% \fnbelowfloat command of the stfloats package.

\section*{Acknowledgments}
This work was supported in part by the US National Science Foundation
(NSF) grants CCF-1723215, CCF-1723432, TWC-1723198, CCF-1518897, and
CNS-1513263.

\newpage

\balance

\bibliographystyle{IEEEtran}

%abbrv

\bibliography{ccf12,babelref}


%\section{Conclusion}
%The conclusion goes here.





% conference papers do not normally have an appendix


% use section* for acknowledgment
%\section*{Acknowledgment}
%The authors would like to thank...





% trigger a \newpage just before the given reference
% number - used to balance the columns on the last page
% adjust value as needed - may need to be readjusted if
% the document is modified later
%\IEEEtriggeratref{8}
% The "triggered" command can be changed if desired:
%\IEEEtriggercmd{\enlargethispage{-5in}}

% references section

% can use a bibliography generated by BibTeX as a .bbl file
% BibTeX documentation can be easily obtained at:
% http://mirror.ctan.org/biblio/bibtex/contrib/doc/
% The IEEEtran BibTeX style support page is at:
% http://www.michaelshell.org/tex/ieeetran/bibtex/
%\bibliographystyle{IEEEtran}
% argument is your BibTeX string definitions and bibliography database(s)
%\bibliography{IEEEabrv,../bib/paper}
%
% <OR> manually copy in the resultant .bbl file
% set second argument of \begin to the number of references
% (used to reserve space for the reference number labels box)
%\begin{thebibliography}{1}

%\bibitem{IEEEhowto:kopka}
%H.~Kopka and P.~W. Daly, \emph{A Guide to \LaTeX}, 3rd~ed.\hskip 1em plus
%  0.5em minus 0.4em\relax Harlow, England: Addison-Wesley, 1999.

%\end{thebibliography}




% that's all folks
\end{document}


