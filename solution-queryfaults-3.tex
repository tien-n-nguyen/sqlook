\subsection{Computing Suspiciousness Scores}

\begin{figure}[t]
    \centering
    \setlength{\tabcolsep}{1pt}
    \renewcommand{\arraystretch}{1.1}
{\sffamily
 \footnotesize
\begin{tabular}{ll@{}ccccl@{}}
    \toprule
                   & & \multicolumn{4}{c}{Row-based test case} & Sus.\\
    \cmidrule{1-1}                                                                  \cmidrule{3-6}
    function display...(\$age, \$gender, \$country) \{                    & & 1-Alice             & 2-Bob             & 3-Carol   & 4-Daniel \\
    \ldots                                                                          & & $\bullet$     & $\bullet$     & $\bullet$     & $\bullet$   & 0.5\\
    4\hspace{5pt}\$result = mysql\_query(\$sql);                                   & & $\bullet$     & $\bullet$     & $\bullet$     & $\bullet$   & 0.5\\
    4a\hspace{30pt}WhereExp* = True             & &               & $\bullet$     & $\bullet$     &    & 0.25\\
    4b\hspace{30pt}WhereExp* = False                                           & & $\bullet$     &               &   & $\bullet$   & \textbf{1.0}\\
    5\hspace{5pt}while(\$row = mysql\_fetch\_array(\$result))\{                   & & $\bullet$  & $\bullet$     & $\bullet$     &  $\bullet$ & 0.5\\
    6\hspace{20pt}echo \$row[`Name'] . `$<$br /$>$';                                    & &         & $\bullet$     & $\bullet$     &  & 0.25\\
    \midrule
    Pass/Fail Status                                                                & & F             & F             & P             & F \\
    \bottomrule
    \addlinespace
    \multicolumn{6}{l}{Test case with \$age = 25, \$gender = `Female', \$country = `USA'}\\
    \multicolumn{6}{l}{* WhereExp: Age $>$= 25 AND Gender = `Female' OR Country $<>$ `USA'}
\end{tabular}
}
    \caption{Suspiciousness scores computed by \tool{} for Fig.~\ref{fig:example-phpcode}}\label{fig:suspiciousness-improved}
\end{figure}

%\begin{figure*}[t]
%    \centering
%    \footnotesize
%    %\setlength{\tabcolsep}{6pt}
%    \renewcommand{\arraystretch}{1.1}
%{\sffamily
% \scriptsize
%\begin{tabular}{llccccl}
%    \toprule
%    Test case with \$age = 25, \$gender = `Female', \$country = `USA'               & & \multicolumn{4}{c}{Row-based test case} & Sus.\\
%    \cmidrule{1-1}                                                                  \cmidrule{3-6}
%    function displaySearchResults(\$age, \$gender, \$country) \{                    & & 1-Alice             & 2-Bob             & 3-Carol   & 4-Daniel \\
%    \ldots                                                                          & & $\bullet$     & $\bullet$     & $\bullet$     & $\bullet$   & 0.5\\
%    4\hspace{10pt}\$result = mysql\_query(\$sql);                                   & & $\bullet$     & $\bullet$     & $\bullet$     & $\bullet$   & 0.5\\
%    4a\hspace{30pt}WhereExp* = True             & &               & $\bullet$     & $\bullet$     &    & 0.25\\
%    4b\hspace{30pt}WhereExp* = False                                           & & $\bullet$     &               &   & $\bullet$   & \textbf{1.0}\\
%    5\hspace{10pt}while(\$row = mysql\_fetch\_array(\$result)) \{                   & & $\bullet$  & $\bullet$     & $\bullet$     &  $\bullet$ & 0.5\\
%    6\hspace{20pt}echo \$row[`Name'] . `$<$br /$>$';                                    & &         & $\bullet$     & $\bullet$     &  & 0.25\\
%    \midrule
%    Pass/Fail Status                                                                & & F             & F             & P             & F \\
%    \bottomrule
%    \addlinespace
%    \multicolumn{6}{l}{* WhereExp: Age $>$= 25 AND Gender = `Female' OR Country $<>$ `USA'}
%\end{tabular}
%}
%    \caption{Suspiciousness scores computed by \tool{} for the PHP function in Figure~\ref{fig:example-phpcode}}\label{fig:suspiciousness-improved}
%\end{figure*}

Based on the monitoring results, \tool{} computes the suspiciousness
scores for all program entities, including the \query{WHERE}
conditions of SQL queries and other statements in the PHP
program. Figure~\ref{fig:suspiciousness-improved} illustrates the
score computation for the example in
Figure~\ref{fig:example-phpcode}. A test case is marked as
\code{Passed(P)} if the presence (or absence) of the
corresponding record in the actual output is as expected; otherwise,
it is marked as \code{Failed(F)}. For example, \code{Alice} does not
appear in the actual output, which is not expected; thus record 1 is a
\code{Failed} test case. Similarly, \code{Bob} is included in the
actual output while it should not, thus record 2 is also a
\code{Failed} test case. The only \code{Passed} test case is record 3,
where \code{Carol} is output as expected. The bullets for the 
statements indicate whether the entities are included in the slice for
the corresponding test case, which is established when \tool{}
monitors the execution trace of row-based test cases. 
The column \code{Sus.} shows Tarantula suspiciousness scores. 

%%for the corresponding program entities using Tarantula metric.

% (as explained in the motivating example).

As seen in Figure~\ref{fig:suspiciousness-improved}, the \code{False}
part of the \query{WHERE} expression on line 4b has a high
suspiciousness score (1.0), indicating that the program state when the
\query{WHERE} of the SQL query is evaluated to \code{False} is likely
incorrect, (i.e., the result set does not contain the corresponding
record whereas the record is expected to be included). Also, the
suspiciousness score of the
\query{WHERE} expression corresponding to the \code{False} case is
higher than the score of any other program entity, which suggests that
the predicates in the \query{WHERE} clause of the SQL query are most
likely to contain an error. In this example, the fault is located at
the last predicate of the \query{WHERE} condition (the operator in
\code{Country $<>$ `\$country'} should be
\code{`='}). 
The suspiciousness scores computed by \tool{} are, thus, useful in
localizing faulty SQL queries.  

%Next, we will explain how {\tool} localizes faulty predicates in a
%\code{WHERE} clause.
